\chapter{Usability Testing} \label{bilag:utestbilag}
The methods used to develop the test plan are based on \citep[p.~65-91]{HandbookofUsabilityTesting}.

\section{Research Questions}
The research questions are used to support observations to gain a better understanding of the test subject's point of view regarding the current state of the application.
However it is not necessarily to be answered.

\begin{itemize}
	\item Are user able to create a document and add a version effortlessly?
	\item Does user understand the Approval functionality in the application?
	\item How well does the flow of the application reflects the expected work flow of user?
	\item How easily and successfully do user navigate through the application?
	\item What obstacles does user meet when handling common task?
	\item Are the symbols and icons understandable and intuitive?
	\item What are user impression of the design?
\end{itemize}

\section{Method Description}
The usability testing will be perform in stages as described below.

\subsection{Session Outline And Timing}
The length of the test session is estimated to be one hour. The first 5 minutes will be used to introduce the test subject the usability testing structure and the think aloud principle and 20 minutes in debriefing session.

\textbf{Introduction to the session}
Test monitor will introduce and explain to test subject how the test going to be conducted, how recording system works and the reason using these and the importance of the thinking aloud method during the test.

\textbf{Tasks}
Test subject attempts to complete the prepared task list while test monitor observes user interaction with the application.
Test subject has an option to ask test monitor for clarification if needed however test monitor will not help solving tasks.

\textbf{Debriefing session}
Post-test questions will be asked to gain a better insight into user's preference and overall experience with the application.
The debriefing session will also allow discussion to take place if desired.

\section{Task List (Second Usability Test)} \label{sec:utest2tasklist}
To be able perform a optimal usability test a task list has been prepared beforehand.
The test subject will be asked to think aloud while solving the task.
The task list is formulated in in Danish as seen below.
 %The usability testing would be screen recorded and the test subject's interaction with the mouse and the keyboard will be recorded as well.

Forestil dig at du arbejder som kvalitetschef hos et firma, der producerer boller.
Du har ansvaret for firmaets håndbog og dens dokumenter.
I firmaet er der ca. 50 medarbejdere i forskellige afdelinger, som hver især skal læse forskellige dokumenter i håndbogen.
Det er din opgave at opdatere dokumenterne og sørge for, at de rigtige medarbejdere læser de relevante dokumenter.
I har netop lanceret en produktion af tranebærboller.

\textbf{Setting:}
Du logger ind på systemet som administrator
På computeren i mappen “Desktop/Håndbøger” ligger “KanLauritzenLevereTranebær.pdf”  og “LauritzenKanLevereTranebær.pdf” af et dokument som skal indføres i håndbogen.

\textbf{Startup:}
\begin{enumerate}
	\item Log ind:
		\begin{itemize}
			\item Email: ips@qmsconsult.dk
			\item Kodeord: K0d30rd   (hvor 0 = nul)
		\end{itemize}

\textbf{Manage documents:}
	\item Tilføj et dokument med titlen “tranebærindkøb”. Placér det i et passende kapitel.
	\item Tilføj første version af dette dokument, “KanLauritzenLevereTranebær.pdf”-
	\item Brug dine administratorrettigheder til selv at godkende første version.
	\item Gå ind og læs dokumentet.
	\item Opdatér dokumentet med en ny version, “LauritzenKanLevereTranebær.pdf”.
	\item Sæt Egon Olsen som godkender af dette dokument og send det til godkendelse.

\textbf{Manage users \& departments:}
	\item Benny Frandsen er blevet ansat. Tilføj ham i systemet.
		\begin{itemize}
			\item Brugernavn: Benny Frandsen
			\item Email: bf@OB.dk
			\item Kodeord: K0d30rd
			\item Rolle: Reader
		\end{itemize}
	\item Tilføj en “Franz Jäger” afdeling.
	\item Tilføj Benny til “Franz Jäger” afdelingen.
	\item Tilføj dokumentet “tranebærindkøb” til “Franz Jäger” afdelingen.

\textbf{Adgang til arkivet:}
	\item I bliver nu auditeret og skal fremvise Jeres dokumentation for tranbeærindkøbene. Find dokumenterne i arkivet.
\end{enumerate}


\section{Task List (Fourth Usability Test)} \label{sec:utest4tasklist}

Du starter ud som en produktionsmedarbejder ved navn Kjeld.
\begin{enumerate}
	\item Log ind
		\begin{itemize}
			\item Brugernavn: keld
			\item Kodeord: K0d30rd  (Bemærk: at det nul begge steder)
		\end{itemize}
	\item Åbn dokument 7.17 ‘Instruktion metaldetektering’.
	\item Find og læs dokumentets changelog.
	\item Du kan se at du har en notifikation om et opdateret dokument. Klik på den.
	\item Du er træt af at nogen har stavet dit navn forkert, ændre dit brugernavn til Kjeld.
	\item Log ud.

Du er nu en afdelingsleder ved navn Yvonne.
	\item Log ind
		\begin{itemize}
			\item Brugernavn: yvonne
			\item Kodeord: K0d30rd  (Bemærk: at det nul begge steder)
		\end{itemize}
	\item Åbn dokument 7.17 ‘Instruktion metaldetektering’.
	\item Tjek hvem der har læst dokumentet.
	\item Luk 7.17 og find i stedet 8.22 ‘Metaldetektorkontrol m. Testpind’
	\item Der er kommet nye testpinde og den på 3mm er nu pink:
		\begin{enumerate}
			\item Find filen på computeren under “Dokumenter/JBJ/”
			\item Ret de nødvendige oplysninger i filen.
			\begin{enumerate}
				\item Husk at rette i sidehovedet også.
			\end{enumerate}
		\item Eksporter filen som PDF.
		\item Upload den nye PDF version til håndbogen.
			\begin{enumerate}
				\item Husk at uploade både PDF- og excel-fil.
				\item Sæt både dig selv og “admin@admin.com” på til at godkende.
			\end{enumerate}
		\end{enumerate}
	\item Opret et nyt dokument med ID: “8.8” og navn “Påklædning i produktionen”.
	\item Lad første version af dette dokument være [en eller anden fil der er tilgængelig på computeren].
	\item Godkend begge dokumenter.
	\item Log ud.

Du er nu en administrator af systemet ved navn [Fortroligt].
	\item Log ind
		\begin{itemize}
			\item Brugernavn: admin
			\item Kodeord: Testp@55word
		\end{itemize}
	\item Kjeld har bedt om at få sit navn ændret i systemet. Gå ind og ændre det fra “Keld” til “Kjeld” .
	\item Gå ind og godkend Yvonnes dokument (8.22).
	\item Find frem til den gamle (tidligere) version af dokument (8.22).
	\item Egon Olsen er atter blevet fængslet og i den anledning fyret. Slet ham fra systemet.
	\item Til at fylde Egon Olsens sko ansætter du “Mette Suhr”. 	Tilføj hende til systemet og giv hende rollen writer.
Undlad at udfylde email.
		\begin{enumerate}
			\item Mette har nu et standard password (123456) som hun kan logge ind med. Dette er en information, ikke en del af opgaven.
		\end{enumerate}
	\item Tilføj en ny afdeling ved navn “Osteskærer-forbundet”.
	\item Forbind dokumenterne 4.1, 6.4 og 7.17 med afdelingen “Osteskærer-forbundet”.
	\item Forbind brugerne Mette Suhr, Yvonne og Kjeld med afdelingen “Osteskærer-forbundet”.
	\item Download et backup af arkivet.
	\item Have fun. Leg. Ødelæg det hele.
\end{enumerate}
