\documentclass[a4paper,11pt,fleqn,dvipsnames,twoside,openright,sort&compress]{memoir} 	% Openright aabner kapitler paa hoejresider (openany begge)

%%%% PACKAGES %%%%

% ¤¤ Oversaettelse og tegnsaetning ¤¤ %
\usepackage[utf8]{inputenc}					% Input-indkodning af tegnsaet (UTF8)
\usepackage[british, danish]{babel}					% Dokumentets sprog
\usepackage[T1]{fontenc}					% Output-indkodning af tegnsaet (T1)
\usepackage{ragged2e,anyfontsize}			% Justering af elementer
\usepackage{nomencl}
\makenomenclature
\usepackage{indentfirst}

% ¤¤ Figurer og tabeller (floats) ¤¤ %
% Tilføjet:
\usepackage{tabularx}
\usepackage{longtable}
\usepackage{graphicx}
\usepackage{pdfpages}
%Linjeskift i tabeller:
\newcommand{\specialcell}[2][c]{%
  \begin{tabular}[#1]{@{}c@{}}#2\end{tabular}}

  %anvend: \specialcell[]{Første del\\ anden del} [t]= align top, [b]= align bottom


% Tyk horisontallinje til tabeller :
\makeatletter
\newcommand{\thickhline}{%
    \noalign {\ifnum 0=`}\fi \hrule height 1.5pt
    \futurelet \reserved@a \@xhline
}
\newcolumntype{"}{@{\hskip\tabcolsep\vrule width 2pt\hskip\tabcolsep}}
\makeatother

% Anvend : \thickhline

\usepackage{tabulary}
\usepackage{graphicx} 						% Haandtering af eksterne billeder (JPG, PNG, EPS, PDF)
\usepackage{wrapfig}
%\usepackage{eso-pic}						% Tilfoej billedekommandoer paa hver side
%\usepackage{wrapfig}						% Indsaettelse af figurer omsvoebt af tekst. \begin{wrapfigure}{Placering}{Stoerrelse}
\usepackage{multirow}                		% Fletning af raekker og kolonner (\multicolumn og \multirow)
\usepackage{multicol}         	        	% Muliggoer output i spalter
\usepackage{rotating}						% Rotation af tekst med \begin{sideways}...\end{sideways}
\usepackage{colortbl} 						% Farver i tabeller (fx \columncolor og \rowcolor)
\usepackage{xcolor}							% Definer farver med \definecolor. Se mere: http://en.wikibooks.org/wiki/LaTeX/Colors
\usepackage{flafter}						% Soerger for at floats ikke optraeder i teksten foer deres reference
\let\newfloat\relax 						% Justering mellem float-pakken og memoir
\usepackage{float}							% Muliggoer eksakt placering af floats, f.eks. \begin{figure}[H]

\graphicspath{{billeder/}}						% Sti til figure

% ¤¤ Matematik mm. ¤¤
\usepackage{amsmath,amssymb,stmaryrd} 		% Avancerede matematik-udvidelser
\usepackage{mathtools}						% Andre matematik- og tegnudvidelser
\usepackage{textcomp}                 		% Symbol-udvidelser (f.eks. promille-tegn med \textperthousand )
\usepackage{amsthm}
\usepackage{rsphrase}						% Kemi-pakke til RS-saetninger, f.eks. \rsphrase{R1}
\usepackage[version=3]{mhchem} 				% Kemi-pakke til flot og let notation af formler, f.eks. \ce{Fe2O3}
\usepackage[binary-units=true, per-mode=fraction]{siunitx}						% Flot og konsistent
\DeclareSIUnit\vote{stemme}
\DeclareSIUnit\votes{stemmer}
%\SI{tal}{enhed}
\sisetup{output-decimal-marker = {,}}		% Opsaetning af \SI (DE for komma som decimalseparator)
\usepackage{icomma}							% Komma som decimal operator

% ¤¤ Referencer og kilder ¤¤ %
\usepackage[danish, english]{varioref}				% Muliggoer bl.a. krydshenvisninger med sidetal (\vref)
\usepackage[numbers]{natbib}							% Udvidelse med naturvidenskabelige citationsmodeller
%\usepackage{xr}							% Referencer til eksternt dokument med \externaldocument{<NAVN>}
%\usepackage{glossaries}					% Terminologi- eller symbolliste (se mere i Daleifs Latex-bog)
% ¤¤ Misc. ¤¤ %
\usepackage{listings}						% Placer kildekode i dokumentet med \begin{lstlisting}...\end{lstlisting}
\usepackage{lipsum}							% Dummy text \lipsum[..]
\usepackage[shortlabels]{enumitem}			% Muliggoer enkelt konfiguration af lister
\usepackage{pdfpages}
\usepackage{pdflscape}					% Goer det muligt at inkludere pdf-dokumenter med kommandoen \includepdf[pages={x-y}]{fil.pdf}
\pdfoptionpdfminorversion=6					% Muliggoer inkludering af pdf dokumenter, af version 1.6 og hoejere
\pretolerance=2500 							% Justering af afstand mellem ord (hoejt tal, mindre orddeling og mere luft mellem ord)

% Kommentarer og rettelser med \fxnote. Med 'final' i stedet for 'draft' udloeser hver note en error i den faerdige rapport.
\usepackage{footnote}
\usepackage[footnote,draft,danish,silent,nomargin]{fixme}
\makesavenoteenv{tabular}


\usepackage[ntheorem]{mdframed} %Bruges for at kunne lave definitionsboksene

\usepackage{filecontents}

%%%% CUSTOM SETTINGS %%%%
%%%%To-do markater%%%%
\usepackage{todonotes}

%%Tikz%%
\usepackage{tikz}
\usepackage{tikz-uml}
\usetikzlibrary{shapes.geometric, arrows, chains, matrix, mindmap, patterns, shapes.misc, trees, positioning}

%rettet fra Emilie%
\tikzstyle{startstop} = [rectangle, ultra thick, rounded corners, minimum width=3cm, minimum height=1cm,text centered, text width=3cm, draw=black, fill=white]
\tikzstyle{io} = [trapezium, trapezium left angle=80, trapezium right angle=80, ultra thick, minimum width=3cm, minimum height=1cm, text centered, draw=black, fill=white]
\tikzstyle{process} = [rectangle, ultra thick, minimum width=3cm, minimum height=1cm, text centered, draw=black, fill=white]
\tikzstyle{specialprocess} = [rectangle, ultra thick, minimum width=3cm, minimum height=1cm, text centered, draw=black, fill=white]
\tikzstyle{decision} = [diamond, ultra thick, minimum width=3cm, minimum height=1cm, text centered, draw=black, fill=white]
\tikzstyle{seconddecision} = [circle, ultra thick, minimum width=3cm, minimum height=1cm, text centered, draw=black, fill=white]
\tikzstyle{predefined} = [rectangle, rounded corners, ultra thick, minimum width=3cm, minimum height=1cm,text centered, text width=3cm, draw=black, fill=white]
\tikzstyle{data} = [trapezium, trapezium left angle=60, trapezium right angle=120, ultra thick, minimum width=3cm, minimum height=1cm, text centered, draw=black, fill=white]
\tikzstyle{terminator} = [ellipse, ultra thick, minimum width=3cm, minimum height=1cm, text centered, draw=black, fill=white]
\tikzstyle{arrow} = [ultra thick,->,>=stealth]

\tikzstyle{decision} = [diamond, draw, fill=blue!20,
    text width=4.5em, text badly centered, node distance=3cm, inner sep=0pt]
\tikzstyle{block} = [rectangle, draw, fill=blue!20,
    text width=5em, text centered, rounded corners, minimum height=4em]
\tikzstyle{line} = [draw, -latex']
\tikzstyle{cloud} = [draw, ellipse,fill=red!20, node distance=3cm,
    minimum height=2em]

%PGF%
\usepackage{pgfplots}
\usepgfplotslibrary{patchplots}

%%Boxes%%
\definecolor{theoline}{gray}{0.5}
\definecolor{theocolor}{gray}{0.9}
\definecolor{lemcolor}{gray}{0.97}
\definecolor{defcolor}{RGB}{214,234,218}

\mdfdefinestyle{box}{ % saving some space
skipabove=1.5em plus 0.4em minus 0.6em,
%skipbelow=0.5em plus 0.4em minus 0.2em,
%leftmargin=-7pt, rightmargin=-7pt,
innerleftmargin=3pt,innerrightmargin=7pt, innertopmargin=2pt, innerbottommargin=2pt,
%linewidth=1pt,
%splittopskip=3.2em minus 0.2em,
%splitbottomskip=0.5em plus 0.2em minus 0.1em,
}
\newmdtheoremenv[style=box,linecolor=theoline,backgroundcolor=theocolor]{theorem}{Theorem}[section]
\newmdtheoremenv[style=box,linecolor=theoline,backgroundcolor=lemcolor]{lemma}[theorem]{Lemma}
\newmdtheoremenv[style=box,linecolor=theoline,backgroundcolor=lemcolor]{proposition}[theorem]{Proposition}
\newmdtheoremenv[style=box,linecolor=theoline,backgroundcolor=lemcolor]{corollary}[theorem]{Corollary}



%\newtheorem{theorem}{Theorem}[section]
%\newtheorem{lemma}[theorem]{Lemma}
%\newtheorem{proposition}[theorem]{Proposition}
%\newtheorem{corollary}[theorem]{Corollary}
%\newtheorem{case}{Bevis}[theorem]
%\newtheorem{defn}{Definition}[section]

\renewenvironment{proof}[1][Proof]{\begin{trivlist}
\item[\hskip \labelsep {\bfseries #1}]}{\end{trivlist}}

\newenvironment{definition}[1][Definition]{\begin{trivlist}
\item[\hskip \labelsep {\bfseries #1}]}{\end{trivlist}}

\newenvironment{remark}[1][Remark]{\begin{trivlist}
\item[\hskip \labelsep {\bfseries #1}]}{\end{trivlist}}



\theoremstyle{definition}
\newmdtheoremenv[style=box,linecolor=theoline,backgroundcolor=lemcolor]{defn}[theorem]{Definition}

\newtheorem{eks}[theorem]{Example}
\surroundwithmdframed[
   topline=false,
   rightline=false,
   bottomline=false,
   leftmargin=\parindent,
   skipabove=\medskipamount,
   skipbelow=\medskipamount
]{eks}

\newenvironment{example}[1][Example]{\begin{trivlist}
\item[\hskip \labelsep {\bfseries #1}]}{\end{trivlist}}
\surroundwithmdframed[
   topline=false,
   rightline=false,
   bottomline=false,
   leftmargin=\parindent,
   skipabove=\medskipamount,
   skipbelow=\medskipamount
]{example}

\renewcommand{\qed}{\nobreak \ifvmode \relax \else
      \ifdim\lastskip<1.5em \hskip-\lastskip
      \hskip1.5em plus0em minus0.5em \fi \nobreak
      \vrule height0.75em width0.5em depth0.25em\fi}

% ¤¤ Marginer ¤¤ %
\setlrmarginsandblock{3.5cm}{2.5cm}{*}		% \setlrmarginsandblock{Indbinding}{Kant}{Ratio}
\setulmarginsandblock{2.5cm}{3.0cm}{*}		% \setulmarginsandblock{Top}{Bund}{Ratio}
\checkandfixthelayout 						% Oversaetter vaerdier til brug for andre pakker

%	¤¤ Afsnitsformatering ¤¤ %
\setlength{\parindent}{3mm}           		% Stoerrelse af indryk
\setlength{\parskip}{3mm}          			% Afstand mellem afsnit ved brug af double Enter
\linespread{1,1}							% Linie afstand

% ¤¤ Litteraturlisten ¤¤ %
%\bibpunct[,]{[}{]}{;}{a}{,}{,} 				% Definerer de 6 parametre ved Harvard henvisning (bl.a. parantestype og seperatortegn)
\bibliographystyle{unsrtnat}			% Udseende af litteraturlisten.

% ¤¤ Indholdsfortegnelse ¤¤ %
\setsecnumdepth{subsubsection}		 			% Dybden af nummerede overkrifter (part/chapter/section/subsection)
\maxsecnumdepth{subsubsection}					% Dokumentklassens graense for nummereringsdybde
\settocdepth{subsection} 					% Dybden af indholdsfortegnelsen

% ¤¤ Lister ¤¤ %
\setlist{
  topsep=0pt,								% Vertikal afstand mellem tekst og listen
  itemsep=-1ex,								% Vertikal afstand mellem items
}

% ¤¤ Visuelle referencer ¤¤ %
\usepackage[colorlinks,
            pdftex,
            pdfauthor={Andreas Madsen, Anja Nielsen, Anna Bonde, Astrid Ipsen, Henrik Christensen, Rasmus Kjær og Taniya Henriksen},
            pdftitle={Fanciful Handbook Management},
            pdfsubject={Developing a Handbook Management System for SMBs},
            pdfkeywords={handbooks,document,compsci},
            pdfproducer={Latex with hyperref},
            pdfcreator={pdflatex, or other tool}]{hyperref}			% Danner klikbare referencer (hyperlinks) i dokumentet.
\hypersetup{colorlinks = true,				% Opsaetning af farvede hyperlinks (interne links, citeringer og URL)
    linkcolor = black,
    citecolor = black,
    urlcolor = black
}

% ¤¤ Opsaetning af - og tabeltekst ¤¤ %
\captionnamefont{\small\bfseries\itshape}	% Opsaetning af tekstdelen ('Figur' eller 'Tabel')
\captiontitlefont{\small}					% Opsaetning af nummerering
\captiondelim{. }							% Seperator mellem nummerering og figurtekst
\hangcaption								% Venstrejusterer flere-liniers figurtekst under hinanden
\captionwidth{\linewidth}					% Bredden af figurteksten
\setlength{\belowcaptionskip}{0pt}			% Afstand under figurteksten
\addto\captionsdanish{\renewcommand{\figurename}{Figure}} %Aendre figurteksten til dansk

% ¤¤ Opsaetning af listings ¤¤ %

\definecolor{commentGreen}{RGB}{34,139,24}
\definecolor{stringPurple}{RGB}{208,76,239}


\lstset{language=[Sharp]C,					% Sprog
	basicstyle=\fontsize{11}{13}\selectfont\ttfamily,		% Opsaetning af teksten
	keywords={for,if,while,else,elseif,		% Noegleord at fremhaeve
			  end,break,return,case,
			  switch,function},
	keywordstyle=\color{blue},				% Opsaetning af noegleord
	commentstyle=\color{commentGreen},		% Opsaetning af kommentarer
	stringstyle=\color{stringPurple},		% Opsaetning af strenge
	showstringspaces=false,					% Mellemrum i strenge enten vist eller blanke
	numbers=left, numberstyle=\tiny,		% Linjenumre
	extendedchars=true, 					% Tillader specielle karakterer
	columns=flexible,						% Kolonnejustering
	breaklines, breakatwhitespace=true,		% Bryd lange linjer
	frame=single,
	morekeywords = {
	public,
	static,
	string,
	bool,
	int,
	void}
}

\lstMakeShortInline[columns=fixed]|
\lstset{language=C,captionpos=b}

% ¤¤ Navngivning ¤¤ %
\addto\captionsdanish{
	\renewcommand\appendixname{Appendix}
	\renewcommand\contentsname{Contents}
	\renewcommand\bibname{Bibliography}
	\renewcommand\appendixpagename{Appendix}
	\renewcommand\appendixtocname{Appendix}
	%\renewcommand\cftchaptername{Kapitel~}
%\renewcommand\cftchaptername{\chaptername~}
% Skriver "Kapitel" foran kapitlerne i indholdsfortegnelsen
	\renewcommand\cftappendixname{Appendix ~}
	%\renewcommand\cftappendixname{\appendixname~}			% Skriver "Appendiks" foran appendiks i indholdsfortegnelsen
}

% ¤¤ Kapiteludssende ¤¤ %
\definecolor{numbercolor}{RGB}{130,0,50}
	% Definerer en farve til brug til kapiteludseende
\newif\ifchapternonum

\makechapterstyle{jenor}{					% Definerer kapiteludseende frem til ...
  \renewcommand\beforechapskip{0pt}
  \renewcommand\printchaptername{}
  \renewcommand\printchapternum{}
  \renewcommand\printchapternonum{\chapternonumtrue}
  \renewcommand\chaptitlefont{\fontfamily{pbk}\fontseries{db}\fontshape{n}\fontsize{25}{35}\selectfont\raggedleft}
  \renewcommand\chapnumfont{\fontfamily{pbk}\fontseries{m}\fontshape{n}\fontsize{1in}{0in}\selectfont\color{numbercolor}}
  \renewcommand\printchaptertitle[1]{%
    \noindent
    \ifchapternonum
    \begin{tabularx}{\textwidth}{X}
    {\let\\\newline\chaptitlefont ##1\par}
    \end{tabularx}
    \par\vskip-2.5mm\hrule
    \else
    \begin{tabularx}{\textwidth}{Xl}
    {\parbox[b]{\linewidth}{\chaptitlefont ##1}} & \raisebox{-15pt}{\chapnumfont \thechapter}
    \end{tabularx}
    \par\vskip2mm\hrule
    \fi
  }
}											% ... her

\chapterstyle{jenor}						% Valg af kapiteludseende - Google 'memoir chapter styles' for alternativer

% ¤¤ Sidehoved ¤¤ %

\makepagestyle{AAU}							% Definerer sidehoved og sidefod udseende frem til ...
\makepsmarks{AAU}{%
	\createmark{chapter}{left}{shownumber}{}{. \ }
	\createmark{section}{right}{shownumber}{}{. \ }
	\createplainmark{toc}{both}{\contentsname}
	\createplainmark{lof}{both}{\listfigurename}
	\createplainmark{lot}{both}{\listtablename}
	\createplainmark{bib}{both}{\bibname}
	\createplainmark{index}{both}{\indexname}
	\createplainmark{glossary}{both}{\glossaryname}
}
\nouppercaseheads											% Ingen Caps oenskes

\makeevenhead{AAU}{Gruppe: DAT2 B2-2}{}{\leftmark}				% Definerer lige siders sidehoved (\makeevenhead{Navn}{Venstre}{Center}{Hoejre})
\makeoddhead{AAU}{\rightmark}{}{Aalborg Universitet}		% Definerer ulige siders sidehoved (\makeoddhead{Navn}{Venstre}{Center}{Hoejre})
\makeevenfoot{AAU}{\thepage}{}{}							% Definerer lige siders sidefod (\makeevenfoot{Navn}{Venstre}{Center}{Hoejre})
\makeoddfoot{AAU}{}{}{\thepage}								% Definerer ulige siders sidefod (\makeoddfoot{Navn}{Venstre}{Center}{Hoejre})
\makeheadrule{AAU}{\textwidth}{0.5pt}						% Tilfoejer en streg under sidehovedets indhold
\makefootrule{AAU}{\textwidth}{0.5pt}{1mm}					% Tilfoejer en streg under sidefodens indhold

\copypagestyle{AAUchap}{AAU}								% Sidehoved for kapitelsider defineres som standardsider, men med blank sidehoved
\makeoddhead{AAUchap}{}{}{}
\makeevenhead{AAUchap}{}{}{}
\makeheadrule{AAUchap}{\textwidth}{0pt}
\aliaspagestyle{chapter}{AAUchap}							% Den ny style vaelges til at gaelde for chapters
															% ... her

\pagestyle{AAU}												% Valg af sidehoved og sidefod


%%%% CUSTOM COMMANDS %%%%

% ¤¤ Billede hack ¤¤ %
\newcommand{\figur}[4]{
		\begin{figure}[H] \centering
			\includegraphics[width=#1\textwidth]{billeder/#2}
			\caption{#3}\label{#4}
		\end{figure}
}

% ¤¤ Specielle tegn ¤¤ %
\newcommand{\decC}{^{\circ}\text{C}}
\newcommand{\dec}{^{\circ}}
\newcommand{\m}{\cdot}


%%%% ORDDELING %%%%

\hyphenation{}

%Mads
\usepackage[strings]{underscore}
%\usepackage{nccmath}
%\usepackage{breqn}
\usepackage{amsmath}
\usepackage{cleveref}
\usepackage{epigraph}

\crefname{table}{table}{tables}
\Crefname{table}{Tables}{Tables}
\crefname{figure}{figure}{figures}
\Crefname{figure}{Figure}{Figures}
\crefname{equation}{equation}{equations}
\Crefname{equation}{Equation}{Equations}
\crefname{section}{section}{sections}
\Crefname{section}{Section}{Sections}
\crefname{subsection}{section}{sections}
\Crefname{subsection}{Section}{Sections}
\crefname{subsubsection}{section}{sections}
\Crefname{subsubsection}{Section}{Sections}
\crefname{chapter}{chapter}{chapters}
\Crefname{chapter}{Chapter}{Chapters}
\crefname{appendix}{appendix}{appendixes}
\Crefname{appendix}{Appendix}{Appendixes}
\crefname{eks}{example}{examples}
\Crefname{eks}{Example}{Examples}
\crefname{theorem}{theorem}{theorems}
\Crefname{theorem}{Theorem}{Theorems}
\crefname{defn}{definition}{definitions}
\Crefname{defn}{Definition}{Definitions}
\crefname{lstlisting}{listing}{listings}
\Crefname{lstlisting}{Listing}{Listings}




\newcommand{\SubItem}[1]{
	{\setlength\itemindent{15pt} \item[-] #1}
}

