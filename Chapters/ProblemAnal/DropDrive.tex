\subsection{Google Drive \& Dropbox}\label{sec:DropDrive}
\todo[inline]{Andreas, har du nogle kilder til dette afsnit?}
The systems Google Drive and Dropbox are, as mentioned in \cref{sec:existingsolutionsintro}, document management systems.

Google Drive is an online storage service with a real time updated view of the document for collaborative working projects, while Dropbox is a file hosting service that allows the user to place files on remote servers.
The Dropbox client creates a folder on the user's PC, which is used to keep the files inside synchronized with the Dropbox servers. 
The system uses binary diff to determine the difference from one file to another, this means that only the difference from each file is saved and saves space occupied by the files \cite{DropboxDiff}.
%Maaske noget vi burde nævne som brugbart i konklusion??? - Astrid

For both systems it is furthermore possible to see the file history.
In Google Drive this is called version history, where it is possible to see how the document looked at different times, as well as who has made the different changes from the last version.
The Dropbox system stores a file change history for 30 days, which makes it possible to restore data changed in this period.

Lastly, both system provides users with the opportunity to share files with different access permissions which are; \textit{view}, \textit{comment} and \textit{edit}.

The basic ideas of these two systems are useful in the fact that they are both systems which manage documents and allow for cooperation while keeping an updated view.
The different access permissions are also quite inspirational since it makes it possible to damage control when only a few have total access instead of everyone.
Since it is important when managing handbooks to keep track of older versions of different documents these two systems are also somewhat useful.
Although, Dropbox only being able to save the change history for 30 days makes this a bit obsolete in this case.
On the other hand, the way Google Drive does is more useful, though since older versions will usually only be accessed at audits and the company at these needs to show how the whole handbook looked at a specific time, it can become a tedious task to find the specific version in each document.

These two systems are close to how the work is done at this time except for the feature of sharing files.
However, it has become clear through the first interview with Ipsen and the research into the official government requirements, that the current system lacks better and easier management of different versions and documents, as well as more specific access permissions not file sharing.
