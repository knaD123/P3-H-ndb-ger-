\section{Tentative Problem Statement} \label{problemstatement}
%\todo[inline]{Anja: Introducerende tekst}

In the previous chapter the case and the consequent analysis has been explored.
This has led to descriptions of the case, ISO-standards, exisiting solutions, and a rich picture and FACTOR-analysis.

The case presented by Ipsen and the closely related ISO-standards are the main problems that the project seems to solve through the design and development of a system that supports the versioning of handbook documents. 
The following exploration of existing solutions outlines the current technological landscape in relation to handbook management which could potentially give inspiration to design of the future system and help differentiate it.
By the end of the chapter an in-depth analysis of Ipsens interviews were made through rich pictures and the FACTOR criterions.

This problem analysis has lead to the following problem statement:

\begin{center}
\textit{How can a document version management software be designed and developed to support a firm's management of their handbook documents so it is easily manageable by the administrators, and also accessible for readers and writers?}
\end{center}

The rest of the project and system will seek to answer this statement.
There are several aspects of the statement to consider during the process.
These are
\begin{itemize}
	\item \textit{Document version management:} The software must be able to manage documents and their versions.
	\item \textit{Design and development:} There are two main processes that the project group must undergo through the project which are the design processes and the development of the system based on the design.
	\item \textit{Administrators, readers, and writers:} There are different roles associated with the system that needs to be considered both during the design and development process.
\end{itemize}