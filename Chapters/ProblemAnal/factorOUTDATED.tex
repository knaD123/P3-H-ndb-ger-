\section{System definition}

Problem domain or Application domain analysis
In the following the FACTOR criterions will be elaborated in detail. An explanation of FACTOR and an overview of our criterions can be found on page xx.

\textbf{Functionality}

The system should manage versions of handbook documents. This means that the system should keep track of versions of one or more documents by their names, dates, versions number. The documents should furthermore be uniquely identifiable with number. When a document is no longer active it should be archived. The storage time in the archive should be adjustable. Title and chapter number are linked. If a title is removed the chapter number cannot be used for anything else. When the system is deployed it should be able to handle existing handbooks and their archives. The handbook should be printable

When a document is updated it should only be possible to edit the newest version of the document. Before an updated document is released it should be approved. When a document hasn’t been updated in a specific amount of time the administrator is notified. Contractors and their documents need approval, and their date and notification for renewal needs to be registered. % contractor for en hvilke dokument? 

Users should have different levels of permissions/access rights to the documents within the handbook. The different levels and permissions rights are prioritized in this order;

% har vi ikke kun 3 levels nu?
0 level: Reading the handbook except secured documents

1st level: Reading the whole handbook

2nd level: Editing selected documents

3rd level: Total access to editing documents and access to archive

There should be a Table Of Contents which gets updated automatically as new versions of documents are released. Differences should be highlighted when a new document has been released. A changelog must be included. It should be adjustable where the changelog is located. The number of previous version’s changelog covers or how much time it should cover is adjustable.

It should be possible to attach the documents to specific departments. These departments should be subscribed to one or more documents. When a document is updated the subscribed departments gets notified. The system should register when a document has been read by a user.

The system should be able to handle documents from external sources. The system needs to handle several different file types, which includes word, excel, power point and image files

\textbf{Application domain}

All users should be able to read and print the documents in the handbook. The Quality Assurance Manager and secretary should have 3rd level access meaning they are able to read, write and edit documents. The managing director has either 2nd or 3rd level access to the document.

\textbf{Conditions}

The user segments consist of small firms who needs either a cheap or free system. The users have limited IT-experience. The system needs to be accessible at the place where it is needed. The user segment can’t maintain the system nor perform sysadmin work.

\textbf{Technology}

The system is only needed for PC. E-mail and SMS will be utilized for the notification feature.

\textbf{Objects}

Handbook, TOC (Table of Contents), Documents, Changelog, and User.

\textbf{Responsibility}

The system’s core responsibility is document management where it facilitates an overview of the documents through the table of contents. It should keep track of the documents and their versions which includes making sure there is a changelog if it is required by the company. Out of date documents should be archived. The system manages which level of control the users have. The system should keep track of documents used in the approval process of contractors and dates for their approval and re-approvals. The system should make sure all new documents are approved before being added to the handbook.
