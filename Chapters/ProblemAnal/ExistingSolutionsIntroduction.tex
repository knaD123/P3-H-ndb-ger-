When developing a new system it is important to explore existing systems.
This is to ensure that the designers and developers know which solutions already exist and thereby how the future solution is going to fit into the technological landscape.
To get inspiration for the new system some existing solutions were examined.
For this case, the report asks the following questions when exploring an existing system \citep[p.~33]{Rod-Aalborg}:

\begin{itemize}
  \item Which ideas lay the groundwork for the system?
  \item Do the ideas seem useful? Why?
  \item Will the ideas work in your context? Why?
  \item Can the ideas be adapted to your system? How?
\end{itemize}

The systems this report will look into, in this section, are a couple of existing systems that specialize in handbook management, are; \textit{BlissBooks} and \textit{D4Handbooks}, as well as a short mention of the more everyday document management systems; \textit{Google Drive} and \textit{Dropbox}.
All the systems were tried, besides \textit{D4Handbooks}, which did not have a publicly available trial version. That chapter is largely based on sales material, which naturally causes a bias.

Many of the solutions introduced are Software as a Service (SaaS).
This refers to software which is not downloaded and run on the users computer, as traditional software, but rather runs in the browser, with much of the backend work happpening on the owner of the softwares server. % Der mangler nok en kilde her

The systems will be analyzed and judged in relation to Ipsen's needs.
These needs were introduced in \ref{sec:CaseDescription} and will be further defined and elaborated upon in \ref{sec:requirements}.
