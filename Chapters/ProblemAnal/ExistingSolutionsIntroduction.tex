When developing a new system it is important to explore existing systems.
This is to ensure that the designers and developers know which solutions already exist and thereby how the future solution is going to fit into the technological landscape.
It is also beneficial to explore exisiting solutions as these can give the designers knowledge about the design convention within the area and perhaps new ideas to explore.
%bør der evt. være en kilde her?
%astrid: evt undgå med omformulering? "To get inspiration for the new system some existing solutions were examined"?
For this case the report asks the following questions when exploring an existing system \citep[p.~33]{Rod-Aalborg}:

\begin{itemize}
  \item Which ideas lay the groundwork for the system?
  \item Do the ideas seem useful? Why?
  \item Will the ideas work in your context? Why?
  \item Can the ideas be adapted to your system? How?
\end{itemize}

The systems this report will look into in this section are a couple of existing systems that specialize in handbook management are; \textit{BlissBook}, \textit{D4Handbooks}, and \textit{IPW}, as well as the more general document management systems; \textit{Google Drive} and \textit{Dropbox}.
\todo[inline]{RASMUS: Tilføj fundet fra hjemmeside shit}
\todo[inline]{Rasmus: Gør det lidt pænere når alting er SaaSS}
\todo[inline]{Astrid: Lav en forward ref til hvad Pia vil have}
