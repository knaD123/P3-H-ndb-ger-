When developing a new system it is important to explore existing systems.
This is to ensure that the designers and developers know which solutions already exist, how the future solution is going to fit into the technologial landscape.
It is also beneficial to explore exisiting solutions as these can give the designers knowledge about the design convention within the area and perhaps new ideas to explore.
%bør der evt. være en kilde her?
%astrid: evt undgå med omformulering? "To get inspiration for the new system some existing solutions were examined"?
There are a few questions that is worth asking when exploring an existing system \citep[p.~33]{Rod-Aalborg}:

\begin{itemize}
  \item Which ideas lay the groundwork for the system?
  \item Do the ideas seem useful? Why?
  \item Will the ideas work in your context? Why?
  \item Can the ideas be adapted to your system? How?
\end{itemize}  

A couple of existing systems that specialize in handbook management, \textit{BlissBook}, \textit{D4Handbooks}, and \textit{IPW}, as well as more general management systems like \textit{Google Drive} and \textit{Dropbox} are explored in this section.
