\subsection{BlissBook}
BlissBook is a SaaS employee handbook system, developed by the company of the same name starting in 2013 \cite{BlissbookInfo}.
This section is mostly based upon information from their website \cite{BlissbookContents}.

BlissBook is based on a rich text editor which supports collabrative editing, and has features such as access control, digital signatures to keep track of which employees have read the documents, as well as sending notifications to employees when new versions are released.

As they are a SaaS platform, all the content and software is actually hosted on their servers.
On their server, they guarantee a 99.9\% uptime, and exports of all data on the platform, even after a canceled subscription.
How long they keep the data is not specified.

On the security front, the BlissBook website does not mention any certificatations on their software, besides  AES-256 encryption \cite{BlissbookSecurity}.
They do not freely offer up any sort of self-hosting solution, which security-conscious corporations may be very interested in.
However, as Ipsen does not have any specific security concern besides data integrity, the SaaS structure is in the end of an advantage, as this frees the user from having to do any sort of server maintenance.

One of the greatest disadvantages to this system is that all the files in the system needs to be written in BlissBook's own online editor.
This would present a great difficulty, as all existing documentation, which Ipsen relies on, is written mostly in Word documents, but also spreadsheets, pictures, and others (see \cref{sec:systemdefinition}).

Additionally, the solution does not seem to be fitted to the use case of standards compliance, but rather handbooks for human resources use cases.
While they do have a system for external handbook consultants, they are again focusing on HR consultants. \cite{BlissbookHandbook}
