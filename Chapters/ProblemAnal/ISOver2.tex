\section{Standards v.2}

% mention how standards play a role of ensuring trust between organisations, in that it can help ensure that everyone is doing stuff properly. Also, auditing is probably a process that should be mentioned, as that is part of standards workflow

%intro
This section covers explanation of standards, theirs purposes and how these standards affect handbook document management. 

%what is a standard
A standard is a technical specification that refers to a set of requirements, rules, guidelines or definitions. In other word a standard is in its core an agreed way of doing things. This provide a basis common understanding for organizations and people.  A standard is a tool to ease communication, marketing, measurement and manufacturing. For instance standards provide employees a basis for the expectation of their work with the right standard operating procedures. This way the organization or company ensure the quality and consistency of the production, which results employees is performing the task properly. 
This leads to establish consumer trust and confidence as the product or the service lives up to the expectation. The standardization benefits everyone from increases efficiency of the organization and product safety to simplify people's everyday lives.

%standard in version control context, ved ikke om det her skal ned under ISO afsnit? 
%  iso 9001 - ud fra interview 1, præsentation
In the project's context regarding the document management of a handbook there exists requirements that need to be met if the company have a certified management system. There exists management system for various fields e.g. quality, environment and working environment, food safety and sustainability.  However the requirements for document management does not vary as much in those different standards. 

The general requirements for document management
\begin{itemize}
	\item Documents can have different format and structure e.g. a text document, spreadsheet, pictures, videos and other file types. For instance a picture as a document is useful to represent a burger's appearance. 
	\item Documents must be uniquely identified. There aren't a specific approach, generally are the documents identified with numbers, name, date and version number. 
	\item Documents must be reviewed and approved to determine suitability and sufficiency for the organization. Often the document consists fields with date and initials or names of the author and  the approver.
	\item The document must be available and accessible where it is needed e.g. for employees at a factory or at a office. 
	\item The document must be sufficiently protected against loss of confidentiality and unintended modifications. 
\end{itemize}

The organization have to consider how  the requirements below is implemented. 

\begin{itemize}
	\item Distribution and accessibility of the individual document. The organization must make a decision whether who need access to the document and if it is possible in the work situation. 
	\item Document storage condition and preservation. The organization need to ensure that the documents is storage properly........... 
\end{itemize}

%importance of document control

% Opbevaringsforhold og bevarelse. Bliver dokumentet ved at være læsbart mm

% krav for dokumentstyrring
% document kan være et billede af hvordan et burger skal ser ud. 
% det skal være en entydedig identificere, men siger ikke hvordan, men typisk det der stpr i powerpoint
% det er et krav at det skal være tilgængelige til dem der skal bruge det fx. ude i produktion skal have adgang til det. oftest it systemer. 

%dokumentation skal være beskyttet - noget med fortrolighed at gøre eller mere at man ikke skal ret den - normalt man password, firma beslutter selv hvordan. 
% firma skal tag stilling til hvor skal vi bruge det henne og hvem og hvordan man får fat i dem.

%Krav: skal sikre sig hvordan de opbevares rigtigt og bevaret - i gamle dage mod vandskade , system skift alle skal kunne læse dem
%Krav styrrer ændringerne, at alle bruge den samme og den rigtige udgave og bruger skal vide hvilke rettelse der er. Så man ikke skal læse hele dok igen 

% hvor længe dok skal opbevare, men skal opbevare udgået dok, standard siger ikke hvor lang tid, typisk er det 3-5 år på myndighedskrav. noget special er det længere krav. arbejder man med okø er 5 år. noget man ikke gå ind så tit i men de skal være der. 

% dok fra ekterne kilde, datablade og lovkrav. er det vigtigt ifth til man lave skal man styrre dem som andre dok. man skal kunne tage de  ude fra system ved at henvise til dem. skal kunne låse hvis det er fortroligt dok. 











% for eksempel med enheder og så for projektets tilfælde er det dokumentstyrring


\subsection{ISO standard}


%This doesn't seem super concrete, and more like a sales speech for ISO. In the problem analysis, we should probably focus on what parts of ISO are essential to our project specifically.

% ISO 9001 og dokument styrring

% how Pia implemented ISO on the company that she is consulent for


The following subsection covers the most widely used standards worldwide [], which includes explanation of what ISO standards contain, their purposes and benefits, as well as why ISO standards are relevant to the handbook version control system. This section is based on [] unless otherwise stated.

ISO is a non-governmental organization that develops and produces International Standards in various fields. The ISO standards are created by leading experts globally, so the knowledge is shared and negotiated. ISO seeks to meet the needs of the market; new standards will be developed if they are requested by an industry or consumer groups. An ISO standard is a document consisting of practical information, requirements, specifications, guidelines, and best practices.

The general purpose of ISO standards is to create safe, reliable and high quality products and services including improving the efficiency of the business which results in minimizing waste, errors and reducing production costs. This way business could gain consumer confidence, promote best practices to fill knowledge gaps, and optimize resource utilization. The ISO standards also provide solutions to challenges such as environment issues and sustainability.

There exist various types of ISO standards from Risk Management standards to Safety and Healthcare standards. The most well-known are listed below.

\begin{itemize}
	\item ISO 9000 Family - Quality Management
	\item ISO 22000 Food Safety Management
	\item ISO 14000 Family - Environmental Management
\end{itemize}

It is relevant to look further into the ISO 9001 which is a Quality Management System that is suitable to any business or organization regardless of size and industry. The ISO 9001 standard sets criteria for a quality management system that contributes to the improvement of efficiency and customer satisfaction. The Quality Management System is based on Quality Management Principles that contain customer focus, leadership and employee engagement, the process approach and continual improvement. Generally, the ISO 9001 serves as a base standard that is compatible with other standards of different specific sectors and industries.

ISO standards are voluntary. However, some are mandatory if for example referenced in regulation. These regulations differ across countries. For instance, the ISO 7010 standard for safety signs is accepted as the European Standard, which is mandatory for all EU countries [].

ISO standards are reviewed at least once every five years to ensure that the standards are relevant and up to date to the present market. This means the handbooks that were written using ISO standards as a starting point likewise need to be revised, so the new versions are updated in terms of ISO standards. Gradually, the handbook will contain numerous versions of documents. It is time consuming to manage and keep track of the current version manually without a version control system.
