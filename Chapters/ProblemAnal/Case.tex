\section{Case Description} \label{sec:CaseDescription}

This report will be based on the case presented by Ipsen, who is a senior consultant for the firm QMS-Consult.
This firm guides companies on how to maintain or write handbooks, to follow given standards, such that the companies can acquire or keep their certifications.
A certification is either granted or denied after an audit, where a professional will go through the company and determine whether the requirements are fulfilled or not.
In her work she helps the companies with their handbooks and in a few cases acts as the company's quality manager.
The case presented in this report is made in connection to a small firm where she is basically acting quality manager rather than a consultant.
All information in this section is based upon interviews with her.

A handbook consists of several documents that each describe procedures or guidelines for different aspects of a company and its workings.
There exist several versions of each document where only the newest version is active, and all of the previous versions are required to be archived for at least three years.
All of these documents must adhere to a set of rules, notably the following:

\begin{itemize}
	\item
	A document must always be kept up to date.
	In reality it is sometimes only done yearly.
	\item
	When a new version is written it must be approved by usually one or two people before it is added to the handbook.
	\item
	Each document has a title with an ID that can only be associated with that specific title.
	Such that, when a title has been paired with an ID, the ID can no longer be associated with any other titles, even if the current title should be deleted at a later point.
\end{itemize}

Additionally, as a service to the readers, Ipsen will highlight changes in between versions.
This is a service, not a requirement.
However, in some standards it is a requirement to submit a changelog detailing the reason for the changes made.
Ipsen expects this to be present in most requirements within the next ten years.

While the documents are only required to be archived for at least three years, in reality, previous versions are seldomly ever deleted.
This is due to the hassle of determining files that are no longer needed and due to the relatively small space required to digitally store them.

Ipsen works together with the firm to make sure that their handbooks are in good shape for audit.
Whenever a new document has been updated it is imperative that the workers read and understand the documents relevant to them and do so, immediately.
Not all documents are relevant for every worker, as the daily responsibilities of the workers can vary.

The problem with the current work procedure is that all the documents are stored in folders on a computer and Ipsen manually keeps track of older documents as well as the active ones.
This manual work method takes time and has a high risk of mistakes as it can be complicated to keep track of the newest documents and their approval stages.
Ipsen requires a system that can manage the documents and their versions.
The system should furthermore be used to ensure that the everyday workers have read the relevant documents.

A more in depth analysis of Ipsens needs will be explored in the FACTOR-analysis in \cref{factor} and the rich picture analysis in \cref{sec:richpictures}.
