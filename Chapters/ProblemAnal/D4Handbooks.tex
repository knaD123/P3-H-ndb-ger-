\subsection{D4Handbooks}
%\todo[inline]{Anja: Start det her afsnit ordentlig} DONE! 
%\todo[inline]{Anja: Useful feature er egen mening. Måske "This matches Ipsen's specifications".} DONE!
D4Handbooks, developed by D4InfoNet, allows you to create digital handbooks that are similar to physical ring binders.
D4Handbooks makes it possible to target documents to specific employees and to register whether employees have read those documents.
When a document has been updated the employees, to whom that document is relevant, will get a notification about the update.
This chapter is based upon information from D4Handbooks's site \cite{D4Hanbook}.
\todo[inline]{RASMUS: find på noget smart om There is also full traceability for all activities}

The handbooks are stored on D4Handbooks' servers and it therefore requires internet access every time the users of D4Handbooks want to update their handbook or an employee wants to read information that is relevant to them.
D4Handbooks also provides a feature that allows for offline use. It is something that the user would have to pay extra for.
%er det relevant at brugeren betaler ekstra? og i så fald burde det så ikke også forklares hvad brugeren ellers betaler for?
        % De betaler for at kan få lov til at bruge systemet - siger vel lidt sig selv?
%seems virker vagt. enten så kan det bruges eller også kan det ikke
        % Har du en bedre formulering, så er du meget velkommen til at ændre den

These features are both very useful because:

\begin{itemize}
        \item
        The employees do not have to search the handbook for the information that is relevant to them
        \item
        It adheres to the ISO standards.
\end{itemize}

Offline management also seems like a very useful feature because this allows users to manage handbooks without having to be connected to D4Handbooks server.
% find på en måde at beskrive, at disse features er nogen som vi gerne vil have i vores system + henvis til Pia's ønsker (interview)
%hvorfor er vi interesserede i offline??? det mener jeg ikke at hun nogensinde har nævnt? and for that matter er det så ikke meningen at man arbejder mens man er på arbejdet og ellers ikke?
