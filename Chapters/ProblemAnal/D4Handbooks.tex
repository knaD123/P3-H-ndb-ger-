\subsection{D4Handbooks}
This chapter is based on information 
%available on the companys website [source]
from \url{ttps://www.d4infonet.com/products/handbooks/}.

D4Handbooks, developed by D4InfoNet, allows you to create digital handbooks that are similar to physical ring binders.\\
\todo{diskussion omkring formatering af afsnit}
D4Handbooks makes it possible to target documents to specific employees and to register that the employees have read the documents relevant to them. When a document has been updated the employees, to whom that document is relevant, will get a notification about the update.
% find på noget smart om "There is also full traceability for all activities."

% nendenstående er en antagelse ud fra, at man upgraderer "for offline use"
The handbooks are stored on D4Handbooks servers and it therefor requires internet access every time the users of D4Handbooks want to update their handbook or an employee wants to read information that is relevant to him/her.\\
D4Handbooks also provides a feature that allows for offline use. This seems like a very useful feature but it is something that the user would have to pay extra for.

Allowing the users of this system to target documents to specific employees and to register that the employees have read those documents are both very useful features because: a) the employees do not have to search the handbook for the information that is relevant to them, and b) \textbf{*pga ISO*}. Allowing the users to manage their handbook(s) offline also seems like a very useful feature because this allows them to manage the handbook(s) wherever they are, i.e. in the bus or train. 
% find på en måde at beskrive, at disse features er nogen som vi gerne vil have i vores system + henvis til Pia's ønsker (interview)
