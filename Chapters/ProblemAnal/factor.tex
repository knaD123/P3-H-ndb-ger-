\section{System definition}
\subsection{FACTOR criteria}

Based on the data gathering we can make a system definition which seeks to define what problem the system should solve. A system definition is “A concise description of a computerized system expressed in natural language” (Mathiassen et al., p. 24). This means that the system definition aims to describe what the system should do, but also in which context is should be used in, as the system is going to be “implemented both technically and socially” (Ibid., p. 23). The system is going to solve a problem that exists in a specific context, it is therefore important to understand this context.
To analyze both the context and the criterions for the system FACTOR is being utilized. The FACTOR criterion consists of six elements and can be used to help with the system definition development or describing an existing system (Ibid.). In this context FACTOR is being used for the former option, which is to help facilitate the system definition development. FACTOR stands for:

\begin{itemize}
\item “Functionality: The system functions that support the application-domain tasks.

\item Application domain: Those parts of an organization that administrate, monitor, or control a problem domain.

\item Conditions: The conditions under which the system will be developed and used.

\item Technology: Both the technology used to develop the system and the technology on which the system will run.

\item Objects: The main objects in the problem domain.

\item Responsibility: The system’s overall responsibility in relation to its context.” (Ibid., page 40)
\end{itemize}

Based on the interview described on page xx we have been able to fill out the FACTOR criterions. In the following an overview of the analysis will be presented. The FACTOR criterion analysis will be elaborated on page xx.

Functionality: Managing versions of handbook documents and each document being uniquely identifiable with number, title, date, and version number(Each identification number is reserved for a specific title once taken). Earlier versions of the documents should also be archived, but can only be viewed by the administrater and are not editable.
The document number, title, date, and version number needs to be in a table of contents that is updated automatically.
Users have different levels of access rights to the documents within the handbook, first level being reading the documents within the handbook, second level being editing/adding documents and a third administrative level being in full control of the system.
The system can register that a user has read a version of a document.
The system should be able to handle different types of documents, incoperate earlier handbooks into the system and the handbook is printable.
A changelog must be included in the system and highlighting of changes would greatly help the reader.
Users with administrative priviliges can assign other users to departments and determine which documents are associated with those departments, thereby showing users the most relevant documents and send notifications to users of specific departments when new changes or versions are uploaded into the system.
A system for approval of new versions of documents and a system for approval of suppliers can be implemented. 
A notification is sent to the administrater when supplier documents need to be updated. The system can sort documents by different attributes.

Application domain: Mainly secretaries and 3rd party consultants will administrate the system. All employees of the organization will have access to the system to read documents and print the handbook. The managing director will have partial access to the system.

Conditions: The target group consists of small firms who needs a cheap 
solution. The users have little IT-experience. The system should be accesible at the place where it is needed.
There should be a backup of the archive making it easy to revert to the previous system.(Skal det ikke ligge under functionalitet?)
No on at the company is able to maintain the system.

Technology: The system is only needed on PC. E-mail and/or sms for notification purposes.

Objects: Handbook, Table Of Contents (TOC), documents, changelog, and user.

Responsibility: Management tool to keep track of document versions, this also includes making a changelog. The table of contents should give an overview of all the documents. The system has to save older versions of documents. Access should be separated into 3 levels, as stated above. Keeping track of documents used in the approval process of contractors and dates for their approval and re-approvals. Making sure all new documents are approved before being added to the handbook.