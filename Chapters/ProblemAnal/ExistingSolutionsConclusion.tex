\subsection{Evaluation}
To compare the different systems the \cref{tab:Exsisting} below is made to show which features are present in which system.
The features will be evaluated are those requested by Ipsen, \cref{sec:requirements}.

\begin{table}[H]
	\begin{center}
		\begin{tabular}{| m{5cm}|m{1.6cm}|m{2cm}|m{1.5cm}|m{1.2cm}|}
			\hline
			& BlissBook  & D4-Handbooks & Dropbox & Google \newline Drive \\
			\hline
			Collaborative editing & \checkmark & & & \checkmark \\
			\hline
			Managing different versions of documents & \checkmark &  &  & \checkmark \\
			\hline
			Table of contents & \checkmark & \checkmark  & & \\
			\hline
			Automatically updating the table of contents & \checkmark & \checkmark  &  & \\
			\hline
			Register when a document has been read & \checkmark & \checkmark &  & \\
			\hline
			Including a changelog & \checkmark & \checkmark  &  & \\
			\hline
			Permissions levels & \checkmark &  & \checkmark & \checkmark \\
			\hline
			Highlighting differences & \checkmark &  &  & \checkmark\\
			\hline
			Documents from external \newline sources &  &  & \checkmark & \checkmark \\
			\hline
			Online use & \checkmark &  \checkmark & \checkmark  & \checkmark \\
			\hline
			Offline use & & \checkmark & \checkmark & \checkmark \\
			\hline
			Users receive notifications when relevant documents are updated & & \checkmark & & \\
			\hline
		\end{tabular}
		\caption{Feature comparison table}\label{tab:Exsisting}
	\end{center}
\end{table}

The features seen in \cref{tab:Exsisting} contains ideas that can be used in the project's context.
Looking at the table above we can answer the questions asked in the introduction to this \cref{chap:existing}.
In the following sections they will be evaluated systematically on whether or not they can be used.

\textit{Collaborative editing} is not a necessary feature in this context.
This is because that there is, usually, only one person writing/updating the handbook documents according to Ipsen.
% Anja: Jeg er ikke heelt sikker paa om det jeg skriver ovenfor er rigtigt. Someone, confirm or deny.

\textit{Managing different version of documents} is an especially useful feature to take into consideration.
This is essentially the feature that Ipsen needs for her workflow.

\todo[inline]{Anja: Ideerne skal struktureres}
\subsubsection*{Which ideas lay the groundwork for the system?}

\textit{Table of contents} (TOC) and \textit{automatically updating the TOC} are also a good ideas to include.
TOC is important to include as this could ensure that Ipsen is able to have an overview of the documents.
Updating the TOC automatically is equally a good idea as this would help the overview as the documents and their versions get updated.

\textit{Including a changelog} is another useful idea as Ipsen has stated that, depending on the firm, a changelog between two different versions is preferred.
This is a feature provided by Google Drive and DropBox, in form of a file history, see \cref{sec:DropDrive}.
This changelog should state what specifically has changed between the previous version and the current.

\textit{Permissions levels} is likewise a good idea as an average employee at the firm would only have to read certain documents whereas Ipsen's work, among other things, entails to manage the handbook documents.
This is a feature provided by Blissbook, see \cref{sec:blissbook}.
Here the permission levels could ensure that the average employee is \textit{only} able to read documents while Ipsen would have full control of the management.

\textit{Highlighting differences} is potentially also a good idea in relation to the changelog as this could further ensure an overview of changes.
This is a feature provided by both Google Drive and DropBox, in form of file history, see \cref{sec:DropDrive}.
Here the specific differences of the new version would be highlighted in relation to the previous.

\textit{Online use} \& \textit{offline use} both seem useful with different advantages and disadvantages for both.
The offline use is a feature provided by D4Handbooks, see \cref{sec:d4handbooks}.
Online use could ensure that all the documents were synchronized across different computers.
Offline use does not rely on working internet and the users does not need to update the documents directly in the browser, as they do if they use BlissBook.
This also allows the users to use their preferred text editing software.
Whether online, offline, or both will be used will be determined at a later state in the development.

\textit{Digital signature} and \textit{notifications} are also seems like very useful features
This is a feature provided by BlissBook, see \cref{sec:blissbook}.
% Anja kan du skrive noget smukt her? :-)

As the identified features has been evaluated the useful ideas are:
\begin{itemize}
	\item Managing different version of documents
	\item TOC
	\item Automatally updating the TOC
	\item Register when a document has been read
	\item Including a changelog
	\item Permission levels
	\item Highlighting differences
	\item Users retreive notifications when relevant documents are updated
\end{itemize}

%\subsubsection*{Do the ideas seem useful? Why?}
%
%Blissbooks and D4Handbooks are both based on the idea that document management is tedious and could be done easier.
%However, Blissbooks is angled more towards an HR-department while D4Handbooks is made specifically for QMS.
%This is exactly the purpose for which Ipsen intends to use the system, in that sense the ideas are useful.
%Furthermore, while many of the functionalities both of these systems offer to facilitate the goal of making document management easier mirror Ipsen's requirements they are also lacking in features.
%Both of them have implemented rich text editors to retain full control over the text entered. 
%This also enables Blissbooks to accurately highlight differences between versions, a feature that Ipsen has requested.
%Though as mentioned in \cref{sec:1-simpleTime} the build in editor is in no way wanted by Ipsen.
%
%Dropbox and Google Drive are both first and foremost cloud storage and secondly file sharing. 
%While these are both features that Ipsen could benefit from they are not her primary concern and the actual document management would work same as it currently does.
%
%\subsubsection*{Will the ideas work in your context? Why?}
%
%Some of the ways that Blissbook and D4Handbooks have implemented the features could be useful, particularly the registration of read statuses and notifications.
%The rich text editor serves its purpose in both of those systems, however, it would not work in this context.
%Ipsen has a backlog of many handbooks that would need to be imported and has stated that this specific feature would only serve as a nuisance since they would need to learn how to write in a new program.
%
%It is necessary to be able to use the files they already have.
%This is one feature that especially dropbox does handle very well but as stated before it does not bring any interesting ideas in terms of document management.
%
%\subsubsection*{Can the ideas be adapted to your system? How?}
%\todo[inline]{Henrik: Der skal staa noget om how}
%
%The useful ones absolutely can, and should too.
