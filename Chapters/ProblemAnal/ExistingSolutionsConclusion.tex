\subsection{Conclusion}
To compare the different systems, the table below is made to show which features are present in what system. These features gives an overview of what is fundamental for the systems, which helps prioritize what is important in the development of the system.
%Anna: er det virkeligt sådan det fungerer med 

The features they will be evaluated on, are the features requested by Ipsen.
% Status for:
 % \checkmark Blissbook
 % D4Handbooks
 % IPW

\begin{table}[H]
	\begin{center}
		\begin{tabular}{| m{5cm}|m{1.6cm}|m{2cm}|m{1.5cm}|m{1.2cm}|}
			\hline
			 & BlissBook  & D4-Handbooks & Dropbox & Google \newline Drive \\ 
			\hline
			Managing different versions of documents & \checkmark &  &  & \\ 
			\hline
			table of Contents & \checkmark & \checkmark  & \checkmark & \checkmark \\ 
			\hline
			Automatically updating the table of contents & \checkmark & \checkmark  & \checkmark & \checkmark \\ 
			\hline
			Approve contractors and \newline relevant documents &  &  &  & \\ 
			\hline
			Register when a document has been read & \checkmark & \checkmark &  & \\ 
			\hline
			Adjustable archive time &  &  &  & \\ 
			\hline
			Including a changelog & \checkmark & \checkmark  & \checkmark & \checkmark \\ 
			\hline
			Permissions levels & \checkmark &  &  & \\ 
			\hline
			Highlighting differences & \checkmark &  &  & \\ 
			\hline
			Documents from external \newline sources &  &  & \checkmark & \checkmark \\ 
			\hline
			Title and chapter numbers \newline reserved forever &  &  &  & \\ 
			\hline
			Does not require technical \newline expertise to manage & \checkmark & \checkmark  & \checkmark & \checkmark \\ 
			\hline
		\end{tabular}
	\caption{{\color{red} Husk at indsætte caption tekst}}\label{tab:Exsisting}
	\end{center}
\end{table}

%Taniya: TOC for dropbox og google drive?
%Taniya: vil også krydse Dropbox af ved Permissions levels
%Taniya: jeg tror at D4Handbooks er ikke begynder venlige og lige så intuitiv (de har en hel manual for hvordan man anvende deres system) som Dropbox og google drive
\todo{Hvis vi allerede nu i tabellen vil have det med supplier frem, skal den del skrives frem helt fra starten af og diskuteres under hvert system i eksisterende systemer}

Looking at this table we can answer the questions asked in the beginnign of this section, \cref{chap:existing}.

\textbf{Which ideas lay the groundwork for the system?}

The overview that the table of contents gives, is placed in every system that we are looking at which shows how fundamental it is to be able to know the contents of the inventory you are working with.
More so all our existing solutions also contain a change log that further expands the overview of the system. Finally, it helps not having a requirement in technical expertise to manage the system, as many users come with novice abilities in this field.
%Anna: Ikke sikker på denne besvarelse giver mening for mig (At jeg forstår den)

\textbf{Do the ideas seem useful? Why?}

This pattern indicates the usefulness of a easy to use with a good overview program, which gives users a much easier conversion from previous methods to newer systems.

\textbf{Will the ideas work in your context? Why?}
  
Yes, the ideas will help giving an edge when approaching the design of such system.

\textbf{Can the ideas be adapted to your system? How?}
  
Yes, the ideas that are apparent in the other solutions is a fundamental tool for such systems as such it will be adapted in consideration with Ipsens requirements of the system.

  
 % Taniya: skal summary afsnit ikke bare fjernes? Existing Solution skal afsluttet med en tekst som leder hen til system definition



