\subsection{Summary}
To compare the different systems, the table below is made to show which features are present in what system.

The features they will be evaluated on, are the features requested by Ipsen.

\begin{table}[H]
	\begin{center}
		\begin{tabular}{| m{5cm}|m{1.6cm}|m{2cm}|m{1.5cm}|m{1.2cm}|}
			\hline
			& BlissBook  & D4-Handbooks & Dropbox & Google \newline Drive \\
			\hline
			Managing different versions of documents & \checkmark &  &  & \checkmark \\
			\hline
			Table of Contents & \checkmark & \checkmark  & & \\
			\hline
			Automatically updating the table of contents & \checkmark & \checkmark  &  & \\
			\hline
			%Approve contractors and \newline relevant documents &  &  &  & \\
			%\hline
			Register when a document has been read & \checkmark & \checkmark &  & \\
			\hline
			Adjustable archive time &  &  &  & \\
			%Anna: Er det en ting vi vil have i tabellen da det ikke rigtigt er noget vi bruger eller forholder os til. Vi har ikke adjustable archive time men næsrmere at det er muligt at søge i archive efter en adjustable timeline?
			\hline
			Including a changelog & \checkmark & \checkmark  &  & \\
			\hline
			Permissions levels & \checkmark &  & \checkmark & \checkmark \\
			\hline
			Highlighting differences & \checkmark &  &  & \checkmark\\
			\hline
			Documents from external \newline sources &  &  & \checkmark & \checkmark \\
			\hline
			Title and chapter numbers \newline reserved forever &  &  &  & \\
			\hline
		\end{tabular}
		\caption{Feature comparison table}\label{tab:Exsisting}
	\end{center}
\end{table}
%\todo{Hvis vi allerede nu i tabellen vil have det med supplier frem, skal den del skrives frem helt fra starten af og diskuteres under hvert system i eksisterende systemer}

Looking at this table we can answer the questions asked in the beginning of this section, \cref{chap:existing}.

\textbf{Which ideas lay the groundwork for the system?}

BlissBook and D4-Handbooks primarily lay the groundwork for the future system, as these software solutions matches the problem that this project seeks to solve.
Both systems have been developed to manage handbook documents and likewise include Table of Contents, registers when a user had read the newest document version, include changelogs, and access rights.
BlissBook stands out of the two as this solution is able to manage different version of documents and highlights differences between versions.

The future system will also integrate these functionalitites as they would meet the requirements from Ipsen and the ISO-standards.

\textbf{Do the ideas seem useful? Why?}

As BlissBook and D4-Handbooks lay the groundwork for the system, there are useful ideas that both Dropbox and Google Drive supply.
Though Dropbox and Google Drive share functionalitites with BlissBook and D4-Handbooks, the two former are compatible with documents from external sources such as Word and PDF-files.
This is especially a useful feature, as Ipsen has stated that she would prefer to work with text editing programs that she's used to rather than learning to use a new one.

%This pattern indicates the usefulness of a easy to use with a good overview program, which gives users a much easier conversion from previous methods to newer systems.

\textbf{Will the ideas work in your context? Why?}

With BlissBook and D4-Handbooks laying the groundwork for the system it would be possible to integrate the external document compatibility functionality from Dropbox and Google Drive.
Here it could work as an option for the user to upload new versions of the document from the computer.

%Yes, the ideas will help giving an edge when approaching the design of such system.

\textbf{Can the ideas be adapted to your system? How?}

This is a matter of integrating the feature into the system from the beginning of the development process.

%Yes, the ideas that are apparent in the other solutions is a fundamental tool for such systems as such it will be adapted in consideration with Ipsens requirements of the system.
