\subsection{Summary}
To compare the different systems, the table below is made to show which features are present in what system.

The features they will be evaluated on, are the features requested by Ipsen, see \cref{sec:requirements}.

\begin{table}[H]
	\begin{center}
		\begin{tabular}{| m{5cm}|m{1.6cm}|m{2cm}|m{1.5cm}|m{1.2cm}|}
			\hline
			& BlissBook  & D4-Handbooks & Dropbox & Google \newline Drive \\
			\hline
			Managing different versions of documents & \checkmark &  &  & \checkmark \\
			\hline
			Table of Contents & \checkmark & \checkmark  & & \\
			\hline
			Automatically updating the table of contents & \checkmark & \checkmark  &  & \\
			\hline
			Register when a document has been read & \checkmark & \checkmark &  & \\
			\hline
			Including a changelog & \checkmark & \checkmark  &  & \\
			\hline
			Permissions levels & \checkmark &  & \checkmark & \checkmark \\
			\hline
			Highlighting differences & \checkmark &  &  & \checkmark\\
			\hline
			Documents from external \newline sources &  &  & \checkmark & \checkmark \\
			\hline
			Title and chapter numbers \newline reserved forever &  &  &  & \\
			\hline
		\end{tabular}
		\caption{Feature comparison table}\label{tab:Exsisting}
	\end{center}
\end{table}

Looking at this table we can answer the questions asked in the beginning of this section, \cref{chap:existing}.

\textbf{Which ideas lay the groundwork for the system?}
%Anja: Der mangler tekst her
\newline\indent
\textbf{Do the ideas seem useful? Why?}

Blissbooks and D4Handbooks are both based on the idea that document management is tedious and could be done easier.
However Blissbooks is angled more towards an HR-department while D4Handbooks is made specifically for QMS.
This is exactly the purpose for which Ipsen intends to use the system, and the exact reasoning behind the need for a new system, and in that sense the ideas are useful.
However, while many of the functionalities both of these systems offer to facilitate the goal of making document management easier mirror Ipsen's requirements they are also lacking in features.
Both of them have implemented rich text editors to retain full control over te text entered. 
This also enables Blissbooks to accurately highlight differences between versions a feature that Ipsen has requested.

Dropbox and Google drive are both first and foremost cloud storage and secondly file sharing. 
While these are both features that Ipsen could benefit from they are not her primary concern and the actual document management would work same as it currently does.

\textbf{Will the ideas work in your context? Why?}

Some of the ways that Blissbooks and D4Handbooks have implemented the features could be useful, particularly the registration of read statuses and notifications.

The rich text editor serves its purpose in both of those systems, however, it would not work in this context.
Ipsen has a backlog of many handbooks that would need to be imported and has stated that this specific feature would only serve as a nuisance as they would need to learn how to write in a new program.
They need to be able to use the files they already have.

This is one feature that especially dropbox does handle very well but as stated before it does not bring any interesting ideas in terms of document management.

\textbf{Can the ideas be adapted to your system? How?}

The useful ones absolutely can, and should too.
