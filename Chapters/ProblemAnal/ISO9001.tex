\section{ISO 9001}
\todo{This whole section seems quite irrelevant. We should probably focus on more concrete requirements that ISO has, like documenting users have read certain documents. I do not know if this information is in ISO 9001}

The ISO 9001 is a Quality Management System(QMS) that gives, if complied with, a certification showing customers that the organization have been checked by a special auditor. These auditors are not from ISO(International Standard Organization) themselves but from a certified third party.
This QMS is based on 7 Quality Management Principles(QMP):
\newline
1. Customer focus
\newline
Customer focus is used to set requirements for customer service and thereby increase quality of service.
\newline
2. Leadership
\newline
This QMP focuses on unity in the work and creating conditions that make people achieve the organization's quality objectives.
\newline
3. Engagement of people
\newline
Giving people of all levels in the organization motivation to enhance quality of work.
\newline
4. Process approach
\newline
Getting consistent results by increasing the quality of the working process.
\newline
5. Improvement
\newline
Successful organizations keep focusing on the improvement of the organization.
\newline
6. Evidence-based decision making
\newline
Making decisions based on gathered data and information.
\newline
7. Relationship management
\newline
An organization manages relationships with suppliers or similar people to gain sustained success.
\newline
\newline
These principles are used in the ISO 9001 to set the criteria for quality,
quality principles are broad which makes the ISO 9001 usable in many different areas of work.
The gained certification is not permanent, but needs to be renewed every few years.
\newline
Kilde: https://www.iso.org/iso-9001-quality-management.html	 	 	

