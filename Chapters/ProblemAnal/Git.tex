\subsection{Git}
Two handbook management systems have been discussed so far. In the following another type of system that is not directly focusing on handbooks will be explored. Git is free and open source version control system that is designed to handle all kind of projects of any size. Git was developed by Linus Torvalds in 2005. This chapter is based on information from gits own website \url{https://git-scm.com/}.

Git is easy to learn and, as mentioned above, free which is a great plus, especially for small companies that cannot afford to pay i.e. $50,000$dkk to use one of the more specialized handbook management systems.
%hvor har vi det her tal fra?
% Anja: "Git is easy to learn" SLEEET! Subjektiv mening. Tjah. Slet hele ovenstående paragraf faktisk, da det er irrelevant.
%Anna: enig med Anja meget subjektivt, vil personligt ikke mene det er nemt at lære men der er en stejlt læringkurve som for folk der ingen IT forståelse har bare vil være sværere at komme igennem

BlissBook and D4Handbooks both require its users to be online to manage their handbooks, D4Handbooks offers its users offline use "if they pay more" though, but Git does not require its users to be online. The users can easily manage their handbooks offline and then.. % noget med at synkronisere senere...
% though..

A disadvantage of using Git is that it does not offer a way to register if the employees have read the documents that is relevant to them.
%evt nævn brugervenlighed

%Anja: Vores git-sektion har brug for mere kærlighed