\subsection{IPW}
IPW advertise the fact that their system can be used "no matter your previous setup", a feature which the Ipsen has stressed is of great importance if the system should ever be put in use.
Making a thorough review of IPW presents the difficulty that the system is not publicly available, and that all the knowledge that can be acquired of it is based on sales material.
Nonetheless, this sections tries to ascertain the capabilities of the system.
The information used in this section is from IPW's own website\cite{IPW}.

IPW offer their services in several different modules, of which the most interesting one is the \textit{IPW Polaris}.
Its features can be expanded with several additional modules but the majority of the needed functionality is already available in this module.
This module is made specifically for quality assurance and to make handbooks in compliance with different standards, including ISO, see \cref{sec:ISOstandards}.

The system ticks several of the necessary boxes: 

\begin{itemize}
        \item 
        It handles different file types
        \item 
        There is a strict version control
        \item 
        Employees are shown relevant documents first
        \item 
        Relevant employees get a notification when a document is updated
        \item 
        There is a system in place for handling the process of approving revisions to the handbook.
\end{itemize}
%do all of these work the way pia wants them to? idk???

%Anna: er det følgende her virkeligt vigtigt at skrive på denne måde, er det ikke bedere at fjerne (Har i første gang udkommenteret det):
%"It does also have several features which Ipsen has not expressed any need for or dismissed as unnecessary.
%Among those are a way to clearly determine each employees' responsibility, the ability to delegate the responsibility for document maintenance and access on mobile and tablet devices."
An interesting feature is the ability to view relevant documents as flowcharts.
\todo[inline]{A concern is a) how does this even mean or work and b) whether this is unnecessary fluff, which would be of no value to Ipsen.}
\todo[inline]{We either need to figure out what we mean by this, or just kill the flowchart part}

As the need for a simple system has been repeatedly stressed all these features might be overdoing it.
The editing is also webbased, a fact that suggests that it does not handle existing Word documents for example, a feature that Ipsen stressed as incredibly important
%Anna: Men havde de ikke netop sagt at de kunne virke uanset tidligere setup?

%Anna: Da vi endnu ikke på nogen måde snakker om at approve supplier tænker jeg ikke dette afsnit hører sig til:
%"They do not have a module made specifically for approving a companys suppliers but it is possible that either the one described or one of their additional modules can be utilized for this purpose."
