\section{Standards}
% Rasmus - I think that it is also important to mention how standards play a role of ensuring trust between organisations, in that it can help ensure that everyone is doing stuff properly. Also, auditing is probably a process that should be mentioned, as that is part of standards workflow
A standard is in its core an agreed way of doing things. This is a broad term that applies to production, management, quality and most activities are done by organizations. So, the point of standards is to cover a wide variety of topics and give workers a basis for the expectation of their work. An organization might use a Quality Management Standard (QMS) to help efficiency and quality of production or use a Food Standard to keep the food from being rotten or more so to keep the quality of the food at level.

If we for example look at the building of a house, then there could be a technical standard that says that a house has a foundation, four walls and a roof. Then another quality standard could say that the foundation should be made of cement as it has a more long-term lifespan. More so there could be a safety standard that says that walls are not allowed to be built higher than a certain height because of the danger of collapsing. These are all simple examples of standards, but organizations like ISO has 22811 standards that organizations can apply in different situations.
\subsection{ISO 9001}
% Rasmus - This doesn't seem super concrete, and more like a sales speech for ISO. In the problem analysis, we should probably focus on what parts of ISO are essential to our project specifically.
The ISO 9001 is a Quality Management System(QMS) that gives, if complied with, a certification showing customers that the organization have been checked by a special auditor. These auditors are not from ISO(International Standard Organization) themselves but from a certified third party.
This QMS is based on 7 Quality Management Principles(QMP):
\newline
1. Customer focus
\newline
Customer focus is used to set requirements for customer service and thereby increase quality of service.
\newline
2. Leadership
\newline
This QMP focuses on unity in the work and creating conditions that make people achieve the organization's quality objectives.
\newline
3. Engagement of people
\newline
Giving people of all levels in the organization motivation to enhance quality of work.
\newline
4. Process approach
\newline
Getting consistent results by increasing the quality of the working process.
\newline
5. Improvement
\newline
Successful organizations keep focusing on the improvement of the organization.
\newline
6. Evidence-based decision making
\newline
Making decisions based on gathered data and information.
\newline
7. Relationship management
\newline
An organization manages relationships with suppliers or similar people to gain sustained success.
\newline
These principles are used in the ISO 9001 to set the criteria for quality,
quality principles are broad which makes the ISO 9001 usable in many different areas of work.
The gained certification is not permanent, but needs to be renewed every few years.
\newline
Kilde: https://www.iso.org/iso-9001-quality-management.html
\newline
\subsection{ISO}
% So ISO 9001 should be moved down here?
% Again, very sales-speech like - Rasmus
This section covers the most widely used standards worldwide [], which includes explanation of what ISO standards contain, their purposes and benefits, as well as why ISO standards are relevant to the handbook version control system. This section is based on [] unless otherwise stated.

International Organization for Standardization (ISO) is a non-governmental organization that develops and produces International Standards in various fields. The ISO standards are created by leading experts globally, so the knowledge is shared and negotiated. ISO seeks to meet the needs of the market; new standards will be developed if they are requested by an industry or consumer groups. An ISO standard is a document consisting of practical information, requirements, specifications, guidelines, and best practices.

The general purpose of ISO standards is to create safe, reliable and high quality products and services including improving the efficiency of the business which results in minimizing waste, errors and reducing production costs. This way business could gain consumer confidence, promote best practices to fill knowledge gaps, and optimize resource utilization. The ISO standards also provide solutions to challenges such as environment issues and sustainability.

There exist various types of ISO standards from Risk Management standards to Safety and Healthcare standards. The most well-known are listed below.

\begin{itemize}
	\item ISO 9000 Family - Quality Management
	\item ISO 22000 Food Safety Management
	\item ISO 14000 Family - Environmental Management

\end{itemize}

It is relevant to look further into the ISO 9001 which is a Quality Management System that is suitable to any business or organization regardless of size and industry. The ISO 9001 standard sets criteria for a quality management system that contributes to the improvement of efficiency and customer satisfaction. The Quality Management System is based on Quality Management Principles that contain customer focus, leadership and employee engagement, the process approach and continual improvement. Generally, the ISO 9001 serves as a base standard that is compatible with other standards of different specific sectors and industries.

ISO standards are voluntary. However, some are mandatory if for example referenced in regulation. These regulations differ across countries. For instance, the ISO 7010 standard for safety signs is accepted as the European Standard, which is mandatory for all EU countries [].

ISO standards are reviewed at least once every five years to ensure that the standards are relevant and up to date to the present market. This means the handbooks that were written using ISO standards as a starting point likewise need to be revised, so the new versions are updated in terms of ISO standards. Gradually, the handbook will contain numerous versions of documents. It is time consuming to manage and keep track of the current version manually without a version control system.

% Definitions:
% Resources utilization definition: this measures ‘how’ effectively your company is making use of the available resources.
% Sustainability means meeting our own needs without compromising the ability of future generations to meet their own needs.
% Business vs Organization: A business is one sort of organization, with the specific purpose of making money (taking in more income than it is spending). Therefore we say a business is a ‘commercial’ organization. An organization in general simply means a group of people working together in some sort of organized set up, working toward common goals.
