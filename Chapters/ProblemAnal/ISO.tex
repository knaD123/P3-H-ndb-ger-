\section{Standards} \label{sec:standards}
When a company wants a certain certification they must adhere to the rules set in the corresponding standard.
Standards are not exclusive to document management, but extend across several different fields and types of work.
The following section covers an explanation of standards, their purposes and benefits, as well as how these standards affects handbook management.

A standard is a technical specification that refers to a set of requirements, rules, guidelines or definitions \citep[p.~5]{Standard} and can be used as a tool to ease "communication, measurement, commerce, and manufacturing" \cite{Standardtool}.
In other words a standard is in its core an agreed way of doing things, enabling a common understanding.

Standardization benefits everyone from increased efficiency of the organization and product safety to simplifying people's everyday lives.
Standards provide employees a basis for the expectation of their work and the consistent product output leads to consumer trust \citep[p.~83]{Standardization}.

In quality management standards, a document management system is required, the purpose of which is to ensure the consistency and quality of the production.
Thorough document management is crucial when passing information between relevant parties and also ensures traceability.

The document management requirements are largely similar across several different bodies of standards and which specific standard a document management system strives to adhere to is largely irrelevant (see \cref{bilag:PiasPPT}).
However the rules set out in \cref{sec:CaseDescription} are based upon the ISO 9001 standard and therefore that organization is the starting point for further analysis on how standards work.

\subsection{ISO standards} \label{sec:ISOstandards}
ISO is a non-governmental organization that develops and produces international standards in various fields.
The standards are created by leading experts globally. \cite{ISOinfo}
ISO seeks to meet the needs of the market; new standards will be developed if they are requested by an industry or consumer group \cite{ISOdeveloping}, and while standards are revised every five years the document management requirements rarely undergo any changes \cite{ISOreviewedevery5years} \textcolor{red}{\cite{Ipsenfirstinterview}}.

Standards are generally speaking voluntary.
However they can be referenced in national regulations in which case they are mandatory for the companies under those jurisdictions. \cite{ISOreviewedevery5years}.

ISO 9001 is a standard for Quality Management Systems (QMS), and is suitable to any business or organization regardless of size and industry.
As the name suggests, the purpose is to maintain quality within the company. \cite{ISO9001}
The QMS is based on Quality Management Principles that contain customer focus, leadership and employee engagement, the process approach and continual improvement \cite{ISO9001-2}.

The contents of ISO 9001 are roughly covered below.
This information is selected based upon what Ipsen deemed as the most relevant parts of the standard, see \cref{bilag:PiasPPT}.

\begin{itemize}
	\item
	Documents can differ in format depending on what makes sense in the given situation. 
	Written documents, pictures and videos are all acceptable as long as they convey the necessary information.
	\item
	Each document must be uniquely identified.
	In this specific case that is typically done with numbers, name, and date or version number.
	\item
	Documents must be reviewed and approved to determine suitability and sufficiency for the organization.
	\item
	The documents must be available where they are needed e.g. for employees at a factory or at a office.
	In reality this often means that the documents are printed.
	\item
	The document must be sufficiently protected against loss of confidentiality and unintended modifications.
\end{itemize}

There are many other points that a company needs to consider how they will handle but where there are no specified instructions.

\begin{itemize}
	\item Who has access and where do they have access?
	\item How will the documents be stored so that they remain readable?
	\item How does the company make sure that every employee has up-to-date document?
	\item How long will they keep all of this information?
	\item How do they handle documents from external sources?
	\item How do they handle confidential documents?
\end{itemize}

As the documents should always be kept up-to-date there will gradually be multiple versions of each document.
It is time consuming to manage the documents and keep track of the current version manually without a version control system and manual control also poses a greater risk for mistakes than automatic control.
When examining the requirements in \cref{sec:systemdefinition} and \cref{sec:requirementsdefinition}.
