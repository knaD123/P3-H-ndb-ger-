\section{Standards} \label{sec:standards}
% TODO: indsæt kilden fra Pias interviews
This section covers explanation of standards, their purposes, and how these standards affect handbook management.
The following subsection covers the most widely used body of standards worldwide - \textit{International Organization for Standardization} (ISO) \cite{ISOworldwidemostused}, which includes explanation of what ISO standards contain, their purposes and benefits, as well as why ISO standards are relevant to the handbook version control system.

A standard is a technical specification that refers to a set of requirements, rules, guidelines or definitions \citep[p.~5]{Standard}.
In other words a standard is in its core an agreed way of doing things.
This provide a basic common understanding for organizations and people.
A standard is a tool to ease communication, marketing, measurement and manufacturing \cite{Standardtool}.
%ovenstående sætning giver ikke mening?
%Standards provide people and organizations with a basis for mutual understanding, and are used as tools to facilitate communication, measurement, commerce and manufacturing. Kan godt skifte ordet ease ud med facilitate?
For instance, standards provide employees a basis for the expectation of their work with the right standardized operating procedures.
This way the organization or company ensures the quality and consistency of the production, which results in the employees performing the task properly.
This leads to establish consumer trust and confidence as the product or the service lives up to the expectation.
The standardization benefits everyone from increased efficiency of the organization and product safety to simplifying people's everyday lives \citep[p.~83]{Standardization}.

There exists a standard concerning document management for a handbook.
Document management is crucial when passing information between relevant parties to ensure that the correct information reaches the end user.
Through document management it is possible to retrieve the correct information at a certain time which also ensures traceability.
This standard is used by Ipsen and will be described in \cref{sec:DMR}.

\subsection{ISO standards} \label{sec:ISOstandards}
ISO is a non-governmental organization that develops and produces international standards in various fields.
The ISO standards are created by leading experts globally, so the knowledge is shared and negotiated \cite{ISOinfo}.
ISO seeks to meet the needs of the market; new standards will be developed if they are requested by an industry or consumer group \cite{ISOdeveloping}.
However the document management requirements are rarely revised \cite{Ipsenfirstinterview}.
An ISO standard is a document consisting of practical information, requirements, specifications, guidelines, and best practices \cite{ISOreviewedevery5years}.

The general purpose of ISO standards is to create safe, reliable and high quality products and services including improving the efficiency of the business which results in minimizing waste, errors and reducing production costs \cite{ISOinfo}. %This way business could gain consumer confidence, promote best practices to fill knowledge gaps, and optimize resource utilization. The ISO standards also provide solutions to challenges such as environment issues and sustainability.
ISO standards are voluntary.
However, some are mandatory if for example referenced in regulations.
These regulations differ across countries \cite{ISOreviewedevery5years}.

There exist various types of ISO standards from Risk Management standards to Safety and Healthcare standards. The most well-known are listed below \citep{ISOmostpopularlist}.

\begin{itemize}
	\item ISO 9000 Family - Quality Management
	\item ISO 22000 Food Safety Management
	\item ISO 14000 Family - Environmental Management
\end{itemize}

It is relevant to look further into the ISO 9001 which also contains the document management requirement.
ISO 9001 is a \textit{Quality Management System} (QMS) that is suitable to any business or organization regardless of size and industry \cite{ISO9001}.
The ISO 9001 standard sets criteria for a QMS that contributes to the improvement of efficiency and customer satisfaction.
The QMS is based on Quality Management Principles that contain customer focus, leadership and employee engagement, the process approach and continual improvement \cite{ISO9001-2}.
Generally, the ISO 9001 serves as a base standard that is compatible with other standards of different specific sectors and industries.

\subsubsection{Document management requirements - ISO 9001} \label{sec:DMR}
In the project's context regarding the document management of a handbook there exists requirements that need to be met if a company has a certified management system. There exists management systems for various fields e.g. quality, environment, working environment, food safety and sustainability. However the requirements for document management does not vary much in those different standards.
% ledelsessystem = management system
The general requirements for document management are described below.

\begin{itemize}
	\item Documents can have different format and structure e.g. a text document, spreadsheet, pictures, videos and other file types. A fast food chain might for instance find it useful to have pictures in their handbook when demonstrating the proper composition of a burger.
	\item Documents must be uniquely identified. ISO 9001 does not require a specific approach. Generally the documents are identified with numbers, name, date and version number.
	\item Documents must be reviewed and approved to determine suitability and sufficiency for the organization. Often a document consists of fields with date and initials or names of the author and  the approver.
	\item The document must be available and accessible where it is needed e.g. for employees at a factory or at a office.
	\item The document must be sufficiently protected against loss of confidentiality and unintended modifications.
\end{itemize}

The organization has to consider how the requirements below is implemented.

\begin{itemize}
	\item Distribution and accessibility of the individual document: The organization must decide who needs access to a document and if it is possible in the work situation.
	\item Document storage condition and preservation: The organization needs to ensure that the documents are stored properly so they will remain readable. For instance, if the paper document is stored in paper, it should likewise be protected by the potential risk of water damage. If the document is stored electronically, the documents must still be readable after a system change or error.
	\item Version control: Consideration regarding management of changes and how to ensure everyone has the current version. The reader must be able to see what has changed between versions. This requirement may vary from standard to standard. This is particularly useful as it would not be necessary for employees to read the whole document, but instead only read the changes.
	\item Retention period: How long should each document be kept? Typically, this can vary from three to five years. This means there is a need to archive expired documents for a specific number of years. Accessing the archived documents is rare but the option must be available.
	\item Documents from external sources: These might be data sheet, legal requirement and so forth and they must likewise need be managed and identified. In case of confidential document from clients or other people the document must be protected against wrongful access.
\end{itemize}

% how Pia implemented ISO in the company that she is consulent for.
The typical approach when Ipsen writes a handbook to a company is to observe the daily work procedures and determine whether each procedure meet the standards.
Ipsen takes a starting point in the company and detect what they already have rather then carry standards over to the company.

%standard er en list af krav som man er certificeret et bestemt standard skal man opfylde de ting. man kan ikke gå under det. men kan godt tage flere krav som man synes er vigtige. Pia tager udgangspunkt i virksomheden og se hvad de har. hvis i gør tingene sådan og sådan og vi skriver ting ned sådan og sådan så sikre os at en beskrivelse. tjekker op og doubbelt tjek om vi har alle kravene i standard er overholdt. istedet for at tager standard og fører dem over til en virksomhed. Fordi vi tænker at hverdagen er det vigtigeste og lads os beskriver hverdagen som et udgangspunkt og bagefter tjekke om vi har husket alle standarden

ISO standards are reviewed at least once every five years to ensure that the standards are relevant and up to date to the present market \cite{ISOreviewedevery5years}.
This means the handbooks that were written using ISO standards as a starting point likewise need to be revised, so the new versions are updated in terms of ISO standards.

Gradually, the handbook will contain numerous versions of documents.
It is time consuming to manage and keep track of the current version manually without a version control system.
This is also one of the reasons why Ipsen is requesting a system that can handle all this for her.
When examining the requirements in \cref{sec:factorcriteria} and \cref{sec:requirementsdefinition}, it is easy to see that the requirements, specified by Ipsen \textbf{>> henvis til hvor det er i rapporten <<}, is almost the same as the requirements specified above.

% de krav som hun har er beskrevet anderledes og hun har også andre krav som ikke er ISO relateret. 