\section{Interview} \label{sec:interview}

To gather information about the problem area and the overall problem it was decided that
we would hold an interview with Pia to ask her about the overall problem, how the handbook is currently handled, and get an overview of how we are going to proceed from this point. This would allow us to get a view of her perspectives and what she perceives to be the core problems and the type of solutions she needs.

There were several aspects to consider in advance. First was the type of interview to hold, of which there are \textit{unstructured interviews, structured interviews}, and \textit{semi-structured interviews} \citep{interactionhci}. Each of these types have their pros and cons: A structured interview generally results in shorter and more precise answers, whereas an unstructured interview allows the conversation to be more spontaneous and results in more unexpected answers and more rich information \citep{interview}. The semi-structured interview was chosen as this is a middle ground between the two interview types in the spectrum. This both allows for the interviewer to ask a range of questions, but gives enough leeway to ask follow-up questions outside of the prepared questions depending on the interviewee’s answers.

In preparation for the interview an interview guide was made. An interview table was formed where research questions where formulated in the left-hand side and interview questions in the right-hand side. Here the research questions were academically formulated to give an overview of what kind of information the interviewer seeks, and the interview question is written in everyday language and seeks to answer the research question \citep{interview}. Structuring the interview guide this way makes it more manageable to see what kind of information we seek to acquire before we formulate the corresponding questions. It should be noted that it is only for the group members benefit to structure the interview guide this way, and that the interviewee doesn’t necessarily gets to see the guide or the questions. The interview guide can be accessed in Appendix xx.

Through the interview we sought to answer what the handbook currently looks like, how it’s handled, what problems exist with the current solution, and which kind of organization uses these handbooks. Furthermore, the interview sought to answer what the solution had to include and how it should solve the existing problems. This resulted in exploring what kind of solution Pia needs, with more specific questions to requirements, operating systems and who’s going to use it. The last part of the interview touched upon ideas the project group had generated beforehand and asked Pia if these ideas were potentially viable or desirable.

\subsection{Execution of the interview}

At the day of the interview it was decided beforehand that Pia gave the project group a presentation of the overall problem and which requirements the handbooks included, and the interview would commence afterwards. All group members were present for the presentation. After the presentation there was an open discussion between Pia and the group concerning about the handbook. The presentation and discussion afterwards were recorded and can be accessed in Appendix xx. When it was time for the interview only three group members remained as this would provide a less crowded feeling and a more natural conversation could occur.
