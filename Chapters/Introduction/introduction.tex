In most industries there are set of standards for how things are done.
These standards range from  entirely voluntary to required by law, depending on the legislation in the specific area.
Each standard has a set of requirements that a company has to adhere to in order to get a ceritificate stating so.
Whether this is done correctly is checked during an audit where a professional auditor examines the company.

This project is based upon a case presented by Pia Ipsen, a quality management consultant.
Quality management is a challenge across multiple industries: For instance Ipsen works with both a cheese slicing and a shipping company.
Generally speaking, quality management will involve document management.

Multiple document management systems already exist, but as none of them fit the needs of the companies she works with she does not use any of them.
Her priorities are:

\begin{itemize}
\item Simplicity. The learning curve should be fast and the system should support the workflow presently in use.
\item Cost. A reason to not use an existing system is that they are too expensive when compared to the expected output.
\end{itemize}

The documents in question being managed are those in the company handbook; a collection of instructions on how things are supposed to be done specific to the company.
Before a document is added to the handbook it has to go through an approval system.
To ensure that everyone has read the required document, a monitoring system is needed.
Additionally document management entails version and access control as well as the ability to provide access to earlier versions of the documents.
