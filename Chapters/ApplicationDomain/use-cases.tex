\subsection{Use cases} \label{sec:usecases}
\todo[inline]{Anna: Tikz taniyas seje statecharts}

\textbf{Access current handbook}
\\
\textit{Any actor} can access the handbook by going to the website where it is located where they need to log in with username and password.

If it is the first time they log in they are automatically led to the settings page.
Here they need to change their standard password and have the option to update their e-mail address and phone number.
They are redirected to the front page, when they click confirm.
If it is not the first time, they are led directly to the front page.

Documents relevant to this specific user are shown first.
Below all the chapters are visible.
A user can view the documents in a chapter by clicking unfold on that specific chapter.
Alternatively they can unfold all chapters by clicking unfold all.
When they click on a document a preview is shown if the file is a pdf.
Otherwise they have the option to download the file.
The changelog from last version to current version is shown above the preview.
Depending on the user's role further actions are available in this view (elaborated in other use cases).
To exit the view the user clicks outside the box or the x in the corner.
\begin{figure}[H]
	\centering
	\begin{tikzpicture}[align=center, scale=1.0, transform shape]
	%noder i billedet:
	\node(start)[n]{};
	\node(login)[predefined, right=2.8cm of start]{Login\\page};
	\node(main)[predefined, below=1.5cm of login]{Main page};
	\node(setting)[predefined, right=2.0cm of login]{Settings\\page};
	\node(info)[predefined, right=2.5cm of setting]{Information\\checked};
	\node(preview)[predefined, right=2.8cm of main]{Document\\preview};
	\node(end)[n, right=2.8cm of preview] {};
	\node(End)[c,fit=(end)] at (end) {};	
	%linjer i billedet:
	\draw[arrow](start.east)--node[above]{open\\OBHandbooks}(login.west);
	\draw[arrow](login.south)--node[left]{Normal\\login}(main.north);
	\draw[arrow](login.east)--node[above]{First time\\login}(setting.west);
	\draw[arrow](setting.north east)--node[above]{Fill out\\informations}(info.north west);
	\draw[arrow](info.west)--node[below]{invalid info}(setting.east);
	\draw[arrow](info.south west)--node[right, xshift=0.5cm]{Valid info}(main.north east);
	\draw[arrow](main.east)--node[below]{Select\\document}(preview.west);
	\draw[arrow](preview.east)--(End.west);
	
	
	\path[arrow] (main) edge [loop below, looseness=7] node[auto] {Unfold a chapter} (main);
	\path[arrow] (main) edge [loop left, looseness=3] node[auto] {Unfold all\\chapters} (main);
	\path[arrow] (preview) edge [loop below,looseness=4] node[auto] {Download}(preview);
	
	\end{tikzpicture}
	\caption{{\color{red}Access current handbook}}\label{fig:Use-access}
\end{figure}

\textbf{Add new file to the system}
\\
An \textit{administrator} or \textit{writer} can add a new a document by clicking an upload button on the front page.
They are then taken to a seperate menu, where they specify the file which they wish to upload.
They can select an existing chapter from a drop-down menu or create a new one. 
They can then either choose an existing ID and title or create a new one.
If they create a new one the uploaded file becomes the first version of this document.
If they choose an existing one the uploaded file is added to the document as the newest version.
Depending on the system settings they may also be prompted to write a changelog for the new version.
In both cases they will be prompted to choose the people who should approve this addition to the handbook.

After clicking 'Send for Approval' the view returns to the front page and they now have the document in their relevant documents.
They have the option to delete it until it has been approved.
After it can only be archived by adding a new version.

Another way to add a new version to the document is from the preview of a specific document.
In this case they are not prompted for chapter, existing ID or title as those are already defined.
The rest of the flow is the same.

\begin{figure}[H]
	\centering
	\begin{tikzpicture}[align=center, scale=1.0, transform shape]
	%Noder i billedet
	\node(start)[n]{};
	\node(main)[predefined, right=2.8cm of start]{Main page};
	\node(new)[predefined, below=1.5cm of main, xshift=-2cm]{New file\\upload page};
	\node(preview)[predefined, right=2.5cm of new]{Document\\ preview};
	\node(doc)[predefined, below=1.5cm of new, xshift=-2cm]{Create\\new Document};
	\node(ver)[predefined, right=2cm of doc]{Create\\new version};
	\node(title)[predefined, below=1.5cm of doc]{Title \& ID\\checked};
	\node(upload)[predefined, right=2.8cm of title]{Upload\\file};
	\node(log)[predefined, right=3.5cm of upload]{Changelog\\created};
	\node(approve)[predefined, above=1.5cm of log]{Approval\\ created};
	\node(sent)[predefined, above =1.5cm of approve]{Approval request\\sent};
	\node(end)[n, above=1.5cm of sent] {};
	\node(End)[c,fit=(end)] at (end) {};
	%Streger i billedet
	\draw[arrow](start.east)--node [above]{login to\\ OBHandbooks}(main.west);
	%Ud fra main page
	\draw[black](main.south)--(4.01,-0.8);
	\draw[arrow](4.03,-0.8)-|node[right, yshift=-0.5cm]{Press\\'Upload'}(new.north);
	\draw[arrow](4.03,-0.8)-|node[left, yshift=-0.5cm]{Select\\document}(preview.north);
	%ud fra new file
	\draw[black](new.south)--(2.02,-3.3);
	\draw[arrow](2.02,-3.3)-|node[right, yshift=-0.5cm]{Add new\\ID}(doc);
	\draw[arrow](2.02,-3.3)-|node[left, yshift=-0.5cm]{Select\\existing ID}(ver);
	%Andre pile
	\draw[arrow](doc.south)--node[right]{Add\\title}(title.north);
	\draw[arrow](title.east)--node[above]{Select file}(upload.west);
	\draw[arrow](ver.south)--node[left]{Select\\file}(upload.north);
	\draw[arrow](preview.south)--node[right]{Press 'Upload\\new version'}(upload.north east);
	\draw[arrow](upload.east)--node[above]{Add changelog}(log.west);
	\draw[arrow](log.north)--node[left]{Add\\approvers}(approve.south);
	\draw[arrow](approve.north)--node[left]{Press ' send\\to approval}(sent.south);
	\draw[arrow](sent.north)--(End.south);
	\draw[black](sent.south east)|-(2.5,-8.15);
	\draw[black](2.5,-8.15)-|(-1.5,-4.0);
	\draw[arrow](-1.5,-4.0)|-node[above, xshift=1.0cm]{Invalid\\info}(new.west);
	\draw[arrow](upload.north east)--(approve.south west);
	\end{tikzpicture}
	\caption{{\color{red}Add new file to the system -suggestion}}\label{fig:Use-AddDoc}
\end{figure}

\textbf{Approve new version or document}
\\
\textit{Any user} can be set as an approver to a document.
When a user has been set as an approver they get an email notification about this.
Once they log in the document is shown in their relevant documents section.
They can then click on it to view and approve.
They can also click approve straight from the front page.
Once they have approved the document it is updated with a new version.
The former version is labeled as inactive and can be accessed through the archive.

\textbf{Access archive}
\\
The \textit{administrator} can access the archive by clicking on the archive in the sidebar menu.
The archive looks like the front page:
All of the chapters and their titles are visible and each has a drop-down menu containing all of the documents.
When a document is clicked a list of versions all versions of this document shows along with all the changelogs.
The administrator accesses a specific version by clicking it.
If the version is a pdf a preview is shown.
If not a download option is also available.

\textbf{Add user}
\\
An \textit{administrator} can add a new user to the system.
This is done by clicking 'User administration' in the sidebar and choosing the 'Add user' button.
A user needs needs a user name and a name.
A name can either be an actual name or an identification number.
It is possible to provide contact information in the form of e-mail and/or phone number.
A standard password is automatically set.
Once the administrator has clicked 'Add user' the new user is in the system and the administrator is returned to the User administration view.

\textbf{Edit user information}
\textit{Any actor} can edit their own information by clicking the cogwheel in the right corner.
They have the option to add, remove or change their e-mail address and phone number.
They can also change their username and password.
The changes are applied when they click confirm.

\textbf{User management}
\\
The \textit{administrator} can change information about all users in the system.
They access all the information by clicking 'User management' in the sidebar.
By clicking a user they access a menu where they can change a user's name, e-mail address and phone number.
An administrator can also reset a user's password.
The changes take affect once they click confirm.
They can delete users both from that menu and the overview.
When a user is deleted only the name remains in the database.
When they delete a user they get a popup asking whether they are sure.
The user is deleted once they click confirm and the popup disappears.

\textbf{Mark document as read}
\\
A \textit{reader} is alerted to a new document they need to read through an email or sms notification.
They access the document by logging into the system and finding the document in their relevant documents.
Once they have the document open in preview they can mark the document as read by clicking the 'Read' button.
The reader can also elect to download the document to read it rather than read it in preview.
Should they do so the 'Read' button is also accesible from the front page.
A popup menu stating that they have read and understood the content of the version will pop up.
The reader then clicks 'Agree', the popup disappears and the read status has been saved.

\textbf{View who has read a document}
\\
The \textit{administrator} can view a specific document in preview.
Here a button called 'Read Status' is available and by clicking it a pop-up appears.
In the popup everyone connected to the document who has read it and who should have read it is listed by name.
They exit the popup by clicking outside it or on the x.

\begin{figure}[H]
	\centering
	\begin{tikzpicture}[align=center, scale=1.0, transform shape]
	%noder i billedet
	\node(start)[n]{};
	\node(main)[predefined, right=2.8cm of start]{Handbook front\\ page displayed};
	\node(preview)[predefined, below=1.5cm of main]{Document preview\\ displayed};
	\node(read)[predefined, right=3.2cm of preview]{Read status overview\\ pop-up displayd};	
	\node(end)[n, above=1.5cm of read] {};
	\node(End)[c,fit=(end)] at (end) {};
	%Pile i billedet
	\draw[arrow](start.east) -- node [above]{login to\\ OBHandbooks} (main.west);
	\draw[arrow] (main.south) -- node [left]{Select\\ document}(preview.north);
	\draw[arrow](preview.east) -- node[below]{Press 'read status\\ overview'} (read.west);
	\draw[arrow](read.north) -- (End) node[right=0.5cm]{Close};
	\draw[arrow](read.north west) -- node[above]{Press 'x'\\ to exit}(preview.north east);
\end{tikzpicture}
	\caption{Statechart diagram for the 'View who has read a document' use case}\label{fig:Use-ReadStatus}

\end{figure}

\textbf{Manage departments}
\\
An administrator can access 'Manage departments' through the sidebar.
When they add a new department they need to provide a name and are then returned to the 'Manage department' view.
They can then click a department to view a list of all connected users sorted by name and a list of all connected documents sorted by number.
They can click edit user and are redirected to a menu where all the users in the system are listed.
They can then tick them on and off to select which users are connected to the department.
Similarly an 'Edit departments' button exists where they are then redirected to a list of all active documents in the handbook.

Worth noting. The administrator can change a user's name while the user themself can change their user name.
