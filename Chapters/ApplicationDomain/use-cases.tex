\subsection{Use cases} \label{sec:usecases}
All of the actors have different needs of the system and different accessibility to the system. The various different workflows each of them has is described in ten use cases, which are covered in this section.

\subsubsection{Access current handbook}
The main functionality that every actor needs is access to the current handbook. How they do so is described in \cref{tab:access-handbook} and shown in \cref{fig:access-handbook}.

\begin{table}
\centering
\begin{tabular}{p{12cm}}
\hline
\multicolumn{1}{c}{\textit{\textbf{Access current handbook}}} \\
\hline
\textit{Any actor} can access the handbook by going to the website where it is located and logging in with username and password.
If this is their first login they are redirected to a settings page, where they need to change password and may update e-mail address and phone number.
When they confirm, they are redirected to the front page.
If it is not their first login, they are led directly to the front page.
Documents relevant to this specific user are shown first.
Below all the chapters are visible.
A user can view the documents in a chapter by clicking unfold on that specific chapter.
Alternatively they can unfold all chapters by clicking unfold all.
When they click on a document a preview is shown if the file is a pdf.
Otherwise they have the option to download the file.
The changelog from last version to current version is shown above the preview.
Depending on the user's role further actions are available in this view (elaborated in other use cases).
To exit the view the user clicks outside the box or the x in the corner.
\\\hline
\end{tabular}
\caption{Use case specification for 'access current handbook'}\label{tab:access-handbook}
\end{table}
\begin{figure}[H]
	\centering
	\begin{tikzpicture}[align=center, scale=1.0, transform shape]
	%noder i billedet:
		\node(start)[n]{};
		\node(login)[predefined, right=2.8cm of start]{Login\\page};
		\node(main)[predefined, below=1.5cm of login]{Main page};
		\node(setting)[predefined, right=2.0cm of login]{Settings\\page};
		\node(info)[predefined, right=2.5cm of setting]{Information\\checked};
		\node(preview)[predefined, right=2.8cm of main]{Document\\preview};
		\node(end)[n, right=2.8cm of preview] {};
		\node(End)[c,fit=(end)] at (end) {};	
	%linjer i billedet:
		\draw[arrow](start.east)--node[above]{open\\OBHandbooks}(login.west);
		\draw[arrow](login.south)--node[left]{Normal\\login}(main.north);
		\draw[arrow](login.east)--node[above]{First time\\login}(setting.west);
		\draw[arrow](setting.north east)--node[above]{Fill out\\information}(info.north west);
		\draw[arrow](info.west)--node[below]{invalid info}(setting.east);
		\draw[arrow](info.south west)--node[right, xshift=0.5cm]{Valid info}(main.north east);
		\draw[arrow](main.east)--node[below]{Select\\document}(preview.west);
		\draw[arrow](preview.east)--(End.west);
		%Loop arrows
		\path[arrow] (main) edge [loop below, looseness=7] node[auto] {Unfold a chapter} (main);
		\path[arrow] (main) edge [loop left, looseness=3] node[auto] {Unfold all\\chapters} (main);
		\path[arrow] (preview) edge [loop below,looseness=4] node[auto] {Download}(preview);
	
	\end{tikzpicture}
	\caption{Statechart diagram for the use case 'access current handbook'}\label{fig:access-handbook}
\end{figure}

There is a fork depending on whether it is the first time a user logs in because a new user has a username, a name and a standard password.
For security reasons this standard password needs to be changed as otherwise anyone could log into the system with only a username.
The first login is also an account activation through which the password is changed and an e-mail address and phone number may be added.

As described in \cref{sec:CaseDescription} the current system is excessively complicated and it was therefore included in the system definition, see \cref{factor}, that the new system should ''facilitate an overview of current [...] versions''.
This is taken care of at the front page.
The documents relevant to the user are shown first, a feature specifically requested by Ipsen, and below all the chapters are visible providing an overview of the content in the handbook.
If a more detailed view is necessary it is possible to unfold a single chapter or all of them.


\subsubsection{Add new file to the system}
As described in \cref{sec:classdiagramhandbook} the system consists of both documents and versions where documents are the structural entities enabling the organizational features of the system and versions contain the files and thereby the actual contents of the handbook.
Adding versions as additions to an existing documents as well as creating a new document along with a version is described in \cref{tab:add-file} and visualized in \cref{fig:add-file}.

\begin{table}
\centering
\begin{tabular}{p{12cm}}
\hline
\multicolumn{1}{c}{\textit{\textbf{Add new file to the system}}} \\
\hline
An \textit{administrator} or \textit{writer} can add a new a document by clicking an upload button on the front page.
They are then taken to a seperate menu, where they specify the file which they wish to upload.
They can select an existing chapter from a drop-down menu or create a new one. 
They can then either choose an existing ID and title or create a new one.
If they create a new one the uploaded file becomes the first version of this document.
If they choose an existing one the uploaded file is added to the document as the newest version.
Depending on the system settings they may also be prompted to write a changelog for the new version.
In both cases they will be prompted to choose the people who should approve this addition to the handbook.
After clicking 'Send for Approval' the view returns to the front page and they now have the document in their relevant documents.
They have the option to delete it until it has been approved.
After it can only be archived by adding a new version.
Another way to add a new version to the document is from the preview of a specific document.
In this case they are not prompted for chapter, existing ID or title as those are already defined.
The rest of the flow is the same.
\\\hline
\end{tabular}
\caption{Use case specification for 'add new file to the system'}\label{tab:add-file}
\end{table}
\begin{figure}[H]
	\centering
	\begin{tikzpicture}[align=center, scale=1.0, transform shape]
	%Noder i billedet
			\node(start)[n]{};
			\node(main)[predefined, right=2.8cm of start]{Main page};
			\node(new)[predefined, below=1.5cm of main, xshift=-2cm]{New file\\upload page};
			\node(preview)[predefined, right=2.5cm of new]{Document\\ preview};
			\node(doc)[predefined, below=1.5cm of new, xshift=-2cm]{Create\\new Document};
			\node(ver)[predefined, right=2cm of doc]{Create\\new version};
			\node(title)[predefined, below=1.5cm of doc]{Title \& ID\\checked};
			\node(upload)[predefined, right=2.8cm of title]{Upload\\file};
			\node(log)[predefined, right=3.5cm of upload]{Changelog\\created};
			\node(approve)[predefined, above=1.5cm of log]{Approval\\ created};
			\node(sent)[predefined, above =1.5cm of approve]{Approval request\\sent};
			\node(end)[n, above=1.5cm of sent] {};
			\node(End)[c,fit=(end)] at (end) {};
		%Streger i billedet
			\draw[arrow](start.east)--node [above]{login to\\ OBHandbooks}(main.west);
			%Ud fra main page
			\draw[black](main.south)--(4.01,-0.8);
			\draw[arrow](4.03,-0.8)-|node[right, yshift=-0.5cm]{Press\\'Upload'}(new.north);
			\draw[arrow](4.03,-0.8)-|node[left, yshift=-0.5cm]{Select\\document}(preview.north);
			%ud fra new file
			\draw[black](new.south)--(2.02,-3.3);
			\draw[arrow](2.02,-3.3)-|node[right, yshift=-0.5cm]{Add new\\ID}(doc);
			\draw[arrow](2.02,-3.3)-|node[left, yshift=-0.5cm]{Select\\existing ID}(ver);
			%Andre pile
			\draw[arrow](doc.south)--node[right]{Add\\title}(title.north);
			\draw[arrow](title.east)--node[above]{Select file}(upload.west);
			\draw[arrow](ver.south)--node[left]{Select\\file}(upload.north);
			\draw[arrow](preview.south)--node[right]{Press 'Upload\\new version'}(upload.north east);
			\draw[arrow](upload.east)--node[above]{Add changelog}(log.west);
			\draw[arrow](log.north)--node[left]{Add\\approvers}(approve.south);
			\draw[arrow](approve.north)--node[left]{Press ' send\\to approval}(sent.south);
			\draw[arrow](sent.north)--(End.south);
			\draw[black](sent.south east)|-(2.5,-8.15);
			\draw[black](2.5,-8.15)-|(-1.5,-4.0);
			\draw[arrow](-1.5,-4.0)|-node[above, xshift=1.0cm]{Invalid\\info}(new.west);
			\draw[arrow](upload.north east)--(approve.south west);
	\end{tikzpicture}
	\caption{Statechart diagram for the use case 'add new file to the system'}\label{fig:add-file}
\end{figure}

Whether a changelog is prompted during upload depends on the system settings.
This is because only some standards require it as explained in \cref{sec:CaseDescription}.

Adding approvers varies slightly depending on the user's role:
A writer has to add at least one administrator, while an administrator does not have to add anyone at all.
If the administrator elects not to add anyone they are automatically assigned as the sole approver of the document.
They still have to approve it but they can do so unilaterally

It is worth noting that it is also possible to add a new document without attaching any file.
The former design did not offer the option of attaching the file immediately but required that the user first made a new document and then made a version to attach to that document.
During the first usability test, see \cref{labellist}, this workflow was a source for confusion and therefore it was changed.
The reason that the empty document remains an option is that it provides the possibility to structure the handbook without necessarily having all of the contents.
Additionally it also serves to save an ID and title combination as these cannot be reused. 

\subsubsection{Approve new version or document} \label{sec:approve}
A file is not a part of the handbook before it has been approved.
Until then only the people assigned as approvers have access to the file and only through the notification horn in the upper right corner or the relevant documents section.
An administrator can also view all pending approvals.
This also means that the administrator has two ways of accessing the approvals while the others only have one.
The approval process is described in \cref{tab:approve} and visualized in \cref{fig:approve}.

\begin{table}
\centering
\begin{tabular}{p{14.5cm}}
\hline
\multicolumn{1}{c}{\textit{\textbf{Approve new version or document}}} \\
\hline
\textit{Any user} can be set as an approver to a document.
When a user has been set as an approver they get an email notification about this, if the user has specified an email address.
Once they log in the document is shown in their relevant documents section.
They can then click on it to view and approve.
They can also click approve straight from the front page.
Once they have approved the document it is updated with a new version.
The former version is labeled as inactive and can be accessed through the archive.
\\\hline
\end{tabular}
\caption{Use case specification for 'approve new version or document'}\label{tab:approve}
\end{table}

\begin{figure}[H]
	\centering
	\begin{tikzpicture}[align=center, scale=1.0, transform shape]
	%noder i billedet
		\node(start)[n]{};
		\node(main)[predefined, right=2.7cm of start]{Main page};
		\node(pending)[predefined, right=3.0cm of main]{Pending\\approval page};
		\node(approve)[predefined, below=1.5cm of pending]{Pending approval\\approved};
		\node(ver)[predefined, left=2.8cm of approve]{pending approval\\verison page};
		\node(active)[predefined, below=1.5cm of approve]{Version active\\in handbook};
		\node(end)[n, left=1.5cm of active]{};
		\node(End)[c, fit=(end)] at (end){};
	%streger i billedet
		\draw[arrow](start.east)--node[above]{Login to\\OBHandbooks}(main.west);
		\draw[arrow](main.east)--node[above]{Admin:Go to\\Pending approval}(pending.west);
		\draw[arrow](pending.south)--node[right]{Press\\'Approve'}(approve.north);
		\draw[arrow](main.south)--node[left]{Press notification button}(ver.north);
		\draw[arrow](pending.south west)--node[left, yshift=0.28cm]{Select an\\ approval}(ver.north east);
		\draw[arrow](ver.east)--node[above, xshift=0.45cm]{Press\\'Approve'}(approve.west);
		\draw[arrow](approve.south)--node[right]{Version approved\\by all approvers}(active.north);
		\draw[arrow](approve.235)--node[left]{Admin unilateraly\\approves version}(active.125);
		\draw[arrow](active.west)--(End.east);
		
	\end{tikzpicture}
	\caption{Statechart diagram for the use case 'Approve new version'}\label{fig:approve}
\end{figure}



If someone were to make a mistake in the process of adding a file to the handbook an approval can be deleted or edited.
It is assumed that if e.g. a wrong file was uploaded it would be discovered during the approval process and therefore not approved.
This is also the reason why even if the administrator can approve unilaterally they cannot approve in the same process as they upload:
They need to rethink whether this is correct.
Once the document has been approved it is added to the handbook and everyone can access it from the front page.

\subsubsection{Access archive}
As per the requirements in \cref{sec:CaseDescription} the documents cannot just be deleted when they are no longer needed in the handbook.
Therefore, when a version or a document becomes obsolete they are archived.
This may be because a more recent version is available or because the information they contain is no longer needed.
In the first case the old version is automatically added to the archive as soon as the new one has been approved.
In the second case an administrator has the power to archive a document without replacing it.
The archive is rarely accessed and is typically only needed during an audit.
Because of this only the administrator has access to it - it would only serve as a potential problem if everybody else could access it and accidentally read the wrong version.
When the administrator accesses the archive they do so through the workflow described in \cref{tab:archive} and visualized in \cref{fig:Use-archive}.

\begin{table}
\centering
\begin{tabular}{p{14.5cm}}
\hline
\multicolumn{1}{c}{\textit{\textbf{Access archive}}} \\
\hline
The \textit{administrator} can access the archive by clicking on the archive link in the sidebar menu.
The archive looks like the front page.
All of the chapters and their titles are visible and each has a drop-down menu containing all of the documents.
When a document is clicked a list of all versions of this document shows along with all the changelogs.
The administrator accesses a specific version by clicking it.
If the version is a pdf a preview is shown.
If not a download option is also available.
\\\hline
\end{tabular}
\caption{Use case specification for 'access archive'}\label{tab:archive}
\end{table}

\begin{figure}[H]
	\centering
	\begin{tikzpicture}[align=center, scale=0.85, transform shape]
		\node(start)[n]{};
		\node(main)[predefined, right=2.7cm of start]{Main page};
		\node(archive)[predefined, right=1.8cm of main]{Archive page};
		\node(download)[predefined, below=1.5cm of archive]{Downloaded\\archive};
		\node(ver)[predefined, left=2.8cm of download]{Version\\page};
		\node(end)[n, below=0.8cm of ver, xshift=2cm]{};
		\node(End)[c, fit=(end)] at (end){};
		\draw[arrow](start.east)--node[above]{Login to\\OBHandbooks}(main.west);
		\draw[arrow](main.east)--node[above]{Go to\\'Archive'}(archive.west);
		\draw[arrow](archive.south)--node[right]{Press\\'Download}(download.north);
		\draw[arrow](archive.south west)--node[left=0.6cm]{Select a\\version}(ver.north east);
		\draw[arrow](5.23,-3.1)--(End.north);
		\draw[black](ver.south)|-(5.25,-3.1);
		\draw[black](download.south)|-(5.25,-3.1);
		\path[arrow] (archive) edge [loop right, in=355, looseness=3]node[auto]{Unfold\\document} (archive);
		\path[arrow] (archive) edge [loop above, in=65, looseness=6]node[auto]{Unfold\\chapter} (archive);
		\path[arrow] (ver) edge [loop left, looseness=4]node[auto]{Download\\version} (ver);

	\end{tikzpicture}
	\caption{Statechart diagram for the use case 'access archive'}\label{fig:Use-archive}
\end{figure}


The technical requirement is that the documents should be kept usually between three and five years.
This resulted in a system definition stating that there should be an adjustable archive time.
However, as described in \cref{sec:CaseDescription}, it was discovered that in practice the files are usually kept indefinitely as deleting the files were too much work.
Additionally the file integrity is of upmost importance, see \cref{sec:ISOstandards} and this is also prioritized as Very Important in the design, see \cref{sec:architecturecriteria}.
As space is clearly not a problem, but the prospect of accidentally deleting a file is it was decided that deleting files from the archive is not an option.
Should somebody accidentally upload the wrong file this should be discovered before the file is approved and added to the handbook as per the approve file use case, \cref{sec:approve}.
After a file has been approved it will always be accessible either through the handbook or the archive.

The option to download the entire archive serves both as a way to backup the archive and as a means to revert back to the old system, should the user dislike the way this works.
The option of easily going back is important in order to convince the users to give a new system a chance.

Viewing the archive is also the only time at which an entire changelog for a document is shown.
The rarity of this situation is also the reason why it was argued that the changelog does not need its own class, described in \cref{sec:Classes}.

\subsubsection{Add user}
As the system monitors whether a user has read a document and because this data is a requirement in the standards, every worker in a company needs access to the system in order for it to work properly.
The administrator adds users through the workflow described in \cref{tab:add-user} and visualized in \cref{fig:add-user}.

\begin{table}
\centering
\begin{tabular}{p{14.5cm}}
\hline
\multicolumn{1}{c}{\textit{\textbf{Add user}}} \\
\hline
An \textit{administrator} can add a new user to the system.
This is done by clicking the ''User administration'' link in the sidebar and clicking the ''Create user'' button.
A user needs needs a username and a name.
A name can either be an actual name or an identification number.
It is possible to provide contact information in the form of e-mail and/or phone number.
It is also possible to specify which departments the new user belongs to.
A standard password is automatically set.
Once the administrator has clicked ''Add user'' the new user is in the system and the administrator is returned to the ''User administration'' view.
\\\hline
\end{tabular}
\caption{Use case specification for 'add user'}\label{tab:add-user}
\end{table}

\begin{figure}[H]
	\centering
	\begin{tikzpicture}[align=center, scale=0.85, transform shape]
		\node(start)[n]{};
		\node(main)[predefined, right=2.7cm of start]{Main page};
		\node(user)[predefined, right=2.8cm of main]{User management\\page};
		\node(create)[predefined, below=1.5cm of user]{Create user\\page};
		\node(update)[predefined, left=1.5cm of create]{User\\created};
		\node(end)[n, left=1.5cm of update]{};
		\node(End)[c, fit=(end)] at (end){};
		\draw[arrow](start.east)--node[above]{Login to\\OBHandbooks}(main.west);
		\draw[arrow](main.east)--node[above]{Go to 'User\\administration'}(user.west);
		\draw[arrow](user.south)--node[right]{Press\\'Create user'}(create.north);
		\draw[arrow](create.west)--node[above]{Press\\'save'}(update.east);
		\draw[arrow](update.west)--(End.east);
		\path[arrow](create) edge [loop below, in=255, looseness=4]node[auto]{Add\\information} (create);
		\path[arrow](create) edge [loop below, out=345, in=320, looseness=3] node[right]{Add\\departments} (create);
			
	\end{tikzpicture}
	\caption{Statechart diagram for the use case 'add user'}\label{fig:add-user}
\end{figure}


In early designs the user was activated through an e-mail address or phone number authentication.
However as seen in \cref{tab:Actor-reader} not all of the system's users have a work related e-mail address or phone number.
Instead the workaround of a standard password was introduced:
When the administrator creates a user they set their name and username and the system automatically gives every new user a standardized password.
This means that there is a timeframe in which anyone could access the archive with only a username.
Therefore this workflow assumes that the user will activate their account immediately after getting it.
This is obviously a security flaw but in order to satisfy the system requirements and make it accessible for all actors coupled with the fact that security has a low priority, see \cref{sec:architecturecriteria}, this workaround was eventually chosen.

During creation the administrator assgins the user both a username and a name.
The purpose of the username is for the user to log in.
The purpose of the name is to identify the user.
Depending on the procedure in the company this may be both their legal name or an identification number.

\subsubsection{Edit user information}
Both the administrator and the user themself can edit their information, however it differs which information they have access to. The user themself can edit their username, password and contact information. This is described in \cref{tab:edit-user} and visualized in \cref{fig:edit-user}.

\begin{table}
\centering
\begin{tabular}{p{12cm}}
\hline
\multicolumn{1}{c}{\textit{\textbf{Edit user information}}} \\
\hline
\textit{Any actor} can edit their own information by clicking the cogwheel in the right corner.
They have the option to add, remove or change their e-mail address and phone number.
They can also change their username and password.
The changes are applied when they click confirm.
\\\hline
\end{tabular}
\caption{Use case specification for 'edit user information'}\label{tab:edit-user}
\end{table}
\begin{figure}[H]
	\centering
	\begin{tikzpicture}[align=center, scale=1.0, transform shape]
		\node(start)[n]{};
		\node(main)[predefined, right=2.7cm of start]{Main page};
		\node(profile)[predefined, right=1.8cm of main]{User setting\\page};
		\node(update)[predefined, below=1.5cm of profile]{Profile\\updated};
		\node(end)[n, left=2.8cm of update] {};
		\node(End)[c,fit=(end)] at (end) {};	
		\draw[arrow](start.east)--node[above]{Login to\\OBHandbooks}(main.west);
		\draw[arrow] (main.east)--node[above]{Press the\\cogwheel}(profile.west);
		\draw[arrow] (profile.south)--node[left]{Press\\'save'}(update.north);
		\draw[arrow] (update.west)--(End.east);
		\path[arrow] (profile) edge [loop right, looseness=3]node[auto]{Change\\information} (profile);	
	\end{tikzpicture}
	\caption{Statechart diagram for the use case 'edit user information'}\label{fig:edit-user}
\end{figure}


The only information a user does not have access to is their name or identification as this is used in the read status data.
This data is used for audits and data integrity is therefore a bigger concern than for the rest of the user's informations.

During the interviews with Ipsen it was determined that it should not be possible for the user to turn off notifications.
However as not all users have e-mails or phone numbers these are optional to add.
It was decided that editing contact information wasn't the administrators responsibility.
However this means that while the users cannot turn notifications off they do have the option of removing their contact information which bears the same result. 
Other ways to handle this is discussed further in \cref{RememberToEditThisInUseCasesToo}.

\subsubsection{Manage users}
While every user can edit their own information the administrator also has access to edit other users' information through the 'User management' tab in the sidebar.
This is described in \cref{tab:user-management} and visualized in \cref{fig:user-management}.

\begin{table}
\centering
\begin{tabular}{p{12cm}}
\hline
\multicolumn{1}{c}{\textit{\textbf{Manage users}}} \\
\hline
The \textit{administrator} can change information about all users in the system.
They access all the information by clicking 'User Administration' link in the sidebar.
By clicking a user they access a menu where they can change a user's name, e-mail address and phone number.
An administrator can also reset a user's password.
The changes take affect once they click confirm.
They can delete users both from that menu and the overview.
When a user is deleted only the name remains in the database.
When they delete a user they get a popup asking whether they are sure.
The user is deleted once they click confirm and the popup disappears.
\\\hline
\end{tabular}
\caption{Use case specification for 'manage users'}\label{tab:user-management}
\end{table}

\begin{figure}[H]
	\centering
	\begin{tikzpicture}[align=center, scale=0.85, transform shape]
		\node(start)[n]{};
		\node(main)[predefined, right=2.7cm of start]{Main\\page};
		\node(user)[predefined,right=2.6cm of main]{User management\\page};
		\node(delete)[predefined, right=1.6cm of user]{User\\deleted};
		\node(end)[n, below=2cm of delete] {};
		\node(End)[c,fit=(end)] at (end) {};

		\node(update)[predefined, left=2.8cm of End]{User\\updated};
		\node(edit)[predefined, left=2.8cm of update]{Edit user\\page};
		\draw[arrow](start)--node[above]{Login to\\OBHandbooks}(main);
		\draw[arrow](main)--node[above]{Go to 'User\\management'}(user);
		\draw[arrow](user)--node[above]{Press\\'Delete'}(delete);
		\draw[arrow](delete.south)--(End.north);
		\draw[arrow](update.east)--(End.west);
		\draw[arrow](edit.east)--node[above]{Press 'save'}(update.west);
		\draw[arrow](edit.north east)--node[right=1.4cm]{Press\\'Delete'}(delete.south west);
		\draw[arrow](user.south west)--node[left=0.4cm]{Select a\\user}(edit.north);
		\path[arrow] (edit) edge [loop left, looseness=3] node[left] {Change\\information} (edit);
		\path[arrow] (edit) edge [loop below, looseness=4] node[below] {Change\\password} (edit);

	\end{tikzpicture}
	\caption{Statechart diagram for the use case 'manage users'}\label{fig:user-management}
\end{figure}


The administrator cannot change the user's username as this is login information. 
This means that an administrator can change a user's name and a user can change their own username.
However as there needs to be a way to access the account if the user forgets their own password and they don't necessarily have their contact information listed it is possible for the administrator to reset their password.

When a user is deleted their name/identification remains within the system for the sake of the read status data.
Everything else is removed.

\subsubsection{Mark document as read}
In order to log when users have read documents they need to inform the system when this has happened.
This is described in \cref{tab:mark-read} and visualized in \cref{fig:mark-read}.

\begin{table}
\centering
\begin{tabular}{p{12cm}}
\hline
\multicolumn{1}{c}{\textit{\textbf{Mark document as read}}} \\
\hline
% Henrik: Burde der ikke ogsaa staa "writer"?
A \textit{reader} is alerted to a new document they need to read through an email or sms notification.
They access the document by logging into the system and finding the document in their relevant documents.
Once they have the document open in preview, they can mark the document as read by clicking the 'Read' button.
The reader can also choose to download the document to read it rather than read it in preview.
Should they do so, the 'Read' button is also accesible from the front page.
A popup menu, stating that they have read and understood the content of the version, will pop up.
The reader then clicks 'Agree', the popup disappears and the read status has been saved.
\\\hline
\end{tabular}
\caption{Use case specification for 'mark document as read'}\label{tab:mark-read}
\end{table}

\begin{figure}[H]
	\centering
	\begin{tikzpicture}[align=center, scale=1.0, transform shape]
	%noder i billedet
		\node(start)[n]{};
		\node(main)[predefined, right=2.7cm of start]{Main page};
		\node(status)[predefined, right=2.8cm of main]{Document\\preview};
		\node(pop)[predefined, below=1.5cm of status]{Pop-up menu};
		\node(update)[predefined, left=2.5cm of pop]{Read status\\updated};
		\node(end)[n, left=1.5cm of update]{};
		\node(End)[c, fit=(end)]at (end){};
	%streger i billedet
		\draw[arrow](start.east)--node[above]{Login to\\OBHandbooks}(main.west);
		\draw[arrow](main.north east)--node[above]{Go through\\notifications}(status.north west);
		\draw[arrow](main.east)--node[below]{Select relevant\\document}(status.west);
		\draw[arrow](status.south)--node[left, yshift=-0.2cm]{Press\\'Read status'}(pop.north);
		\draw[arrow](pop.west)--node[below]{Press 'Agree'}(update.east);
		\draw[arrow](update.west)--(End.east);
	%loop i billedet
		\path[arrow](pop) edge [loop right, looseness=1]node[right]{Press\\'Cancel'} (status);	
	
	
	\end{tikzpicture}
	\caption{{\color{red}Mark document as read}}\label{fig:Use-Read}
\end{figure}

\todo[inline]{Anna kommentarer:Vær opmærksom på jeg har ændret en del pile i diagrammet \\husk kun writer og reader kan det her.\\ Husk som det ser ud i implementationen i dag er det ikke et pop-up vindue}

This is the only use case which the administrator cannot do.
As described in \cref{sec:Actors} this is because none af the actors have any use for this specific functionality and it is not necessary to log their readings.

It is the department head's job to make sure that everyone in their department has read the updated documents.
In the case where readers do not have any contact information listed, they do not get a notification and it is the department head's responsibility to make them aware of any changes.

The wording of the message ''has read and understood'' was agreed upon with Ipsen following a discussion that it is not enough to have read the document.

\subsubsection{View who has read a document}

This is described in \cref{tab:read-status} and visualized in \cref{fig:read-status}.

\begin{table}[H]
\centering
\begin{tabular}{p{14.5cm}}
\hline
\multicolumn{1}{c}{\textit{\textbf{View who has read a document}}} \\
\hline
The \textit{administrator} and \textit{writer} can view a specific document in preview.
Here a button called 'Read Status' is available and by clicking it a pop-up appears.
In the popup everyone connected to the document, who has read it and who should have read it, is listed by name.
They exit the popup by clicking outside it or on the \textit{x}.
\\\hline
\end{tabular}
\caption{Use case specification for 'view who has read a document'}\label{tab:read-status}
\end{table}

\begin{figure}[H]
	\centering
	\begin{tikzpicture}[align=center, scale=1.0, transform shape]
	%noder i billedet
		\node(start)[n]{};
		\node(main)[predefined, right=2.8cm of start]{Main page};
		\node(preview)[predefined, below=1.5cm of main]{Document\\ preview};
		\node(read)[predefined, right=3.2cm of preview]{Read status\\ overview pop-up};	
		\node(end)[n, above=1.5cm of read] {};
		\node(End)[c,fit=(end)] at (end) {};
	%Pile i billedet
		\draw[arrow](start.east) -- node [above]{login to\\ OBHandbooks} (main.west);
		\draw[arrow] (main.south) -- node [left]{Select\\ document}(preview.north);
		\draw[arrow](preview.east) -- node[below]{Press 'read status\\ overview'} (read.west);
		\draw[arrow](read.north) --node[left]{Press 'x'\\ to exit}(End) node[right=0.5cm]{};%{Close};
		%\draw[arrow](read.north west) -- node[above]{Press 'x'\\ to exit}(preview.north east);
	
	\end{tikzpicture}
	\caption{{\color{red}View who has read a document}}\label{fig:Use-ReadStatus}
\end{figure}

\subsubsection{Manage departments}\label{managedepartments}

This is described in \cref{tab:man-dep} and visualized in \cref{fig:man-dep}.

\begin{table}[H]
\centering
\begin{tabular}{p{14.5cm}}
\hline
\multicolumn{1}{c}{\textit{\textbf{Manage departments}}} \\
\hline
An \textit{administrator} can access 'Manage departments' through the sidebar.
When they add a new department they need to provide a name and are then returned to the 'Manage department' view.
They can then click a department to view a list of all connected users sorted by name and a list of all connected documents sorted by number.
They can click edit user and are redirected to a menu where all the users in the system are listed.
They can then tick them on and off to select which users are connected to the department.
Similarly an 'Edit departments' button exists where they are then redirected to a list of all active documents in the handbook.
\\\hline
\end{tabular}
\caption{Use case specification for 'manage departments'}\label{tab:man-dep}
\end{table}

\begin{figure}[H]
	\centering
	\begin{tikzpicture}[align=center, scale=1.0, transform shape]
	%Noder i billedet
		\node(start)[n]{};
		\node(main)[predefined, right=2.8cm of start]{Main page};
		\node(dep)[predefined,right=2.8cm of main]{Department\\main page};
		\node(add)[predefined, below=1.5cm of dep]{Add name of\\new department};
		\node(detail)[predefined, left=4.5cm of add]{Detail view\\of department};
		\node(editD)[predefined, below=1.5cm of detail,xshift=1.5cm]{Manage documents\\in department page};
		\node(editU)[predefined, left=0.5cm of editD]{Manage users in\\department page};
		\node(delete)[predefined, right=2cm of editD]{Department\\deleted};
		\node(update)[predefined, below=1.5cm of editU, xshift=2cm]{Department\\updated};
		\node(end)[n, below=1.9cm of delete] {};
		\node(End)[c,fit=(end)] at (end) {};	
	%streger i billedet
		\draw[arrow](start)--node[above]{Login to\\OBHandbooks}(main);
		\draw[arrow](main)--node[above]{Go to\\'Departments'}(dep);
		\draw[arrow](dep)--node[left=0.9cm]{Select a\\department}(detail);
		\draw[arrow](dep)--node[right]{Add a new\\department}(add);
		\draw[arrow](add)--node[above]{New department's page}(detail);
		\draw[arrow](delete)--(End);
		\draw[arrow](update)--(End);
		%ud fra detail view
		\draw[black](detail.south)--(1.65,-3.5);
		\draw[arrow](1.65,-3.5)-|node[left=1cm, below]{Press 'edit\\documents'}(editD);
		\draw[arrow](1.65,-3.5)-|node[left=0.9cm, below]{Press 'edit\\users'}(editU);
		\draw[arrow](1.65,-3.5)-|node[left=1.1cm, below]{Press 'delete\\department'}(delete);
		%Ned til updated
		\draw[arrow](1.32,-6.0)--node[right]{Press\\'Apply list'}(update.north);
		\draw[black](editU)|-(1.32,-6.0);
		\draw[black](editD)|-(1.32,-6.0);
	%loops i billedet
		\path[arrow] (editU) edge [loop left, out=208, in=190, looseness=4] node[below] {Add/remove\\documents} (editU);
		\path[arrow] (editD) edge [loop right,out=335, in=350, looseness=4] node[below] {Add/remove\\users} (editD);
	\end{tikzpicture}
	\caption{{\color{red}Manage departments}}\label{fig:Use-dep}
\end{figure}