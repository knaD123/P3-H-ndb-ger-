\begin{table}
\centering
\begin{tabular}{p{12cm}}
\hline
\multicolumn{1}{c}{\textit{\textbf{Access current handbook}}} \\
\hline
\textit{Any actor} can access the handbook by going to the website, where it is located, and logging in with username and password.
If this is their first login they are redirected to a settings page, where they need to change password and may update e-mail address and phone number.
When they confirm, they are redirected to the front page.
If it is not their first login, they are led directly to the front page.
Documents relevant to this specific user are shown first.
Below all the chapters are visible.
A user can view the documents in a chapter by clicking unfold on that specific chapter.
Alternatively they can unfold all chapters by clicking \textit{unfold all}.
When they click on a document, a preview is shown if the file is a pdf.
Otherwise, they have the option to download the file.
The changelog from last version to current version is shown above the preview.
Depending on the user's role, further actions are available in this view (elaborated in other use cases).
To exit the view the user clicks outside the box or the \textit{x} in the corner.
\\\hline
\end{tabular}
\caption{Use case specification for 'access current handbook'}\label{tab:access-handbook}
\end{table}
