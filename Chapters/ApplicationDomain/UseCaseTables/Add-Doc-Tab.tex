\begin{table}[H]
\centering
\begin{tabular}{p{14.5cm}}
\hline
\multicolumn{1}{c}{\textit{\textbf{Add new file to the system}}} \\
\hline
An \textit{administrator} or \textit{writer} can add a new a document by clicking an upload button on the front page.
They are then taken to a seperate menu, where they specify the file which they wish to upload.
They can select an existing chapter, from a dropdown menu, or create a new one. 
They can then either choose an existing ID and title or create a new one.
If they create a new one, the uploaded file becomes the first version of this document.
If they choose an existing one, the uploaded file is added to the document as the newest version.
Depending on the system settings, they may also be prompted to write a changelog for the new version.
In both cases, they will be prompted to choose the people who should approve this addition to the handbook.
After clicking 'Send for Approval' the view returns to the front page and they now have the document in their relevant documents.
They have the option to delete it until it has been approved.
After it has been approved, it can only be archived by adding a new version.
Another way to add a new version to the document, is from the preview of a specific document.
In this case they are not prompted for chapter, existing ID or title as those are already defined.
The rest of the flow is the same.
\\\hline
\end{tabular}
\caption{Use case specification for 'add new file to the system'}\label{tab:add-file}
\end{table}
