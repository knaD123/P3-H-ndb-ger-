\section{Usage}\label{sec:Usage}
\todo[inline]{Andreas: Vejleders tidligere kommentar omkring Usage afsnit skal rettes til}

In this section the usage of the system will be analyzed.
Who uses and how they use the system will be explored systematically through analyzing who the actors are, and the use cases of the system.
The actors consists of the different user groups who will be utilizing the system and the use cases will describe the different uses the system will potentially have.

The actors that have been identified within the case are \textit{administrator, writer}, and \textit{reader}.
%It should be noted here that the software in this context refers to the software that is under development throughout this project.
%This will outline what responsibilities that the system will eventually have when it is deployed.
The use cases of the system have been identified to be \textit{manage documents, manage users, manage departments, approve new documents, read status}, and \textit{track differences between document versions}.

An overview of the relationships between the actors and the use cases are shown through the table below, \cref{tab:UseCases}.
Here it is outline which actors have which responsibilities in relation to the system.

\begin{table}[H]
	\begin{center}
	\begin{tabular}{| m{10em} | c | c | c | c | c |}
		\hline
		& \multicolumn{3}{c|}{\textbf{Actors}} \\
		\hline
		\textbf{Use cases} & Administrator  & Writer & Reader \\
		\hline
		Manage documents & x & & \\
		\hline
		Manage users & & x & \\
		\hline
		Manage departments & & x & \\
		\hline
		Approve new docs & & x & x \\
		\hline
		Read documents & & & \\
		\hline
		Track differences\newline between document\newline versions & x & x &\\
		\hline
	\end{tabular}
	\end{center}
	\caption{ {\color{red}husk at sætte caption og label så den kan referes til} }\label{tab:UseCases}
\end{table}
\todo[inline]{Anja og Astrid: Efter actors og use cases er paa plads, saa skal vi lige kigge paa tabellen igen.}
\todo[inline]{Astrid: Tænker at vi er nødt til at tilføje en use case omkring at læse dokumenter eller noget fordi det ikke rigtigt giver mening at have en aktør der intet kan.}

In the rest of this section both the actors and use cases will be elaborated, starting with the actors.

\subsection{Actors}\label{sec:Actors}
\todo[inline]{Anna: Formater the shizzle}
\todo[inline]{Anna: Dræb highlightning}
\todo[inline]{Anna: Knæk din ryg}

\textbf{Administrator}
Goal: Manages document versions, users, and user departments.
\\
Characteristic: There is at least 1 administrator per handbook who administrates the documents and the users who has access to the handbook.
\\
Examples: The administrator has access to all parts of the system which includes the archive, users, editing users, updating the documents and so on.

\textbf{Writer}
\\
Goal: Update specific documents in the handbook.
\\
Characteristic: One or more employees who need to update one or more documents in the handbook.
\\
Examples: When someone, who is not the administrator, needs to update one or more documents they get temporary writer access rights.

\textbf{Reader}
\\
Goal: To read and understand documents in the handbook
\\
Characteristic: There are many readers who have various positions in the firm. Their shared characteristic is that they all need to read parts of the handbook.
\\
Examples: When a document has been updated all affected readers must read the newest version and understand the difference between the previous and current one.

\subsection{Use cases} \label{sec:usecases}
\todo[inline]{Taniya: Tegn nogle statecharts, og få Anna til at tikze dem}
\todo[inline]{Astrid: Konkretiser use cases}
\todo[inline]{Andreas: Sikr tiden real quick}
\todo[inline]{Rasmus: Lav nogle ordentlige aktører. Reader, writer, administrator}
\textbf{Manage documents}
\\
It is desired that a handbook containing several documents and their versions is managed by a software.
This includes a TOC from which users are able to have an overview of the documents.
There are active and inactive documents, where the active documents consists of the newest versions of the documents, and inactive ones are archived.
The system also provides the person adding new versions with the option to write a changelog, where the user can write why there are changes to the document.

\textbf{Manage users}
\\
It is the administrator and software's responsibility to manage users and their administrative rights.
The administrator updates the system by adding, editing or removing users, while the software stores the users and their data in a database.
There are three types of users with varying degrees of administrative rights.
The administrator is able to create, edit, and delete users.

\textbf{Manage departments}
\\
The administrator manages departments within the system.
The administrator is able to create, edit, and delete departments by adding or removing users from them.
A department is subscribed to one or more documents which are relevant to the users within the department.
A user can be within one or more departments at once.
The software is responsible for notifying the departments whenever a document has been updated.

\textbf{Approve new documents}
\\
Whenever a new version of a document is available, it needs to be approved by one or more persons.
The user(s) who is responsible for approving the document(s) could either be the manager, or blue-collar worker.
When a document has been approved it is updated within the system, the older document is archived and the TOC is updated.

\textbf{Read status}
\\
Whenever an updated document is available, the subscribed departments are alerted.
All users within the subscribed department must read and understand the new document version.
When this is done, they check off box which indicates to the system and secretary that the new document has been read.
\todo[inline]{Henrik: Henrik: Der skal generelt henvises til tidligere tekst + er lidt forvirret mht consultant}
