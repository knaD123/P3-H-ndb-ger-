\section{Usage}\label{sec:Usage}
\todo[inline]{Andreas: Vejleders tidligere kommentar omkring Usage afsnit skal rettes til}

In this section the usage of the system will be analyzed.
Who uses and how they use the system will be explored systematically through analyzing who the actors are, and the use cases of the system.
The actors consists of the different user groups who will be utilizing the system and the use cases will describe the different uses the system will potentially have.

The actors that have been identified within the case are \textit{administrator, writer}, and \textit{reader}.
%It should be noted here that the software in this context refers to the software that is under development throughout this project.
%This will outline what responsibilities that the system will eventually have when it is deployed.
The use cases of the system have been identified to be \textit{manage documents, manage users, manage departments, approve new documents, read status}, and \textit{track differences between document versions}.

An overview of the relationships between the actors and the use cases are shown through the table below, \cref{tab:UseCases}.
Here it is outline which actors have which responsibilities in relation to the system.

\begin{table}[H]
	\begin{center}
	\begin{tabular}{| m{10em} | c | c | c | c | c |}
		\hline
		& \multicolumn{3}{c|}{\textbf{Actors}} \\
		\hline
		\textbf{Use cases} & Administrator  & Writer & Reader \\
		\hline
		Manage documents & x & & \\
		\hline
		Manage users & & x & \\
		\hline
		Manage departments & & x & \\
		\hline
		Approve new docs & & x & x \\
		\hline
		Read documents & & & \\
		\hline
		Track differences\newline between document\newline versions & x & x &\\
		\hline
	\end{tabular}
	\end{center}
	\caption{ {\color{red}husk at sætte caption og label så den kan referes til} }\label{tab:UseCases}
\end{table}
\todo[inline]{Anja og Astrid: Efter actors og use cases er paa plads, saa skal vi lige kigge paa tabellen igen.}
\todo[inline]{Astrid: Tænker at vi er nødt til at tilføje en use case omkring at læse dokumenter eller noget fordi det ikke rigtigt giver mening at have en aktør der intet kan.}

In the rest of this section both the actors and use cases will be elaborated, starting with the actors.

\subsection{Actors}\label{sec:Actors}
%Anna: Vil evt foreslå at man under eksempler også gav eksempler på hvem i firmaet kunne hadve de forskellige roller?
\begin{table}[H]
	\begin{tabular}{l m{11.3cm}}
		\hline
		\multicolumn{2}{c}{\textbf{\textit{Administrator}}}\\
		\hline
		\textbf{Goal} &  Manages document versions, users, and user departments. \\
		 &  \\
		\textbf{Characteristic} & There is at least 1 administrator per handbook who administrates the documents and the users who has access to the handbook. \\
		 &  \\
		\textbf{Example} & The administrator has access to all parts of the system which includes the archive, users, editing users, updating the documents and so on. \\
		\hline
	\end{tabular}
\caption{Actor specifiactions for \textit{Administrator}}\label{tab:Actor-admin}
\end{table}

\begin{table}[H]
	\begin{tabular}{l m{11.3cm}}
		\hline
		\multicolumn{2}{c}{\textbf{\textit{Writer}}}\\
		\hline
		\textbf{Goal} & Update specific documents in the handbook. \\
	 	 &  \\
		\textbf{Characteristic} &  One or more employees who need to update one or more documents in the handbook. \\
		 &  \\
		\textbf{Example} & When someone, who is not the administrator, needs to update one or more documents they get temporary writer access rights.\\
		\hline
	\end{tabular}
	\caption{Actor specifiactions for \textit{Writer}}\label{tab:Actor-write}
\end{table}

\begin{table}[H]
	\begin{tabular}{l m{11.3cm}}
		\hline
		\multicolumn{2}{c}{\textbf{\textit{Reader}}}\\
		\hline
		\textbf{Goal} & To read and understand documents in the handbook \\
		 &  \\
		\textbf{Characteristic} & There are many readers who have various positions in the firm. Their shared characteristic is that they all need to read parts of the handbook. \\
		 &  \\
		\textbf{Example} & When a document has been updated all affected readers must read the newest version and understand the difference between the previous and current one. \\
		\hline
	\end{tabular}
	\caption{Actor specifiactions for \textit{Reader}}\label{tab:Actor-read}
\end{table}

\subsection{Use cases} \label{sec:usecases}
\todo[inline]{Taniya: Tegn nogle statecharts, og få Anna til at tikze dem}
\todo[inline]{Astrid: Konkretiser use cases}
\todo[inline]{Andreas: Sikr tiden real quick}
\todo[inline]{Rasmus: Lav nogle ordentlige aktører. Reader, writer, administrator}



\textbf{Access current handbook}
\\
Any actor can access the handbook by going to the website where it is located where they need to log in with username and password.
They are then redirected to the front page.
Documents relevant to this specific user are shown first.
Below all the chapters are visible.
A user can view the documents in a chapter by clicking unfold on that specific chapter.
Alternatively they can unfold all chapters by clicking unfold all.
When they click on a document a preview is shown if the file is a pdf.
The changelog from last version to current version is shown above the preview.
Depending on the user's role further actions are available in this view (elaborated in other use cases).
To exit the view the user clicks outside the box or the x in the corner.

\textbf{Add new document}
\\
An \textit{administrator} or \textit{writer} can add a new a document by clicking an upload button on the front page.
They are then taken to a seperate menu, where they specify the file which they wish to upload.
This file will serve as the first version of a new document.
They can select an existing chapter from a drop-down menu or create a new one. 
They can then either choose an existing ID and title or create a new one.
If they choose an existing one they do not add a new document but add a new version to the specified document.
If they do so they may also be propmpted to write a changelog.
This depends on the system settings.
In both cases they will be prompted to choose the people who should approve this addition to the handbook.
After clicking 'Send for Approval' the view returns to the front page.

\textbf{Add new version}
\\
An \textit{administrator} or \textit{writer} can add a new version through the flow presented in the 'Add new document' use case.
Alternatively they can click on a document from the front page.
The current version of that document is then shown.
They then click upload and are taken to a seperate menu where they add the file they wish to add, may add a changelog and people to approve the new version.
They then click 'Send for Approval' and the view returns to the front page.

\textbf{Approve version}
\\
When a user has been set as an approver they get an email notification about this.
Once they log in the document is shown in their relevant documents section.
They can then click on it to view and approve.
They can also click approve straight from the front page.
Once they have approved the document it is updated with a new version.
The former version is labeled as inactive and can be accessed through the archive.

\textbf{Access archive}
\\
The \textit{administrator} can access the archive by clicking on the archive in the sidebar menu.
The archive looks like the front page:
All of the chapters and their titles are visible and each has a drop-down menu containing all of the documents.
When a document is clicked a list of versions, the date they were added to the handbook, the date they were archived, the version number is shown and if one exists, the changelog.
When they click one version a file preview and download options are shown.

\textbf{Mark document as read}
\\
A \textit{reader} can mark that they have read a document by previewing the document and clicking the 'Read' button. 
Alternatively they can access this button from the front page.
A popup menu stating that they have read and understood the content of the version will pop up.
The reader then clicks 'Agree', the popup disappears and the read status has been saved.

\textbf{See who has read a document}
\\
The \textit{administrator} can view who has marked a specific document as read by previewing it in the system.
The information is also accessible through the Read Statuses sidebar option.
\todo[inline]{mere info om hvordan det her faktisk ser ud}

\textbf{Add user}
\\
An \textit{administrator} can add a user to the system.
This is done by clicking 'User administration' in the sidebar and choosing the 'Add user' button.
A user needs needs a username, a password and either an email or a phone number.
Once the administrator has clicked 'Add user' the new user is in the system and the administrator is returned to the User administration view.

\textbf{Edit user}
\\
All of the actors can change their own password and contact information in settings.
The \textit{administrator} can change others' contact information as well through the User management view.
An administrator can also elect to delete a user.
They must confirm that they wish to delete this user before the change takes effect.


\textbf{Add department}
\\
An administrator can access the 

\textbf{Edit department relations to user and document}

\textbf{Manage departments}
\\
The administrator manages departments within the system.
The administrator is able to create, edit, and delete departments by adding or removing users from them.
A department is subscribed to one or more documents which are relevant to the users within the department.
A user can be within one or more departments at once.
The software is responsible for notifying the departments whenever a document has been updated.


\textbf{Read status}
\\
Whenever an updated document is available, the subscribed departments are alerted.
All users within the subscribed department must read and understand the new document version.
When this is done, they check off box which indicates to the system and secretary that the new document has been read.
\todo[inline]{Henrik: Henrik: Der skal generelt henvises til tidligere tekst + er lidt forvirret mht consultant}
