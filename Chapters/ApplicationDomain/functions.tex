\section{Functions}
When creating a system it is vital to get an overview of the functions and to determine the complexity thereof.
This is done by analysis of functions, in which the first step is to identify all functions.
The result is a list of functions with a related complexity and type.
This section is based on Mathiassen, Munk-Madsen, Nielsen and Stage \citep[ch.~7]{Rod-Aalborg}, unless anything else is stated.

To commence the analysis, a look at the system definition and use cases is necessary, see \cref{sec:SystemDefinition,sec:usecases}, as it creates a template of functions that are ready to be analysed.
The next step is then to determine what type each function belong to, this means determining if it is an \textit{update, signal, read} or \textit{compute function}.
A description of each of these the function types can be found in the following item list, \citep[p.~140]{Rod-Aalborg}:
\begin{itemize}
	\item 
	''\textit{Update} functions are  activated by a problem-domain event and result in a change i the model's state.''
	\item 
	''\textit{Signal} functions are activated by a change in the model's state and result in a reaction in the context; \ldots''
	\item
	''\textit{Read} functions are activated by a need for information in an actor's work task and result in the system displaying relevant parts of the model.''
	\item
	''\textit{Compute} function s are activated by a need for information in an actor's work task and consist of a computation involving information provided by the actor or the model; \ldots''
\end{itemize}


% Astrid, jeg har tilføjet et par definationer. Det kan godt være at de skal rettes til (se side 140 i OOA&D) - Henrik
%Anna: har rettet dem til da de ikke var definitioner, så har lavet dem som citeret liste
\todo[inline]{Astrid: Hvad er the characteristics ved de forskellige ovenfor}
The most present type of function in the system, as shown in the function table below, are of the update type.
%These are described in \citep[p.~140]{Rod-Aalborg}
%as "Update functions are activated by a problem-domain event and result in a change in the model's state."\citep[p.~140]{Rod-Aalborg}


\todo[inline]{Henrik: Vi skal opdatere tabellen, og argumentere hvor vi har dem fra. Cref og analyser}

\begin{table}[H]
\centering
\begin{tabular}{lll}
	\hline
	Function						& Complexity & Type    \\
	\hline
	Add document					& Simple     & Update  \\
	Archive document				& Simple     & Update  \\
	Reactivate document				& Simple     & Update  \\  % Henrik: Lidt i tvivl om vi overhovedet gør det her, eller om der bare laves et nyt dokument?
	Delete document					& Simple     & Update  \\
	Add version						& Medium	 & Update  \\
	Approve version					& Simple     & Update  \\
	Mark version as read			& Medium     & Update  \\ % Henrik: I tvivl med den her - HAAALP!
	Archive version					& Simple     & Update  \\
	Add user						& Simple     & Update  \\
	Update user						& Simple     & Update  \\
	Delete user						& Simple     & Update  \\
	Request approval				& Medium     & Update  \\
	Delete approval					& Simple     & Update  \\
	Add department					& Simple     & Update  \\
	Delete department				& Simple	 & Update  \\
	Add user to department			& Simple     & Update  \\
	Remove user from department		& Simple     & Update  \\
	Add document to department		& Simple     & Update  \\
	Remove document from department & Simple     & Update  \\
	Notify user						& Simple	 & Signal  \\
	\hline
\end{tabular}
\caption{Function list}
\end{table}

The reason in this is in the nature of the system, as it primarily lies in management of users and documents.
This also means that the complexity is mostly simple as the margin for error is rather small in these cases.
This is because of the concreteness of the functions, as each has minimal amount of uncertainty when used.

%Only exceptions are "Manage documents", where the complexity is in the validation where overlapping of information is not allowed.
%Furthermore "Track differences between document versions" can be difficult as different file types can be used by the users.

The "Notify user" is of the type signal, a function type that responds to changes inside the system.

%"Track differences between document versions" is of the type read as it retrieves information from each document and presents the differences to the user.


% Taniya: mener manage documents ikke rigtig er en funktion i sig selv men  der findes flere funktioner som til sammen gør det muligt at manage documents?

%Anja: Manage users, Manage documents osv. er ikke funktioner, men use cases. Vi har lavet en tabel over funktioner i vores google drive mappe. Tabellen ovenfor skal lige have en overhaul og generelt mere beskrivelse ind af enten alle funktioner eller blot nøglefunktioner.

% Henrik: Synes dette kapitel virker meget "tyndt"?
% Henrik: Der mangler en "scope of application domain"/"summary" for hele application kapitlet :-)

% Anja: Generelt for afsnittet mangler der nok mere metatekst - f.eks. åbning og slutning
% Henrik: Anja please fix merge conflicts før du pusher :-P

\todo[inline]{Anja: Skriv noget metatekst}
