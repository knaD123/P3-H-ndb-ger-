\section{Functions}
When creating a system it is vital to get an overview of the functions and to determine the complexity of these functions. This is done by analysis of functions, in which the first step is to identify all functions. The result is a list of functions with a related complexity and type.

To begin the analysis a look at the system definition and use cases is necessary. The next step is then to determine what type each function belong to, this means determining if it is an update, signal, read or compute function.

The larger part of functions are of the update type, as the nature of the system primarily lies in management of users and documents. This also means that the complexity is mostly simple as the margin for error is rather small in these cases. Only exceptions are "Manage documents", where the complexity is in the validation where overlapping of information is not allowed. Furthermore "Track differences between document versions" can be difficult as different file types can be used by the users.

\begin{table}[H]
\centering
\begin{tabular}{lll}
Function                                    & Complexity & Type    \\
Manage documents                            & Medium     & Update  \\
Manage users                                & Simple     & Update  \\
Manage departments                          & Simple     & Update  \\
Manage suppliers                            & Simple     & Update  \\
Approve new suppliers                       & Simple     & Update  \\
Approve new documents                       & Simple     & Update  \\
Update TOC                                  & Simple     & Update  \\
Track differences between~document versions & Medium     & Read    \\
Notify user                                 & Simple     & Signal  \\
Read status                                 & Simple     & Update 
\end{tabular}
\caption{Function list}
\end{table}

