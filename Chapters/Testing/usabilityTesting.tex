\section{Usability Testing} \label{sec:usabilitytesting}
% Har svært ved hvornår jeg skal bruge usability testing eller usability test. Så det retter I bare til 

To gain a better understanding and detecting what issues could possibly arise when Ipsen interacts with the application, usability tests were conducted during the project. 
Usability testing results in possible usability issues which are guidelines to further improve the system's design. 

\subsection{Usability test setup and structure}
Before usability testing is conducted it is important to find out the purpose and objectives of the test. 
A task list will be defined based on the objectives of the usability test that the test subject need to complete using the application. The observations of the test consists of a test monitor, a screen video recording including audio and a video recording of the view of the PC screen and the test subject's interaction with the keyboard and the mouse. 
During the usability testing the test subject was asked to think aloud while performing a task to easily identify and categorize possible issues in \textit{critical}, \textit{serious} and \textit{cosmetic} which are described below \citep[p.~154]{brugervenligtwebdesign}.

\begin{itemize}
  \item Critical: An issue is classified as critical if the test subject are unable to continue and stopped completing tasks.
Critical issues can also involving high irritation while interacting with the application or if the test subject misunderstood one or more significant points of the application.
  \item Serious: If the test subject is significantly slowed down to completing tasks. 
This means the test subject has to use several minutes on a task.
	\item Cosmetic: When the test subject spends few minutes to completing a task and the issue is still should be solved. 
\end{itemize}

After a usability testing is conducted a debriefing session immediately  begins. 
Under debriefing session post-test questions will be asked for gain a deeper understanding of test subject's experience with the application overall. 
Here test subject is able to mention criticism of the interaction design and suggests solutions and design ideas. 
Besides a debriefing session's time frame should also allow for discussion if needed. 

Evaluation of the test will proceed as follows; while performing the usability testing the test monitor is taking notes of the possible issues and difficulties that test subject seems to encountered during interacting with the application. 
To better identify and categorize issues a screen recording with audio is reviewed and to ensure there is no missing observations.

\subsection{First usability test}
This section will describe the first usability test which was conducted on Ipsen and its results.

\subsubsection{Purpose and objectives of the test}
The majority requested functionalities for administrator role was implemented in the application at this point. 
Therefore the first usability test will only tested the administrator role part of the application.
The interaction between the test subject and the application must be validated to determine usefulness of the implemented functionalities. 
The purpose of the test is to identify general issues concerning the perception of the handbook application for further improvements of the application design.
The research questions, the method description and the task list can be found in \Cref{bilag:utestbilag}.

\subsubsection{Results}
The identified issues discovered during the usability test will be presented and categorized below in \Cref{tab:utest1} with explainations following. 

\begin{table}[H]
	\begin{center}
	\begin{tabular}{| c | m{21em} | c | c | c | c | c |}
		\hline
		Index & \textbf{Identified usability issues} & Critical  & Serious & Cosmetic \\
		\hline
		 1 & Confusion regarding add and upload a new document   & x &  &  \\
		\hline
		 2 & No overview of existing document ID when creating a new document &  & x & \\
		\hline
		 3 & Confusion when creating a new version to a document & x & &  \\
		\hline
		4 & No indication when a version is sent to approval & x & & \\
		\hline
		5 & No indication on which page is currently showing &  &  & x \\
		\hline
		6 & Negative document ID &  & x & \\
		\hline
		7 & Not able to correct/modify a document ID & & x &  \\
		\hline
		8 & Department's associated users only presents as email &  &  & x \\
		\hline
	\end{tabular}
	\end{center}
	\caption{Identified issues for the first usability test}\label{tab:utest1}
\end{table}

The identified issue '1' occurred when Ipsen tried to create a new document and add a version to the document. 
As Ipsen thought that it is possible to both add a document and upload at the same time. 
The application at the current state would work such that user need to add a document first by enter the document name, chapter ID and section ID. 

Afterwards the user needed to find the document in the handbook and upload a new version. 
This created confusion which cause an interrupted work flow with the handbook.

The identified issue '2' took place when Ipsen attempted to create a new document. 
Ipsen needed to go back to the handbook's overview to check the existing document ID to be able find out which chapter the new document should belong to. 

The identified issue '3' occurred when Ipsen tried to create a new version to an existing document. 
It was unclear where to upload a new version. 
Therefore Ipsen tried to add a new document and then uploaded a new version. 
The upload button was not noticeable and took time before Ipsen found it. 

The identified issue '4' turned up as Ipsen tried to send a version to approval and then  approve the version herself.
Ipsen did not detect when she had sent the version to approval or whether she just approved the version. 
An indication or a message displayed to the user after a version is sent to approval or that the version is now approved and now are available in the handbook should solve this issue.

As identified issue '5' point out the confusion that occurred while Ipsen attempted to approve a version could be that the sidebar did not indicate which page she was currently on. 

The identified issues '6' and '7' appeared when Ipsen made a mistake as the application allowed her to enter a negative chapter and section ID. 
Ipsen felt the need to correct the mistake which the application did not support at the moment.
Ipsen wished to be able to delete a document that do not contain a version yet if a mistake occur.

The identified issue '8' occurred as Ipsen assigned a user to a department and noticed that users was presented as email address.
Ipsen preferred name instead of email address.

In the debriefing session Ipsen mentioned that she was surprised that the application required user to attached a pdf file as she has seen other document version control system has their own file editor.
However, Ipsen could see the reason behind it and commented that it is a good approach.
%Anna: Evt. kommenter at hun i virkeligheden selv havde sagt dette ikke skulle være en del af systemet til at starte med (evt henvis til iteration 1)

Ipsen also remarked that the features' names were clear and understandable. 

During the test session it had been clarified that administrator must be able to use the administrator right to get an awaiting approval approved without the other approvers had been through it.
The situations where other approvers either do not get it done, are too slow or are on a vacation could occur. 
The administrator right will therefore prevent possible bottleneck in the work flow.
%%% Andre kommentarer %%%
%Afdelingsleder holder styr på hvem der har læst hvad
%Kom frem til at admin skal kunne bruge sin admin ret til at få det godkendt hvis andre som blev sat til approvals ikke få det gjort (Kunne måske være relevant at kommenterer da det var noget vi tog til efterretning til næste test ) 
%Det virker ulogisk for pia i starten at den gældende version også i arkiver og ikke kun under handbook. Men synes til sidst at det smart når man skal audit og så det er nemt at finde frem (Anna: Tror jeg i stedet tager denne ned i iterationsafsnittet i stedet for hvis det er ok )

%en version er kun uploadet -- kan man slet den 
%en version som er blevet godkendt -- arkveret 

%Admin har en overrolloing hvis folk ikke går ind og godkender. Hvis folk tager for lang tid om det eller hvis folk er på ferie. (Anna: Samme kommentar som du har i starten af de her kommentarer )

\subsection{Second usability test}\label{secondtest}
The second usability test was conducted on Ipsen, Ipsen's husband and their daughter. 
The reader, writer and administrator roll were all been tested in the application.

\subsubsection{Purpose and objectives of the test}
The usability issues from the first test has been corrected and ready to be tested to determine whether the new functionalities are improved or not. 
All the rolls need to be tested as well as the recently added functionality which are required by the system definition. 
The department and user management in the application will also be a part of the testing. The task list for the second usability test can be found in \Cref{bilag:utestbilag}.

\subsubsection{Results}

The detected usability issues are present and categorize in table 
\Cref{tab:utest2} which will be clarified afterwards. 

\begin{table}[H]
	\begin{center}
	\begin{tabular}{| c | m{19em} | c | c | c | c | c | c |}
		\hline
		Index & \textbf{Identified usability issues} & Role & Critical & Serious & Cosmetic \\
		\hline
		 1 &  & & & &  \\
		\hline
		 2 & & & & & \\
		\hline
		 3 & & & & & \\
		\hline
		4 & &  & & &\\
		\hline
		5 & &  &  &  &\\
		\hline
		6 & &  &  & &\\
		\hline
		7 &  & & &  &\\
		\hline
		8 &  &  &  & & \\
		\hline
	\end{tabular}
	\end{center}
	\caption{Identified issues for the second usability test}\label{tab:utest2}
\end{table}

%Skriv her hvordan usability test slet ikke var klar
The preparation for the second usability test was not done thoroughly because of the time pressure and poor planning. 
The documents in handbook that test subject need to find according to the task list were not the same in the application. 
Since the application did not been tested thoroughly to check whether the test subject would be able to complete and ensure no error. 


%\begin{itemize}
%\item Ipsen does not think that notification button are not visible
%\item Confusion regarding upload a new version whether to find the file in the application or the system 
%\item Ipse did not know the purpose with the working file 
%\item Skal kunne tilføje flere approvers
%\item Når filen ikke er en pdf fil burde man kunne download filen fordi det vises ikke i preview.
 %\item Department det kunne være rart at kunne se dok id når man associate doks
%\item Ipen er forvirret over upload og download ikoner uden tekst - skal tænke over det. fortrækker en knap med tekst i stedet for med som add og get - så er der ingen tvivl om det 
%\item ikoner: 
%\item Hvis Ipsen skulle gå ind og godkende en dok så vil hun gå ind og læse det først (åbne og check om det var den rigtige dokument der blev lagt ind) også godkender det - mangler en download knap og approved knap under Approval - dok preview	
%\item - Det skal tydeliggjores at man har godkendt en version og venter på at de andre skal godkender dem.  se mockup
%\item Ipsen ændre Kjelds navn både i name and username- Admin skal ikke have adgang til at ændre i username
%\item upload icon inde i handbook page er ikke helt den rette tankegang i workflow, det vil passe bedre ind hvis man er inde i documentet og skal til at uoload dok, det er som om man har sprunget et trin over 
%\item Ipsen foretrækker at kunne se hvad en reader er blevet sat på til at læse og det bruger mangler at læse - kan det jo ske at hun bliver sat forkert på et  document .- kommet frem til at have en liste af de afdelinger en bruger er koblet op på og en liste over alle de dok brugeren mangler at læse på de aktive dok 
%\item kun admin der må arhciveret et dok fra håndbogen 
%\item Department heads writer skal have adgang til readstatus %overview. 
%\item Perspektiveret: bliver enige om at når man laver en %department at man markerer en bruger en som department %heads og det er dem som får notifikationer.  
% \end{itemize}

%Debriefing: Ipsen vil gerne finde karoline og se hvad hun ikke har læst.
%Gennem en read status: hvis man kan se at en reader ikke har læst sin ting: 1 uge bliver department alerted og 2 uger bliver admin alerted
%Ipsen vil gerne kunne tilføje hvorfor admin har slettet en bruger.

%Ipsen har det fint med bare pdf fil og ingen working file. fordi det skal være så simple som muligt 

%Henrik forklarer principet med en ny tilføjet bruger, og hvad aktiveret bruger og email adresse betyder og hvordan processen fungerer 


% synes at afdelingsleder skal sørger for at folk har læst. 

% Mangler at teste read status til 
% approvers til kunne tilføje flere
% approval id er ikke dok ide  
% add doc til department: mangler en cancel knap og doc id i listen


%Second usability test var dårligt opsat, gruppen var tidspresset og fik ikke gennemgået task list ordenligt. Mismatch mellemndok ID  hvad der er inde i application og task list. Notification del er ikke blevet testet da vi nåede ikke sæt det op.
%Department er blevet lavet bagefter. 
%Det med at beder bruger om at rette i en fil uden for system er forvirrende så det skal man undgå og bare beder dem at upload en dummy fil 

%Prefererly pdf skal ændres 

%Ændre brugernavn var ligetil 
%Ipsen kunne finde changelog nemt.

%Ipsen ønsker at systemet også kan ændre side hoved i selve fil/dokumentet , hvor der står udgave og dato på - når man printe det ud, så kan man se at det er den opdateret version de har der. Forslag: ku man når man upload dokumentet at du i virkelighed send et blank header med pdf format og så bliver det udfyld og også i håndbogen? i dont get that



%Anders: 

%-Ved ikke hvad read status er 
%-Astrid guider Anders for meget
%-Når man skal tilføje flere approvers så, er det ikke intuitive at man skal separert dem med email 
%-Det er ikke tydeligt kun med ikoner, især  upload og download 
%-Anders kunne tænke sig at pilen hightlighter de row pilen peger på, så man ikke kommer til at trykke på delete på en forkerte row 

%- Department: add user, der skal stå navn og ikke brugernavn 
%-Ander nævner at de kunne være rart at se hvem man er logget ind som 
%- Hvorfor har mette en mulighed for at delete en fil i håndbogen ?

%Tove:
%Det er godt at høre at alle er forvirret her
%Errors er en joke her 

%Der mangler en unfold all dok under full handbook 

%Lægger mærke til at layout og farven ikke er det samme som de andre pc? 

%Problemer med at opret en ny dok 

%Godkend dok er ikke tydeligt for tove 
%Tove ændrer både Kjelds navn og brugernavn
%Archive ser mærkelig ud?

%Slettet bruger skal ikke være under add user under department 

%Department var meget nemt at forstå

%Udført på den forkerte version af programmet

\subsection{Usability issue that are not a bug}

% Sidste afsnit: brugervenlighed issues som ikke er bug


% objective: something that you plan to do or achieve
% Goal is something which we strive to achieve.
% Purpose is something that influences goal. Purpose is the reason for achieving the goal.
% Objective is the specific action which one try to achieve as a short term plan.

%to gain more insight of the test subject understanding of 

