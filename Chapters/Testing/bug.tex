\section{Bugs}\label{sec:Bug}
% Hvad er en bug
There exists a lot of terms to describe that a program is not behaving as intended or crashes, like; \textit{Fault, failure} and \textit{incident}.
A reason for all these different terms is that e.g. incident might not sound negative enough to be used for very severe or dangerous errors. \cite{SoftwareTesting}

The term \textit{bug} will be used throughout this chapter to avoid any confusions or even misunderstandings.
A bug occurs when one of these five rules is true: \cite{SoftwareTesting} % Denne sætning skal muligvis omformuleres

\begin{enumerate}
	\item \textit{''The software doesn't do something that the product specification says it should do.''}
	\item \textit{''The software does something that the product specification says it shouldn't do.'''}
	\item \textit{''The software does something that the specification doesn't mention.''}
	\item \textit{''The software doesn't do something that the product specification doesn't mention but should.''}
	\item \textit{''The software is difficult to understand, hard to use, slow, or - in the software tester's eyes - will be viewed by the end user as just plain not right.''}
\end{enumerate}

% Hvorfor opstaar bugs (side 16)
Studies keeps showing over and over that the number one cause of bugs is the product specification.
There can be multiple reasons for this: \cite{SoftwareTesting}

\begin{itemize}
	\item It has not been written
	\item It is not thorough enough
	\item It keeps being updated
	\item It is not communicated well enough
\end{itemize}

Other causes to why bugs happens in software are the design and code errors.
The reasons to why bugs are introduced to the software by the design and code errors are similar to the reasons mentioned above.
Though, bugs caused by code errors can also happen because of the complexity of the code or poor documentation in the code.
Many of the bugs caused by code errors can often be traced back to errors in the specification and design. \cite{SoftwareTesting}
%Anna: evt i en bisætning (eller bare kort sætning) gør klart hvad der ligger i product specifications?

% Pris for at fixe det (side 18)
Finding bugs early in the development process can reduce the cost of fixing it greatly.
A development process can be split up into multiple phases:
%Evt referer til workflow afsnittene om waterfall og iterative method

\begin{itemize}
	\item Specification
	\item Design
	\item Programming
	\item Testing
	\item Release
\end{itemize}

The cost of fixing bugs increases logarithmic over time.
This means, a bug that is found while making the specification might cost nothing to fix, a bug found during the programming phase might cost thousands of dollars to fix and bugs found by the users could potentially cost millions of dollars to fix. \cite{SoftwareTesting}

% Henrik: Kan der indsættes en graf der viser dette?

% Ikke alle bugs vil blive fixet (side 42)
Finding a bug does not automaticly mean that it will be fixed before the software is released.
This however, does not mean that the released software will be of poor quality.
There are several reasons why a bug is not being fixed before the software is released: \cite{SoftwareTesting}

\begin{itemize}
	\item There is not enough time to fix it
	\item It is too risky to fix it
	\item It is not worth it to fix it
	\item It is not a bug
\end{itemize}

% "It's not a bug, it's a feature"
One might think, when looking at the last point in the list above, why a bug is not a bug.
The reason for this can be because of mis-understandings or that the specification changes after the software already have some features implemented. \cite{SoftwareTesting}
