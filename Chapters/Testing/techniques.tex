\section{Testing techniques} \label{sec:testingtechniques}
A lot of different techniques can be used to test software.
This section will focus on two common approaches to test software \textit{black-box testing} and \textit{white-box testing} \cite{SoftwareTesting}.
It will be explained what their differences are, when to use them and how they are used in this project.

% static and dynamic testing (side 56)
When doing black-box and white-box testing it is possible to do it staticly or dynamicly, see \cref{defn:statictesting,defn:dynamictesting}.

\begin{defn} \label{defn:statictesting}
Testing software without running it
\end{defn}

\begin{defn} \label{defn:dynamictesting}
Testing software by running it
\end{defn}

% Black-box testing (side 55)
\subsection{Black-box testing} \label{sec:blackboxtesting}
Black-box testing is performed without any knowledge of why the software behaves as it does \cite{SoftwareTesting}.
This means that the person (the tester) that tests the software cannot look at the code.
The tester is only able to give input to the software and receive output based on the input.

% Static black-box testing (side 56)
	% Product specification (side 57)
		% Forestille sig at være brugeren (side 57)
		% Undersøg lignende software (side 58-59)

% Dynamic black-box testing (side 64)

% Other black-box techniques (side 87)

% Eksempel paa black-box testing

% Hvordan har vi brugt det i vores projekt
	% Usability tests

\subsection{White-box testing} \label{sec:whiteboxtesting}
