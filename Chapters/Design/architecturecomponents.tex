\subsection{Component design theory} \label{sec:archicomponents}
The component design \cite{Rod-Aalborg} is the structural view of the system concerns, where the each of the components are being viewed in relation to their specifications and responsibilities within the system.
A component is defined, \citep[p.~192]{Rod-Aalborg}, as.
\begin{defn}\label{defn:component}
	A collection of program parts that constitutes a whole and has well-defined responsibilities. 
\end{defn}

These components when combined will make up the component architecture which defines the system structure, \citep[p.~192]{Rod-Aalborg}, see the following definition. 
\begin{defn}\label{defn:Structure}
	A system structure is composed of interconnected components. 
\end{defn}

\subsubsection*{The Layered Architecture Pattern}

The layered architecture design means that the components are designed in layers such that the components are above or below each other.
Here each layer or component describes its responsibilities and which other components it can access.
As seen in {\color{red}figure xx} the layers are shown as components where arrows denotes which layers are dependent on each other.
This also means that when one component is being changed, then the other components down the layers will likewise change.
The component layers does not exclusively have to be vertical as they can also be designed with horizontal decomposition which denotes sub components.

The layered architecture can be distinguished between open and closed, strict and relaxed architecture which provides four combinations:

\begin{center}
	\begin{tabular}{| c | c |}
		\hline
		Closed-Strict & Open-Strict \\
		\hline
		Closed-relaxed & Open-Relaxed \\
		\hline
	\end{tabular}
\end{center}

A \textit{closed} architecture means that the layer can only access components which are immediately adjacent to it thus it can only access those that are directly above or below it.
An \textit{open} architecture means that the component can access any other component no matter how far above or below it is.
A \textit{strict} archicture is only able to access components that are either above or below, but not both.
Whereas, the \textit{relaxed} architecture is able to access components from both directions.

\subsubsection*{Interface, function, and model components}

When designing the component architecture there can usually be defined \textit{interface, function}, and \textit{model} components.
Each of these have their own role and responsibilty within the system.

''A model component's main responsibility is to hold the objects that represent the problem domain. '' \citep[p.~203]{Rod-Aalborg}.
In other words the model component should seek to solve the problems of the problem domain. 
Whenever something changes in the problem domain, the model component should be changed accordingly to solve the new problem that has arisen.

''The main responsibilty of a function component is to provide the model's functionality'' \citep[p.~205]{Rod-Aalborg}.
This component ensures that the solutions and algorithms from the model component is accessible for the user interacting with the system.
''The main responsibilty of an interface component is to handle the interaction between the actors and the functionality'' \citep[p.~207]{Rod-Aalborg}.
This component is what the users directly interacts with to access the underlying function and model.
