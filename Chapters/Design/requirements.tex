\section{MoSCoW rules}

It is important to prioritize a systems' requirements after the first draft of a system has been designed.
This is because that there are no guarantees that all of the requirements and design ideas are doable within the time frame of the project.
There is in this project set a hard time limit by Aalborg Univerity.

These priorities are set to determine which requirements are absolutely necessary for the system to function, the \textit{minimal viable product} (MVP), and which requirements are simply nice to include.
The \textit{MoSCoW rules} have been chosen to help priorities the requirements for this project. 

The MoSCoW rules classifies these priorities into: \cite{Benyon}

\begin{itemize}
        \item \textit{Must have}.
        \item \textit{Should have}.
        \item \textit{Could have}.
        \item \textit{Want to have but Won’t have this time around}.
\end{itemize}

Must have are the fundamental requirements to include of which the system would not function without.
Must have also determines the MVP.
Should have are essential requirements to include if the time frame allowed, but the system could still function without.
Could have are less important requirements than must have and should have and they would easily be excluded from the system.
Want to have but Won't have are requirements that can wait until a later point of the development of the system. \cite{Benyon}

% Henrik: Er dette en okay måde at lave denne definination? DEB-bogen s. 147 øverst
A definition of a requirement is provided to prevent any misunderstandings throughout this chapter. \cite{Benyon}

\begin{defn}
Something the product must do or a quality that the product must have
\end{defn}

% Henrik: Noget om hvordan man finder frem til disse "requirements"
\subsection{Defining requirements}
% vigtigt at finde ud af hvad "the client(s)" ønske, før man sætter "requirements"
Before defining a projects' requirements it can be a good idea to figure out what the client wants because it is the client, and not the designer, that is going to use the final system.
A lot of different techniques can be used when trying to define a projects' requirements, e.g.:

\begin{itemize}
        \item Interviews
        \item Observing people and record their activities on video
        \item Organizing focus groups
\end{itemize}

These techniques can help the designer to better understand the problems that the client has with the current system, and at the same time, a better understanding of the requirements for the new system.

% måske meget lidt om prototyper og interviews
% det kan være en iterativ process, og det sker ofte at der kommer nye "requirements" fra tid til anden

% Henrik: To typer af requirements skal nok skrives om
% Henrik: Måske lav dem til definitioner sammen med "requirements" længere oppe?
There are two types of requirements:

\begin{itemize}
        \item \textit{functional}
        \item \textit{non-function}
\end{itemize}

functional requirements are requirements that the system must do and non-functional requirements are the qualities that the system must have.

% Henrik: Noget om hvordan vi er kommet frem til vores "requirements"

% Henrik: Afsluttende tekst
