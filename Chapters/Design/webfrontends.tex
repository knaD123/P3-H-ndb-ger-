\documentclass[../../master.tex]{subfiles}
\begin{document}
\subsection{Web Frontends} \label{sec:webfrontends}
The languages of UI on the web are HTML, CSS, and JavaScript, and each of these three languages have their own respective roles.\cite{nixonweb}
\subsubsection*{HTML}
HTML is the most important of the three.
It's a markup language that is used to describe the structure and much of the contents of a webpage.
\begin{lstlisting}[language=HTML,caption={An example HTML page},label=lst:HTML]
<!DOCTYPE html>
<html>
<head>
<title>HTML Example</title>
</head>
<body>
<h1>HTML</h1>
<p>This is a <b>webpage<b></p>
</body>
</html>
\end{lstlisting}
The different tags can contain child tags, which in turn can have their own tags.
This gives the webpage a tree structure.

Two important types of tags are the \texttt{<script>} and \texttt{<style>} tags.
These allow for embedding CSS and JavaScript into the page.

\subsubsection*{CSS}
CSS allows for defining the style of the document.
\begin{lstlisting}[caption={A simple CSS style with blue text on a green background},label=lst:CSS]
body {
	background-color: green;
}

p {
	text-color: blue;
}
\end{lstlisting}
As an example the stylesheet in \cref{lst:CSS} colours the background green, and the text in paragraph (\texttt{<p>}) tags blue.
The possibilities with CSS are of course far greater than just colours, as the language also allows for positioning elements, changing their visibility, and far more features than can be covered in this report.\cite{nixonweb}
In working with CSS, the primary focus will be working from an existing CSS Framework called Bootstrap.
Having Bootstrap, which predefines a lot of web components, greatly eases the development of web applications.
While this naturally comes with a trade-off for customization, it was a requirement of the system that it was to be simple, and it was therefore concluded that this project did not need any cutting edge UI to achieve its goals.

\subsubsection*{JavaScript}
JavaScript is a scripting language built for the web.
JavaScript allows for running code on the page, which can interact with the contents of the page, play sounds, send requests in the background, and much more.\cite{nixonweb}
\lstinputlisting[firstline=48, lastline=56, caption={A JavaScript function in the project that shows document previews to the side}]{OBHandbooks/OBHandbooks/wwwroot/js/site.js}
The JavaScript function above performs an Asyncronous JavaScript and XML (AJAX) request, where it fetches the preview page for a specific document.
It then, on success, sets the \texttt{div} with the id \texttt{previewColumn}'s content to be the received data.
\end{document}
