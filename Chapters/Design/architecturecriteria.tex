\documentclass[../../master.tex]{subfiles}
\begin{document}
\subsection{Criteria}\label{sec:architecturecriteria}
A good software system is one that meets its requirements and has no major weaknesses \citep[p.~179]{Rod-Aalborg}.
This system's requirements are listed and prioritized in \cref{sec:requirements} and the efforts put into avoiding weaknesses are laid out in this section.

Due to the conditions this system is developed under, see \cref{factor}, flaws are to be expected.
To minimize the amount of critical flaws the \textit{criteria} \citep[p.~180]{Rod-Aalborg} upon which the system will be judged have been prioritized.
The idea is that development will be focused on the 'very important' and 'Important' criterias to make sure that all of these work.
Less effort will be put into the 'Less important', 'Irrelevant' or 'Easily fulfilled' criterias, making flaws in these areas more likely.
\todo[inline]{Astrid: Lav en ny sætning til easily fulfilled}
However, as established, they are less important and therefore the flaws will, most likely, not be critical.
These priorities can be found in \cref{fig:criteria} and the reasoning behind them follows below.

\begin{table}[H]
	\begin{center}
		\begin{tabular}{|l|c|c|c|c|c|}
			\hline
			Criteria $\backslash$ priority & \rotatebox{90}{Very important} &  \rotatebox{90}{Important} & \rotatebox{90}{Less important} & \rotatebox{90}{Irrelevant} & \rotatebox{90}{Easily fulfilled}\\
			\hline
			Usable & \xmark & & & & \\
			\hline
			Secure & & & & \xmark & \\
			\hline
			Efficient & & & & & \xmark \\
			\hline
			Correct & & \xmark & & & \\
			\hline
			Reliable & \xmark & & & & \\
			\hline
			Maintainable & & & \xmark & & \\
			\hline
			Testable & & \xmark & & & \\
			\hline
			Flexible & & & & \xmark & \\
			\hline
			Comprehensible & & \xmark & & & \\
			\hline
			Reusable & & & \xmark & & \\
			\hline
			Portable & & & & \xmark & \\
			\hline
			Interoperable & & & & \xmark & \\
			\hline
		\end{tabular}
	\end{center}
	\caption{Prioritized list}\label{fig:criteria}
\end{table}

It is very important that the system is \textbf{usable}.
As has been described in the previos chapters, \cref{sec:PACT-people}, the users have limited IT experience, and the users in the reader role will get limited exposure to the system, only using it to read new editions of the handbook and marking it as read.

While the administrator or writer role will use the system more often, they too have limited IT experience, so the complexity should still be kept to a minimum in their case.

Ipsen has expressed that whether or not the system is \textbf{secure} is no major concern as the information is not confidential.
In the use case envisioned with the user, the program is only accessible inside a secured network, and it has therefore been determined that security is outside this systems area of responsibility and then irrelevant.

\todo[inline]{Astrid: Sammenlign med det system der er i brug lige nu i stedet for existing solutions afsnittet}
Comparing the existing solutions, examined in \cref{chap:existing}, to the case described in this report of making an \textbf{efficient} system is easily fulfilled as the existing solution is incredibly inefficient, due to all the work they had to do manually.

The \textbf{correct} criteria from \cref{fig:criteria} covers whether or not the system fulfills the system definition, that was specified in \cref{sec:systemdefinition}, and meets all the requirements, see \cref{sec:requirements}.
The reasoning for labeling 'correct' as important instead of very important is because of the long list of requirements where not all elements are equally important.

It is very important that the system is \textbf{reliable} as it is a problem if the system malfunctions and e.g. deletes files, assigns wrong version numbers or automatically approves a new version.
It could, should any of these happen, mean the company will lose their certification which, as described in \cref{sec:standards}, is at best a problem for consumers' trust to the company and at worst a requirement by law.

Whether the system is \textbf{maintainable} is less important as it is assumed that the user is not capable of maintaining it anyway.

The system needs to be \textbf{testable} in order to verify that it is reliable.
Therefore this is important.

According to the \textit{OOA\&D} method \citep[p.~182]{Rod-Aalborg}, it is very important that a system is \textbf{flexible}.
On the ohter hand, as the project will only be developed for half a year, this does not take priority and has been classifield as irrelevant.
The system is, however already developed with some future elements in mind: The changelog, which Ipsen has requested, is not a requirement under the current conditions, however it is a change which she assumes will happen within the next ten years or so, as stated in \cref{sec:CaseDescription}.

It is important that the code is \textbf{comprehensible} as we are seven diffferent people with varying areas of expertise and skill working on the same code simultaneously and we all need to understand it.

As stated before it seems unlikely that this project will be developed further after its completion and therefore the \textbf{reusability} is less of a concern.
However, the system is structured so that components can be reused to implement some of the requirements classified as 'Want to have but Won't have this time around', see \cref{sec:requirements} for the list.
This is most obvious with the abstract role class, which has remained even though it has no purpose in the current layout of the system.
As a result this criteria has been defined as less important.

As tablets and mobile phones are not allowed in the production, and it is assumed that the majority of employees use a standard internet browser, it is irrelevant whether or not the system is \textbf{portable}.

It is also irrelevant whether or not the system is \textbf{interoperable} as the only systems it needs to interoperate with are sms and e-mail, which it has been designed for from the start as an integral feature.

To summarize the top prioritized criteria are the systems usability and reliability.
Less important but still notable are correctness, testability and comprenhensibility.
These criteria will therefore be in focus when designing the components the system is made up of.
\end{document}
