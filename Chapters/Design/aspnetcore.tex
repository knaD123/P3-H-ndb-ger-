\documentclass[../../master.tex]{subfiles}
\begin{document}
\subsection{ASP.NET Core} \label{sec:aspnetcore}

ASP.NET core is a web framework developed by Microsoft for developing web applications in the .NET Core Framework \cite{aspnetcore2}.
ASP stands for \textit{Active Server Pages} and offers a variety of application models, one of which being the Model-View-Controller (MVC) pattern.
The framework has built-in support for many common features, such as database integration, identity and authorization with multi-factor, external authentication and CSS frameworks.

The reason for this web API to be used is because of it is part of the scope of the development project to develop an application using the programming language C\#.
ASP.NET Core supports, among others, C\# and is able to integrate the language with a web application.

\subsubsection{Model-View-Controller design pattern}\label{sec:mvc}
The MVC pattern is an architectural pattern that splits an application into three basic components; the model, the view, and the controller.
Originally meant for Graphical User Interface (GUI) development, the pattern has been adapted into web development, and is used by many of the frameworks in the space. \cite{gangoffour}

The advantages of using the MVC architecture are that it helps reducing the complexity of the application by dividing its components into three individual components each with its own function and responsibilites.
These components are as mentioned before \textit{model, view}, and \textit{controller}, see \cref{fig:MVC-components}.
Furthermore the MVC archictecture supports test driven development and also does not use server-based forms which gives the developers full control over the application. \cite{mvcarticle}

\begin{figure}[H]
\centering
	\begin{tikzpicture}[node distance=2cm]
	\node[process] (model) {Model};
	\node[process] (view) [below of=model, left of=model] {View};
	\node[process] (controller) [below of=model, right of=model] {Controller};
	\draw[arrow] (model.south west) -- node[left] {Updates} (view.north);
	\draw[arrow] (controller.north) -- node[right] {Updates} (model.south east);
	\node[process] (user) [below of=model, node distance=3.5cm] {User};
	\draw[arrow] (view.south) -- node[left] {Perceives} (user.north west);
	\draw[arrow] (user.north east) -- node[right=0.2cm] {Interacts with} (controller.south);
	\end{tikzpicture}
	\caption{The Model-View-Controller design pattern.}\label{fig:MVC-components}
\end{figure}

The model component's responsibility is to implement the logic of the data domains \cite{mvcarticle}.
In other words it is here the main algorithms reside that which provide the solutions for the main function of the application.
The responsibility of the view component is to provide an interface for the user and makes it possible for the user to interact with the application.
The controller component's responsibility is to respond to the user's request and gives a means to make the applications underlying algorithms and logics accessible for the user. \cite{mvcarticle}

The MVC architecture is thus a three-layered structure where each component is loosely dependent on each other.
This gives the developers a possibility of developing each component simultaneously.
This separation between the components also makes it easier to test the application in the test-driven development approach. \cite{mvcarticle}

The data managed by the application is encapsulated into the model classes.
These can then be accessed or subscribed to by the views, which the user sees.
The user then subsequently interacts with the controller, which in turn updates the model.
The seperation of concerns allows for a quite flexible architecture, where you can change the view without touching the model or controller, or the controller without changing the view and model.\cite{gangoffour}

\subsection{Entity Framework}\label{sec:efcore}

Entity Framework (EF) is a framework for mapping object-oriented data structures into a relational database, also referred to as an object-relational mapping framework, for .NET.{\color{red}\cite{efcore}}
The job of the EF is translating the objects in memory into Structured Query Language (SQL) statements so that these objects can be stored in a database.
\begin{figure}[H]
	\centering
	\begin{tikzpicture}[node distance=1cm, minimum height=0.6cm]
		\node[process] (net) {ASP.NET Core};
		\node[process] (efcore) [below=0.5cm of net] {Entity Framework Core};
		\node[process] (sql) [below=0.5cm of efcore] {Relational Database};
		\draw[arrow] (net) -- (efcore);
		\draw[arrow] (efcore) -- (net);
		\draw[arrow] (sql) -- (efcore);
		\draw[arrow] (efcore) -- (sql);
	\end{tikzpicture}
	\caption{The relationship between ASP.NET Core and the Relational Database, with the EF acting as a glue between the two layers}\label{fig:ASP-Entity}
\end{figure}

To use the EF, everything needed to do in the code is to provide a context object.
which mediates the connection, giving the configuration arguments a connection to a database requires.
This context defines the sets of objects to be stored in the database, and also the database to be used.
The EF supports multiple kinds of databases, from MariaDB to PostgreSQL.
The support for multiple database systems is provided by the respective database providers.
\end{document}
