\documentclass[../../master.tex]{subfiles}
\begin{document}
\section{Desktop Client}
A desktop client for the application could be developed.
This would allow for many interesting features, such as desktop notifications, or regular distributed backups on workstations.
\subsection{WebDAV workflow}
An idea which was considered, but never implemented, was a feature which allowed accessing the handbook via webDAV.

WebDAV is an extension of the HTTP protocol which allows for more operations in regards to editing content.
These operations are not commonly supported by web browsers, but have specific applications instead.

Using a standard webDAV client, the handbook should be mountable by writers and administrators, by a link specific to each of them.
When they then use, for example, Microsoft Word to make changes to the document, the changes are saved on the server immediately.
Then, when the user feels that they have made the changes they want to make, the next time they visit the website, they will be prompted to turn the changes they have made into a new approval request, or possibly just share it with another user, allowing for some level of collaboration on versions.
Or possibly, if a custom desktop client is developed anyway, the webDAV functionality could be included in it.
\end{document}
