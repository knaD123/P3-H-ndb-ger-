\section{Monitoring}
\epigraph{A commitment to monitoring is the distinguishing characteristic of a professional system administrator.}{\textit{Unix and Linux System Administration Handbook\cite{sysadmin}}}

Monitoring is the practice of gathering metrics, events, and trends from the application, in order to get an understanding of how the application is performing.
In server applications, this is crucial, as it allows system administrators to notice and prevent the problem before the screaming starts.
Most monitoring solutions allow for overviews of how the system is currently performing, but also sending of notifications under certain circumstances, for example if a server is running out of storage, or it is often under stress, or crashes occur.\cite{sysadmin}

To integrate our application with monitoring, it needs to expose metrics.
One of the standard ways to do it is hosting a text document at \texttt{/metrics}, consisting of key-value stores in the form \texttt{metric: value}.
A metrics system will then poll the endpoint at a set interval, and use the fetched values to provide statistics about the monitored system.

Adding a metrics endpoint to the application would allow for getting an idea of how the system is doing, and avoiding downtimes before they happen. If the application was to be used in a real context, adding monitoring would be essential.
%Taniya: vil gerne have det forklaret når vi gennemgår det