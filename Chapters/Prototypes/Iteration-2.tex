\section{Second iteration}
\subsection{Simple timeline}\label{sec:2Iteration-timeline}
After the second interview with Ipsen the second iteration started with the evaluation activity from \cref{fig:DEBModel}. 
During this activity the data gathered at the interview was reevaluated and used to update the relevant models and analysis' such as the FACTOR model and the class diagram.
During the interview the first signs of the iterative model being the relevant model to work with became clear.
Since Ipsen had changed her decision upon how the archive should work.
 During the small interview in iteration 1, it was said that the date search function in the archive would be obsolete.
But when it was a part of the prototype and therefore clearer illustrated it turned into a requirement.

After the evaluation activity an understanding phase started where the first analysis of the functions in the application domain activity was done as well as the development of rich picture analysis in the problem domain activity, \cref{fig:RP-Oversigt,fig:RoleIllustration}.
Hereafter, the architectural design was reevaluated from the first rough ideas in iteration 1, and the development of component design started.
leading the workflow into a mix of the understanding and design activities. 
See \cref{fig:DEBModel,fig:SUModels}.

During this iteration a design process concerning the interface design began, using a mix of envisionment and design activities from \cref{fig:DEBModel}.
The prototype presented at the second interview was during this period further developed and implemented with some of the required functionalities.

The second iteration ended with at usability test with Ipsen based upon the iterated prototype. 
At the end of the meeting with Ipsen the designs of the interfaces was presented and shortly discussed to clarify whether or not the differences between reader, writer and administrator was correctly understood.

\subsection{Elements to take notice of}
\subsubsection*{Supplier sub-problem} 
The general idea of the supplier sub-problem was presented by Ipsen in the initial interview in the first iteration as an impulsive thought.
In the course of the first iteration there was an awareness of the sub-problem but it quickly became clear through the discussions in the analysis that more information was needed to completely analyze it and its role in the problem domain.

At the second interview the supplier sub-problem was discussed through a series of questions prepared beforehand to resolve the needed information.
Afterwards in the second iteration, the problem domain analysis was as mentioned in \cref{sec:2Iteration-timeline}. 
This was done especially with the focus on in-cooperating the supplier subproblem.
This meant changes to the FACTOR analysis and the class diagram.
As for the class diagram the introduction of the supplier and later the refinement of the approval classes led to the final draft seen in \cref{fig:ClassDiagram}.

\subsubsection*{FACTOR}
In addition to the supplier subproblem mentioned above the FACTOR analysis went through two significant changes during the second iteration.

The first change is the role hierarchy which was turned from four levels into four. 
During the first iteration it had become unclear whether or not a fourth level was needed in the role hierarchy.
In the original FACTOR it was written as follows
\footnote{Note: The writing style was more detailed in the first FACTOR to make sure the requirements was understood correctly, when the second interview came around.}:
\newline
''Different levels of permissions/access rights to the documents within the handbook
\begin{itemize}
	\item 
	0 level:
	Reading the handbook except secured documents
	\item 
	1st level:
	Reading the whole handbook
	\item 
	2nd level:
	Editing selected documents
	\item 
	3rd level:
	Total access to editing documents and access to archive''
\end{itemize}
Though it was concluded from the second interview that only three roles were needed.
These are mentioned in the FACTOR analysis found in \cref{factor}.

The second significant difference is how most of the functionality and some of the conditions from the FACTOR analysis has turned into requirements in the MoSCoW presented in \cref{sec:requirementsdefinition}. 
These are among other ''The handbook should be printable'' and ''Title and chapter number are linked. If a title is removes the chapter number cannot be used for anything else.'', from the FACTOR analysis draft in iteration 1.

\subsubsection*{Event table}

\subsubsection*{Prototyping}

%Tænker de større konklusioner fra prototype testen her bruges som en del af introen til anden iteration.


