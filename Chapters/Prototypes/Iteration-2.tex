\section{Second iteration}
\subsection{Simple timeline}
After the second interview with Ipsen the second iteration started with the evaluation activity from \cref{fig:DEBModel}. 
During this activity the data gathered at the interview was reevaluated and used to update the relevant models and analysis' such as the FACTOR model and the class diagram.
During the interview the first signs of the iterative model being the relevant model to work with became clear.
Since Ipsen had changed her decision upon how the archive should work.
 During the small interview in iteration 1, it was said that the date search function in the archive would be obsolete.
But when it was a part of the prototype and therefore clearer illustrated it turned into a requirement.

After the evaluation activity an understanding phase started where the first analysis of the functions in the application domain activity was done as well as the development of rich picture analysis in the problem domain activity, \cref{fig:RP-Oversigt,fig:RoleIllustration}.
Hereafter, the architectural design was reevaluated from the first rough ideas in iteration 1, and the development of component design started.
leading the workflow into a mix of the understanding and design activities. 
See \cref{fig:DEBModel,fig:SUModels}.

During this iteration a design process concerning the interface design began, using a mix of envisionment and design activities from \cref{fig:DEBModel}.
The prototype presented at the second interview was during this period further developed and implemented with some of the required functionalities.

The second iteration ended with at usability test with Ipsen based upon the iterated prototype. 
At the end of the meeting with Ipsen the designs of the interfaces was presented and shortly discussed to clarify whether or not the differences between reader, writer and administrator was correctly understood.

\subsection{Elements to take notice of}
\subsubsection*{Supplier} 
The biggest difference from the first iteration and to the final analysis presented in this report is the supplier subsystem.
The general idea of this was first presented by Ipsen in the initial interview as an impulsive thought. 
The group was therefor aware of the sub-problem but it quickly became clear through the discussions in the analysis that more information was needed to completely analyze this sub-problem and its part in the problem domain.
This confusion has consequently affected the rest of the problem domain analysis since it could not play any major role in the first iteration.


\subsubsection*{Supplier sub-problem}
The general idea of the supplier sub-problem was first presented by Ipsen in the initial interview as an impulsive thought. 
The group was therefor aware of the sub-problem but it quickly became clear through the discussions in the analysis that more information was needed to completely analyze this sub-problem and its part in the problem domain.
This confusion has consequently affected the rest of the problem domain analysis in this iteration since it could not play any major role  before things has been cleared up.

\subsubsection*{FACTOR}
There are two big differences between the final FACTOR analysis and the preliminary one.
First is that during the first iteration the role hierarchy was turned from three levels into four. 
In the original FACTOR it was written as follows
\footnote{Note: The writing style was more detailed in the first FACTOR to make sure the requirements was understood correctly, when the second interview came around.}:
\newline
%Anna: HVordan tydeliggøre mman at vi citere noget tidligere skrevet (skrives det i italics eller skal placeringen være anderledes eller er der andre trics?)
''Different levels of permissions/access rights to the documents within the handbook
\begin{itemize}
	\item 
	0 level:
	Reading the handbook except secured documents
	\item 
	1st level:
	Reading the whole handbook
	\item 
	2nd level:
	Editing selected documents
	\item 
	3rd level:
	Total access to editing documents and access to archive''
\end{itemize}

The second difference is how most of the functionality and some of the conditions from the FACTOR analysis has turned into requirements in the MoSCoW presented in \cref{sec:requirementsdefinition}. These are elements such as,
\begin{itemize}
	\item From conditions
	\begin{itemize}
		\item 
		''The system needs to handle several different file types \ldots''
		\item 
		''The handbook should be printable''
	\end{itemize}
	\item From functionality
	\begin{itemize}
		\item 
		''Title and chapter number are linked. If a title is removes the chapter number cannot be used for anything else.''
		\item 
		''There needs to be a Table Of Contents with title, chapter number and date/version number''
	\end{itemize}
\end{itemize}

\subsubsection*{Event table}

\subsubsection*{Prototyping}

%Tænker de større konklusioner fra prototype testen her bruges som en del af introen til anden iteration.


