\section{Third iteration}
\subsection{Simple timeline}\label{sec:3Iteration-timeline}
The third iteration started with the evaluation activity from \cref{fig:DEBModel} based upon the fifth meeting with Ipsen, where the interactive prototype, with added functionality, and the mock-up design were presented.
At the end of the evaluation activity the work proccess flowed into a combination of the design and envisionment activities as well as the understanding activity.
The combination of design and activity was mainly used when further developing the mock-up.
As for the understanding activity, this was used mainly in combination with the evaluation activity when revisiting the analysis and design of the system.

During the iteration the class diagram belonging to the broblem domain in \cref{fig:SUModels} was evolved to its final evolution presented in this report, see \cref{fig:ClassDiagramNoSupplier}.
This diagram indicating the actual sub-problem being in focus of this report.
The workflow also went through the apllication domain activity seen in.
During this process, the function list, actors and use cases were revisited
Furthermore the third iteration also included a visit to the architectual design and component design activities.
Until this point these activities had been shortly discussed and sketched.
Though in this iteration the elements in these activities were more thoroughly designed and worked upon.
Leading to the results seen in \cref{sec:architecturecriteria,sec:AppMethod,sec:componentdesign}.
Parallel with this workflow the design process on the Mock-up continued.

Iteration three ended with a usability test done with Ipsen and her immediate family as test subjects.
Followed by a short discussion with Ipsen about the system in general.
This iteration was meant to end with another usability test done two days after the one with Ipsen and her family though because of some complications this was not held.

\subsection{Elements to take notice of}
\subsubsection*{Actors and use-cases}
The larger element to take notice of which will be mentioned is that of the actors and use-cases from the application domain.
In this iteration the actors was turned from:
\begin{itemize}
	\item
	Consultant
	\item
	Secretary
	\item
	Manager
	\item
	Worker
	\item
	Software
\end{itemize}

Into those found in \cref{tab:ActorTable} which are 'administrator', 'Writer' and 'Reader'. Instead the relevant specific cases that existed to begin with was turned into examples in the different actor tables in \cref{sec:Actors}.

As for the use cases they also evolved quite a bit in this third iteration.
The following paragraphs will first present the original use-case names followed by a description of the changes done in this iteration.

\textbf{Manage documents} was turned into three more specific cases such that there would be no question about what was included in the use-case.
These new cases which are those used in this report are \textit{Access current handbook}, \textit{Add new file to the system} and \textit{Access archive}.

\textbf{Manage users} was in the same way turned into more specific cases.
These being \textit{Add user} and \textit{User management}.
Furthermore, the use-case \textit{Edit user information} was added to the table.
The reason for this being that the other two cases are features meant for the administrator role, while this last one is meant for one to change his own information.

\textbf{Read status} is a feature meant to help keep track of who has read the document among those that need to.
The use-case was divided into two since it includes two angles.
The first angle being the one for those that need to be kept track of and the last, the one for those who are keeping track.
These more specified use-cases is respecitvly called \textit{View who has read a document} and \textit{Mark document as read}.

Other than these aforementioned three use-cases two were renamed but otherwise kept the same and three other cases was removed.
Those that were removed were \textit{Update TOC}, \textit{Track differences} and \textit{Manage suppliers}.
The first one about the TOC was removed since this was not a use-case relevant for the application domain after 'software' was removed from among the actors.
The use-case about tracking differences was removed for two reasons.
One being the same as the one for the TOC use-case.
The other reasons was that this case was evaluated to a nice to have requirement that in no way was necessary for the work.
Furthermore, before it could be useful and not obstructing the users workflow it would need a rather huge analysis and design process in itself.
Lasly, the supplier use-case was removed since the sub-problem concerning this use-case was scoped out in the Problem-domain analysis.

\subsubsection*{Some conclusions on last prototypes}
At the end of the second iteration a usability test was held with Ipsen based upon the functional prototype available at that point in time, \cref{secondtest}.
One specific note worth specifically mentioning in this section is how the issue with the negative ID numbers, which is described in more details in \cref{secondtest}, caused for the document class to be reevaluated.
Thereby leading the class into another iteration concerning its design.
The results of the evaluation being that it should be possible to delete a document if a file have not yet been attached and approved.

The last thing that will be mentioned here as well is how Ipsen when interacting with the archive found it weird that the active version of a document would be present as well.
Though shortly after in the discussion it became clear that it was a likable feature that yet again would be helpful during audits, as well as when wanting to retrieve the whole handbook.

At this meeting with Ipsen, the Mock-up which had been designed until then was also presented.
This was done  to grant a visual of how the interface for writer and reader was understood and if this understanding was correct.
The conclusion of this came quickly and turned out positive.
For further detail on this test see \cref{thirdtest}.


\subsubsection*{Further prototype development}
During the third iteration the mock-up was further developed, now mainly focusing upon the administrators interfaces.
The main designs among these can be found in \cref{sec:2Mock}.

Lastly,  the prototype from the end of the second iteration was further developed and implemented as well.
This new prototype was then used in the usability tests with Ipsen and her family, \cref{fourthtest}.
Some interfaces connected to the prototype at that point can be found in \cref{sec:3prototype}\footnote{Please note that the pictures used here does not include the actual documents used at the usability test due to confidentiality issues.}.
After the test the prototype and the results was evaluated and some of the bugs fixed.
Which is the status for the program where this reports ends.
