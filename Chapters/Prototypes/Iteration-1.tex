\section{First iteration}\label{sec:Iteration1}
\subsection{Simple timeline}
The first iteration started with the group getting a short presentation to the software development project through a mediator on Ipsen's behalf.
Right after this an initial interview to talk about the requirements was planned.
During the time up to the interview the workflow was done in the understanding phase from \cref{fig:DEBModel}.
In this phase the group looked into existing solutions and idea generating, such that there was an opportunity to discuss unbiased ideas, and through this also get Ipsen to reflect upon her requirements.
The interview was the start of the first evaluation phase in the first iteration.
The ideas presented at this interview include among other an build-in editor such that everything about the handbooks would be managed in the system.
Though this was shut down fast since for that to be useful it would need to always keep up to date on how Microsoft's UI for eg. Word would look, since it otherwise would take too long to get used too every time it was necessary to change anything in the handbook.
The notification system which is one of the requirements stated in this report, was also one of these beforehand generated ideas presented at the first interview. 
Another idea was to make a Web Interface which would either be locally on the clients server or on the world wide web. This was first shot down, though through discussion it became clear that this was not impossible, since there was better opportunities with the notification system which was turned into a requirement.

After the first interview the analysis work started, leading the workflow back to the understanding phase yet again.
The analysis done during this phase was mainly done based on two basic elements from \cref{fig:SUModels}.
The problem domain analysis, which in this iteration focused on; Requirements, FACTOR, initial class diagram and the event table.
At this point in time the group also slowly started looking into elements based in the application domain analysis, which were; PACT and the actor table.
During this period another small interview was held with Ipsen to evaluate the understanding of the vocabulary concerning the problem domain, to make sure the terminology was correct.
After all this analysis and right before the second interview which closed the first iteration and started the next one,a couple of prototypes were drawn up on paper and then afterwards turned into an interactive prototype, by programming, which could be presented at the second interview.
This combined both the envisionment and design activities from \cref{fig:DEBModel}.

The first iteration ended with an evaluation, the second interview, where the requirements in form of the FACTOR analysis was updated and agreed upon.
After this a couple of questions was asked clearing up confusion and misunderstandings surrounding the approval system, the role hierarchy and the supplier subsystem.
Lastly, the first final prototype of this iteration was presented.

\subsection{Elements to take notice of}
\subsubsection*{Supplier sub-problem}
The general idea of the supplier sub-problem was first presented by Ipsen in the initial interview as an impulsive thought. 
The group was therefor aware of the sub-problem but it quickly became clear through the discussions in the analysis that more information was needed to completely analyze this sub-problem and its part in the problem domain.
This confusion has consequently affected the rest of the problem domain analysis in this iteration since it could not play any major role  before things has been cleared up.

\subsubsection*{Class diagram}
During the first iteration the class diagram itself went through multiple developments, even without the supplier system being present.
In the first development, see \cref{fig:FirstClassDiagram}, the approval class was not seen as part of the main classes in the problem domain and therefore not present in the diagram.
Another difference in the first development was how the different roles was only thought of as attributes instead of classes in themselves and therefore were not present either.

\begin{figure}[H]
	\centering
	\begin{tikzpicture}[align=center, scale=1.0, transform shape]
	%De enkelte noder i diagrammet
		\node(hand)[process]{Handbook};
		\node(doc)[process, below=1cm of hand]{Document};
		\node(dep)[process, right=1.5cm of doc]{Notification};
		\node(ver)[process, below=1cm of doc]{Version};
		\node(read)[process, right=3cm of ver]{Read Status};
		\node(user)[process, right=2.1cm of hand]{User};
	%Linjer i billedet:
		\draw[{open diamond}-](hand.south)node[right, yshift=-0.2cm]{1}--(doc.north)node[left, yshift=0.2cm]{1..*};
		\draw[{open diamond}-](doc.south)node[right, yshift=-0.2cm]{1}--(ver.north)node[left, yshift=0.2cm]{0..*};
		\draw[black](doc.east)node[above, xshift=0.2cm]{1}--(dep.west)node[below, xshift=-0.35cm]{0..*};
		\draw[black](ver.east)node[above, xshift=0.2cm]{1}--(read.west)node[below, xshift=-0.35cm]{0..*};
		\draw[black](hand.east)node[above, xshift=0.2cm]{1}--(user.west)node[below, xshift=-0.35cm]{1..*};
		\draw[black](user.south)node[right, yshift=-0.2cm]{1}--(dep.north)node[left, yshift=0.2cm]{0..*};
		\draw[black](user.east)node[above, xshift=0.2cm]{1}-|(read.north)node[left, yshift=0.2cm]{0..*};
	\end{tikzpicture}
	\caption{First class diagram of developing process}\label{fig:FirstClassDiagram}
\end{figure}

Throughout the first iteration approval turned into a class in its own rights and therefore was cooperated into the diagram.
The role pattern came forth during this iteration as well though the final form of it went through a couple of evolutions.
the main discussion about this element was whether or not the reader class was necessary since all users by default was readers. 
Though it was decided for clarity to keep it.
%Anna: er ikke sikker på det her er nødvendigt og ikke bare bliver for meget.
%Another debate was revolved around the writer and why they were necessary. 
%Though it soon became clear through the second interview with Ipsen that this was not to be changed since there would be more Writers than administrators and it was good the writers did not have total access to the system the same way the administrator has.

\subsubsection*{Prototyping}
The development of the prototype used in the second interview with Ipsen started out as drafted versions which can be found in \cref{sec:First-sketches,sec:Second-sketches}.
Based on these sketches an interactive prototype was made. 
Screenshots of the different interfaces in this prototype can be found in \cref{sec:1prototype}.
This prototype did not have any of the system required functionality implemented.
It was used to let Ipsen click around and thereupon create a debate of whether or not the envision of the final system was correct or if something needed to be adjusted.

%Tænker de større konklusioner fra prototype testen her bruges som en del af introen til anden iteration.




%\subsection{Major differences between first iteration and final analysis}
%Anna: Kan ikke helt finde ud af om det her afsnit i virkeligheden ikke burde komme i iteration 2 og 3 hvor der diskuteres forskellen fra den tidligere og til denn pågældende iteration. (Dog vil jeg argumentere for at class diagram, supplier og prototyper som minimum skal nævnes her dog måske i mindre grad)

%\subsubsection*{Supplier} 
%The biggest difference from the first iteration and to the final analysis presented in this report is the supplier subsystem.
%The general idea of this was first presented by Ipsen in the initial interview as an impulsive thought. 
%The group was therefor aware of the sub-problem but it quickly became clear through the discussions in the analysis that more information was needed to completely analyze this sub-problem and its part in the problem domain.
%This confusion has consequently affected the rest of the problem domain analysis since it could not play any major role in the first iteration.


%\subsubsection*{Class diagram}
%Anna: Skal nok skrive kommentarer til udviklingen inden for denne, men mangler billeder af de oprindelige klasse diagrammer

%\subsection{Prototyping}
%Anna: Vil meget gerne gennemgå lidt af den udvikling vi havde ift den første prototype udvikling, kræver dog at jeg lige få konverteret nogen ting over til billeder så jeg kan referer dem i appendiks, og nu er det ærligt talt for sent