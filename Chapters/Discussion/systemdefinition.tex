\section{System definition}\label{sec:dissystemdef}

This section will discuss the system definition from \cref{sec:systemdefinition} and how the implemented system lives up to this definition.
An in depth look at the system in its current state will then be compared to our final system definition.

The system is capable of management of documents that is uploaded from a PC.
The files that are placed on the server are viewable in the table of contents where each document is also identifiable with a unique ID.
Each document also has been given a changelog that can be written when pressing to make changes to said document.
There exist version of the documents that has its own version number.
When documents are updated, older versions are archived.
And lastly new versions must be approved if it is submitted to the handbook.

All the fulfilled requirements are also supported by multi-level access rights, these are separated into three roles, \textit{Administrator, Writer} and \textit{Reader}.
These users are also grouped in different departments that is associated with chosen documents.

The handbook system is also complimented with the notification system that sends notifications to users when changes appear, although mail or message notifications are not included in the system.

Although the system has fulfilled most requirements, it is not capable of editing different file types.
This was due to most documents stemming from Microsoft Office software, that appeared difficult to solve through the system.
A feature with such complexity could in itself be a system.
The handbook has no import functionality which was requested by the system definition.
Furthermore, the TOC that presents all documents are not sortable as originally desired.

In spite of the fact that parts of the system definition are not completed, the system is functional and capable of working in production.
The lacking parts and non-essential as to the core functionality of the system, which means that the system definition requirements are therefore fulfilled to a high degree.

A critique of a part of the definition that states that ''Reading access to the handbook for any the levels is possible at any relevant locaction''.
This formulation of the definition is unclear and can be interpreted in various ways.
By the project group and the implementation, this has been interpreted as that the handbook documents should be printable as to make it possible for the users to take these printed documents with them as they go.
Though this definition in and of itself could have been interpreted and discussed in length, this discussion did not happen within the project group.
If such discussion about this specific system definition had happened a more creative or out of the box idea or solution could have occured.
Though, as it is, the definition remains vague and no interesting discussions about it happened during the process.
% Henrik: Gentager de tre ovenstaaende linjer ikke sig selv? :-)
% Anja: Tjah. Tjoh.

Another possible solution could have been to simple rephrase the system to state that the handbook documents should be printable.
