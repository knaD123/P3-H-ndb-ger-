\section{Backups}
No production-ready backup system was implemented in using the system.
During the usability tests, an ad-hoc backup and restore system was used to setup identical environments on multiple machines, which involved dumping the postgresql database in it's entirety, and the entire \texttt{aspnet} container.
This is an approach which has much room for improvement.
For example, the \texttt{/var/obhandbooks} directory, which is used for all permament storage of files, such as handbook document versions, is not mounted as a volume, but is a folder on the container.
This goes against best docker practices, where everything not in a volume should be immediately replacable.
Mending this issue would consists of adding a single line to the \texttt{docker-compose.yml} file.

However, using the ad-hoc container backup and restore approach was quite simple.
While a script for creating the backups was never developed, due to this only having to be done once, restoring from backup was turned into a script.
Due to this script being used in testing, it also deletes any previous restored backups.
\lstinputlisting[language=bash]{OBHandbooks/restore.sh}
What the script does is that it loads from a backup two containers stored in tar files, and turns them into docker images.
Subsequently, it creates a volume named \texttt{psql_vol}, and fills it with the data in the \texttt{psql.tar} file.
This volume can then be mounted in the postgresql container used with the app.
Using this approach, everything was backed up, even user sessions.

Turning the approach into something that is done and tested automatically would, from the experience gathered in the usability testing phase, not be very difficult.
