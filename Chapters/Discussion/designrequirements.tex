\section{Design requirements}\label{sec:disdesignrequirements}
With the MoSCoW rules the functionalities were placed into different classes of priority, these can be seen in \cref{sec:requirements}.
In this section there will be a look into the requirements and their fulfilment in the system.

The design requirements in \cref{sec:requirementsdefinition} were tailored to the needs of the user, these requirements were mostly complete with the exception of a few that will be looked at in this chapter:

\begin{itemize}
	\item \textit{"It should be easy to switch from an existing system, and back to the existing system"}
	As the current system stands there is no easy way to implement an existing system.
Ipsens current system is a collection of documents of different file types, that have to be converted before it is put into the project system.
It might be possible to create a system that can convert file types to pdf, but that was not the focus of this project.
	\item \textit{"The handbook should be printable"}
	Unfortunately, the handbook is not printable.
MANGLER ÅRSAG.
	\item \textit{"Option to sort documents according to different attributes"}
	This functionality that makes the program more user friendly did not enter the system.
The problem could be solved by implementing a sorting algorithm, but unfortunately this functionality was not included in the final product.
\end{itemize}


