\section{User rights}
User rights was implemented in the system so users not managing the handbook do not have the power to change, delete or otherwise administrate the system.

As for the administrator, reader, and writer roles; these were iterated upon and implemented through every step of the development.
A separation between the administrator and the reader and writer roles were thought into the design from the beginning.
It was clear from the start that different permission rights for different employees from the firm that Ipsen was consultant for, were essential.
Because of this the three different roles were created.

In relation to the problem statement, it is possible for the administrator to manage the handbook, but interacting with the interface, as it is currently implemented, is not wholly intuitive.
This has been touched upon in the description of the fourth usability test in \cref{fourthtest} and in the discussion in \cref{sec:dissystemdes}.
