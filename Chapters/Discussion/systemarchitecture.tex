\section{System architecture}	
% Rasmus: Det lyder som om vi forveksler MVC med client-server, og samtidigt web?
When choosing the system architecture, Model-View-Controller stood out as it functions well in web development, but it was potentially possible to use a different system architecture with success.

The primary downside of an architecture that is not made for web development, lies in the distribution of the system and how the system is updated to the latest version.
This means the developer and the user have to use third party websites or personally meet to install said update.	
This also means software problems would take longer time to patch as developers have to meet users or otherwise make contact another way.	
Furthermore the product would be harder to make available to users on other platforms such as phones.	

Although it is not optimal for this case, it is cost free and once implemented, the system should not have to be maintained.	
And although it is not easy to update, there is no need for internet connection.	

On the other side of the spectrum, the Model-View-Controller pattern stood out, because of the ease of use, optimal accessibility and minimal hardware requirements.
A security concern is also real and although this could be solved by encrypting the data, password protection was enough for this system.