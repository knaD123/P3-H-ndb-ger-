\section{Problem statement}
In \cref{problemstatement} the problem statement is presented and this section will answer if the definition was complied with.
The project had the following statement:

\begin{center}
\textit{How can a document version management software be designed and developed to support a firm's management of their handbook documents so it is easily manageable by the administrators, and also accessible for readers and writers?}
\end{center}

Furthermore the problem statement highlighted three aspects of the problem statement that were worth noting:

\begin{itemize}
	\item
		\textit{Document version management:}
		The software must be able to manage documents and their versions.
	\item
		\textit{Design and development:}
		There are two main processes that the project group must undergo through the project which are the design processes and the development of the system based on the design.
	\item
		\textit{Administrators, readers, and writers:}
		These are the different roles associated with the system that need to be considered both during the design and development process.

\end{itemize}

Firstly there has been developed and implemented a software that is able to manage documents.
This has been concluded as the implementation lives up to most of the system definition (see \cref{sec:dissystemdef}) and both \textit{must have} and \textit{should have} priorities from the MoSCoW prioritations (see \cref{sec:disdesignrequirements}).
Living up to these ensures that it is possible to manage handbook documents and their versions with the implemented system.

As mention in relation to design and development the project included both processes as described in the iterations in \cref{sec:workflow}.
Here the system design and implementation underwent several iterations based upon interviews with Ipsen in each iteration.
%The handbook management was developed and designed by conducting several semi-structured interviews with Ipsen.
These interviews, in conjunction with a look at other handbook management systems, gave a fundamental understanding of what was needed in such system.
This understanding was then used to develope classes and functions as described in OOA\&D\ cite{Rod-Aalborg}.

%The interface was moulded through several iterations, each time changed based on feedback from the interviews and in this way coming closer to the optimal graphical user interface for Ipsen.
%section{User rights}
%User rights was implemented in the system so users not managing the handbook do not have the power to change, delete or otherwise administrate the system.

As for the administrator, reader, and writer roles; these were iterated upon and implemented through every step of the development.
To design and implement these into the system the role pattern were utilized.
This stems from the OOA\&D where the role pattern is described as having role as an abstract class and having other classes inherit from role.
The way it is implemented does not provide a dynamic role pattern, but if you follow OOA\&D you can use the pattern where each user has an attribute that is activated, that is, each role is a superstructure of underlying roles.
This would provide a dynamic way of managing roles, where if you could simply change the roles by changing an attribute.

Although this implementation does not have the dynamic power of the OOA\&D role pattern, it still provides the desired access rights in the system.
