\section{System design}

In \cref{sec:systemdesign} the component design was presented and was implemented as shown in \cref{sec:codeexamples}.
Though despite the layout of the written report the reality is not as linear as simply designing the system and then implementing it afterwards.
This is conveyed through the several iteration the system went through described in \cref{sec:workflow}.

The design of the system and the implementation of it happened synchronously throughout the iterations as the system requirements and the usability of the system was discovered through these iterations.
Because of this the design of the system was discovered through several iterations whereas the final iteration of the design is presented in \cref{sec:systemdesign}.
Discussion of the iterative model has occured in \cref{sec:iterativModel}.
This section seeks to add additional advantages and disadvantages that the project group identified throughout the process.

One of the advantages are that the design and the implementation evolves at the same time, and there exist no disparrancy between either one.
As a result the implementation closely reflects the system design as these were written synchronously.

One of the disadvantages is that the system and its features were at time implemented without the existense of the design.
Instead the system were at times designed based on the implementation.
As written in \cref{sec:dissystemdef} all of the system requirements were fulfilled and the implmementation did not suffer in that sense.
Though the problem occured during the development of the system's interface.
As illustrated in \cref{tab:utest2} in \cref{fourthtest} there were problems with 
