\section{Fourth iteration}
\subsection{Simple timeline}\label{sec:3Iteration-timeline}
After the updates made at the end of iteration three in preparations of the intended last usability test the workflow turned into a complete evaluation activity \cref{fig:DEBModel}.
During this processs the focus has been on evaluating the results gotten from Ipsen and her family at the usability test, and to evaluate and update all the analysis' represented in this report.
The focus has furthermore been upon looking into discussing e.g. the results and the process as well as looking into future work. 

\subsection{Elements to take notice of}
\subsubsection*{Some conclusions on last prototype}
At the last usability test, see \cref{fourthtest}, with Ipsen one really interesting issue arouse in the form of a important misconception.
Where the development had focused on saving the necessary information concerning the different documents and files into the system; as it was first understood that it was only needed for the information to be accessible.
It became clear at this meeting that what really was necessary was for the system to automatically update the header of a given file  with the relevant information.
This is yet another example on how designing and developing interactive systems may bring problems with it, because of misconceptions.
Again, it is with this situation confirmed how working iterativly is useful as changing the analysis would just be another iteration compared to the waterfall method where adding changes at the point of testing could turn out to be costly, see \cref{sec:Bug}.

Another good example of how designing and developing interactive systems is an ever changing process, can be found in the same usability test.
Here Ipsen suddenly, through interacting with the prototype, became aware of a new need.
This being the need to have a list of what a user may not have read yet of those documents they were supposed to.
