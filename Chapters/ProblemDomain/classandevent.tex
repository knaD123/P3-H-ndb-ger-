\todo[inline]{Astrid: Sørg for at reactivate er i orden (Iteration 4 - elemetns to take notice of)}
This chapter covers the problem domain analysis based on data gathered in the initial interviews with Ipsen.
It includes the selection of classes and events, a rundown of the structure between classes, and the description of the behavioral patterns through statechart diagrams.

\section{Classes and events}\label{sec:ClassEvent}
The following subsections explore the potential class and event candidates as well as the thought process behind them.
The generated class and event candidates will be systematically selected based upon their relevance and whether they satisfy the system definition.
The process of systematically selecting classes and events is from the Object Oriented Analysis and Design (OOA\&D) method \cite{Rod-Aalborg}.

\subsection{Classes} \label{sec:Classes}
According to  \citep[p.~55]{Rod-Aalborg} a class is defined as,
\begin{defn}\label{defn:class}
	A description of a collection of objects sharing structure, behavioral pattern, and attributes.
\end{defn}

The questions below help to evaluate a class. The more affirmative the answers are the more likely it is that a class is the correct representation. \citep[p.~63]{Rod-Aalborg}
\begin{itemize}
	\item ''Can you identify objects from the class?''
	\item ''Does the class contain unique information?''
	\item ''Does the class encompass multiple objects?''
	\item ''Does the class have a suitable and manageable number of events?''
\end{itemize}

In the following, the classes from \cref{tab:ClassCandidates} are deemed qualified or not qualified through a systematic selection based on the criteria mentioned above.

\begin{table} [H]
	\centering
\begin{tabular}{l l}
	\hline
	Classes & Comment\\
	\hline
	\texttt{Company} & The companies that need to be managed \\
	\texttt{Handbook} & The newest version of all the documents\\
	\texttt{Document} & The documents in the handbook\\
	\texttt{User}& Users with either reader, writer or administrator role\\
	\texttt{Version} & Document’s version and the validity period\\
	\texttt{Changelog} & Descriptions of changes for each version\\
	\texttt{Responsibility} & Different user access levels to the handbook\\
	\texttt{Approval} & Keeps track of who has to approve specific
	versions\\
	\texttt{TOC} & The Table of Contents of the documents in the handbook\\
	\texttt{Notification} & Sends notice to relevant users when a new version is submitted\\
	\texttt{Read Status} & Keeps track of who has read which document and version\\
	\texttt{Supplier} & Supplier who has to be approved \\
	\hline
\end{tabular}
\caption{Class candidates}\label{tab:ClassCandidates}
\end{table}

\subsubsection{Qualified classes}

The \texttt{Company} keeps track of the handbook and the different suppliers that would be present in the system.
It is therefore a qualified class acting as glue between the handbook and supplier sub problems, discussed in \cref{sec:classdiagram}.

The \texttt{Handbook} keeps track of the documents and is an important part of the problem domain and system definition.

The \texttt{Document} works as a descriptor for the various versions in an item-descriptor pattern.

The \texttt{Version} encapsulates the version data in the documents, and allows the document to have multiple instances of them.
They are the items og the item-descriptor pattern.

The \texttt{User} is qualified because there would be different types of users with different access levels in the system.

The \texttt{Approval} is vital to substantiate who has to or already has approved each version and at what date it was approved.

The \texttt{Read Status} is qualified since there is a need to keep track of which users has read what versions and when.
This is useful when the company has to report a total number of employees that have read the current version of the document.

The \texttt{Supplier} is qualified seeing that the administrator needs to keep a record of all documents related to approval of suppliers.

\subsubsection*{Not qualified classes}

The \texttt{Changelog} simply consists of a text string from version to version.
Because of this, making it an actual object would be overcomplicating the model, and it has therefore been preserved as an attribute of the \texttt{Version} class.
An argument could be made that the \texttt{Changelog} class should be qualified if the changelog in its entirety were accessed more often than it is, see \cref{sec:access-archive}.

The \texttt{Responsibility} is not qualified because the different levels of access and permissions that are present in the \texttt{User} class.

The \texttt{TOC} is a property of the \texttt{Handbook} class and does not contain unique information.
The purpose with \texttt{TOC} is to create an overview of the valid documents in the handbook.

The \texttt{Notification} is more an event, see \cref{sec:Events}, rather than a class.
The functionality in the \texttt{Notification} class is to assign users to specific documents therefore the name is modified to \texttt{Department} instead.

\subsubsection*{Summary}
Hereby, the Qualified classes are:
\begin{multicols}{4}
	\begin{itemize}
	\item \textit{Company}
	\item \textit{Handbook}
	\item \textit{Document}
	\item \textit{Version}
	\item \textit{User}
	\item \textit{Approval}
	\item \textit{Department}
	\item \textit{Read Status}
	\item \textit{Supplier}
	\end{itemize}
\end{multicols}

\subsection{Events} \label{sec:Events}
According to \citep[p.~53]{Rod-Aalborg} an event is defined as,
\begin{defn}\label{defn:Event}
An instantaneous incident involving one or more objects.
\end{defn}

Once again, the questions below help to evaluate an event if those are affirmative \citep[p.~65]{Rod-Aalborg}.
\begin{itemize}
	\item ''Is the event instantaneous?''
	\item ''Is the event atomic?''
	\item ''Can the event be identified when it happens?''
\end{itemize}

The event being instantaneous refers to there not being any time frame which can be described as "while the event is happening". 
There is only a before and after.
The event being atomic means that the event is indivisible. 
It cannot be divided into smaller events.
The event being identifiable refers to it being obvious when it happens. The system will be able to know of the event happening, either due to the event happening within the system, or because it is informed of the event happening by an user.

In the rest of the section the generated events in \cref{tab:EventCandidates} will be reviewed and deemed either qualified or not qualified regarding the systematical selection based on the criteria written above.

\begin{table}[H]
	\centering
	\begin{tabular}{m{4.2 cm} l}
		\hline
		Event & Comment\\
		\hline
		\textbf{Document added} & A document is created\\
		\textbf{Document archived} & A document where version is archived instead of updated\\
		\textbf{Document reactivated} & Archived document is made accessible in the handbook again\\
		\textbf{Document deleted} & Wrong document is created, and is still without version\\
		\textbf{Version approved} & A version has been marked as approved\\
		\textbf{New version released} &  New version becomes available in the handbook\\
		\textbf{Version archived} & The version is no longer in the handbook\\
		\textbf{Version marked \newline as read} & User has verified that the newest version has been read\\
		\textbf{User added} & An administrator created a user\\
		\textbf{User updated} & An administrator or user updated the user's information\\
		\textbf{User deactivated} & An administrator removed a user's access for the handbook\\
		\textbf{User reactivated} & Deactivated user is granted access to the handbook again\\
		\textbf{User deleted} & User deleted by an administrator\\
		\textbf{Supplier added} & New supplier has been registered\\
		\textbf{Supplier updated} & Supplier's information has been updated\\
		\textbf{Supplier deleted} & Supplier has been removed from the system\\
		\textbf{Supplier approved} & A supplier has been approved based on attached documents\\
		\textbf{Approval requested} & New version is created and the approval request is sent\\
		\textbf{Approval confirmed} & Approval is confirmed by an approver\\
		\textbf{Approval denied} & Approval is rejected by an approver\\
		\textbf{Department added} & Department is created\\
		\textbf{Department deleted} & Department is removed from system\\
		\textbf{Department's users\newline updated} & Associating/removing a user to/from a department\\
		\textbf{Department's\newline documents updated} & Associating/removing a document to/from a department.\\
		\textbf{Department notified\newline user} & Associated user received a notification of new version released\\
		\textbf{Document printed} & Document is printed\\
		\hline
	\end{tabular}
	\caption{Event candidates}\label{tab:EventCandidates}
\end{table}

\subsubsection*{Qualified events}

The \textit{Document added} event is qualified due to the creation of a document object being necessary for a document to exist, which in turn is necessary for tracking documents and versions.

The \textit{Document archived} event is qualified because it is an important functionality to archive expired documents for a number of years according to the ISO 9001 document management requirements described in \cref{sec:ISOstandards}.

\textit{Document reactivated} event is qualified since it must also be able to reactivate the document and make it available in the handbook again in the case where a document and ID combination should become relevant again.

The \textit{Document deleted} is qualified for the reason that if an administrator made a mistake and added a wrong document then the delete option should be available.

The \textit{Version approved} event is qualified since it is important for the system to register and keep track of when a version is through the approval process and whether it should be accessible in the handbook.

The \textit{Version marked as read} event is qualified because it is important for the company to register if employees have read specific documents and versions and are up to date with the current standards.

The \textit{Version archived} event is qualified for the same reason as \textit{document archived} event.

The \textit{User added}, \textit{user updated} and \textit{user deleted} events are all qualified as an administrator needs to manage the company's user list.

The \textit{Supplier added}, \textit{supplier updated},  \textit{supplier deleted} and \textit{supplier approved} events are all qualified since suppliers need to be managed by an administrator whether they are approved or need to be re-approved.

The \textit{Approval requested} event is qualified seeing that it is an important part of the system definition that the system must support an approval process.

The \textit{Department added} and \textit{department deleted} events are qualified due to departments is how the system organizes read statuses and associated documents with users.

The \textit{Department's documents updated} and \textit{department's users updated} events are qualified since it is important to associate a user to a specific department and to associate documents to the relevant departments.

The \textit{Department notified user} event is qualified seeing that the system must be able to send notifications to the correct user from a specific department that a associated document has released a new version.

\subsubsection*{Not qualified events}

The \textit{New version released} event is not qualified seeing that it is a duplicate of the \textit{version approved} event.

The reason for the \textit{User deactivated} event was so that the system would always be able to report who has read a specific version, as is the requirements to such a management system.
The principle was that a deactivated user object would only contain user's name while the rest information is erased.
Instead this will be the effect of the \textit{user deleted} event and thus \textit{user deactivated} is obsolete.

The \textit{User reactivated} event is not qualified, since a user cannot be deactivated but is instead deleted. In the rare occasion a company wishes to reemploy a person they have to create a new user in the system again for this person.

The \textit{Approval denied (Version)} event is not qualified as Ipsen does not desire this functionality since the approvers themselves will contact Ipsen directly and afterwards a new approval request is made and the previous approval request will be deleted.

The \textit{Approval confirmed} event is not qualified since its purpose is the same as \textit{version approved} event.

The \textit{Document printed} event is not qualified since the problem domain does not include controlling of printed documents.

\subsubsection*{Summary}
Hereby, the Qualified events are:
\begin{multicols}{2}
	\begin{itemize}
	\item Document added
	\item Document archived
	\item Document reactivated
	\item Document deleted
	\item Version approved
	\item Version marked as read
	\item Version archived
	\item User added
	\item User updated
	\item User deleted
	\item Supplier added
	\item Supplier updated
	\item Supplier deleted
	\item Supplier approved
	\item Approval requested (Version)
	\item Approval deleted (Version)
	\item Department added
	\item Department deleted
	\item Department's users updated
	\item Department's documents updated
	\item Department notified user
	\end{itemize}
\end{multicols}

\section{Event table}\label{sec:EventTable}
The result of the selected qualified events and classes from \cref{sec:Classes,sec:Events} is an event table representing their relations, see \cref{fig:eventtable}.
The symbols used in the table is to be understood such that; $+$ is an event that can occur zero or one time, and $\star$ is an event which can occur zero or multiple times.
A class is associated to an event if the event affects the class.
A class is not associated with an event if the class is the cause of or otherwise triggers the event.

\begin{table}[H]
	\begin{center}
		\begin{tabular}{|l|c|c|c|c|c|c|c|c|c|}
			\hline
			& \rotatebox{90}{Company} &  \rotatebox{90}{Handbook} & \rotatebox{90}{Document} & \rotatebox{90}{Version} & \rotatebox{90}{User} & \rotatebox{90}{Approval} & \rotatebox{90}{Department} & \rotatebox{90}{Read Status} & \rotatebox{90}{Supplier}\\
			\hline
			Document added &  & $\star$ & $+$ &  &  &  &  &  & \\
			\hline
			Document archived &  & $\star$ & $\star$ &  &  &  &  &  & \\
			Document reactivated &  & $\star$ & $\star$ &  &  &  &  &  & \\
			\hline
			Document deleted &  & $\star$ & $+$ &  &  &  &  &  & \\
			\hline
			Version approved &  & $\star$ & $\star$ & $+$ &  & $+$ & $\star$ &  & \\
			\hline
			Version marked as read &  &  &  & $+$ & $\star$ &  &  & $\star$ & \\
			\hline
			Version Archived &  & $\star$ & $\star$ & $+$ &  &  &  &  & \\
			\hline
			User added &  & $\star$ &  &  & $+$ &  &  &  & \\
			\hline
			User updated &  &  &  &  & $\star$ &  &  &  & \\
			\hline
			User deleted &  & $\star$ &  &  & $+$ &  &  &  & \\
			\hline
			Supplier added & $\star$ &  &  &  &  &  &  &  & $+$\\
			\hline
			Supplier updated &  &  &  &  &  &  &  &  & $\star$\\
			\hline
			Supplier deleted & $\star$ &  &  &  &  &  &  &  & $+$\\
			\hline
			Supplier approved & $\star$ &  &  &  &  & $\star$ &  &  & $\star$\\
			\hline
			Approval requested &  &  &  & $+$ & $\star$ & $+$ &  &  & \\
			\hline
			Approval deleted &  &  &  & $+$ &  & $+$ &  &  & \\
			\hline
			Department added  &  &  &  &  &  &  & $+$ &  & \\
			\hline
			Department's users updated &  &  &  &  & $\star$ &  & $+$ &  & \\
			\hline
			Department's documents updated &  &  & $\star$ &  &  &  & $+$ &  & \\
			\hline
			Department deleted &  &  &  &  &  &  & $+$ &  & \\
			\hline
			Department notified user &  &  &  &  & $\star$ &  & $+$ &  & \\
			\hline
		\end{tabular}
	\end{center}
	\caption{Event table}\label{fig:eventtable}
\end{table}

Looking at the event table above it can be seen that \textit{Document added} and \textit{Document deleted} both can occur zero or multiple times for \textit{Handbook} and zero or one time for \textit{Document}.
This is because, during the lifetime of a \textit{Handbook} it is possible to add and delete documents to it as many times as desired, but a single \textit{Document} can only be added or deleted once.
Keep in mind, that it is not possible to trigger the \textit{document deleted} event if the \textit{Handbook} does not contain any documents.
A less trivial event from the event table could be the \textit{approval requested} event.
Here it can be seen that this event occurs zero or one time for a \textit{Version} and an \textit{Approval}, but it can occur zero or multiple times for \textit{User} because multiple users can be assigned to approve a new version.

This event table is useful when designing the system, as it displays all the events that the system needs to keep track of, and what objects should be included in each event.
