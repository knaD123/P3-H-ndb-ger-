\section{Statechart diagrams} \label{sec:statechart}
Statechart diagrams are a useful way to get an overview over the lifetime of objects in the problem domain.
This section describes the state chart diagram of objects whose behaivour is both interesting and unique to the development project.

\subsection{Approval Request}
The approval request is one of the most central concepts in the application.
When someone wants to create a new version of a document, they instantiate an approval request, which has a version object attached.
The change will have to be approved by all the users set as ''approvers'', before it can be added to the handbook.
When the change is approved, it will append the version to the document object.
For the sake of simplicity, there is no option to actually deny a proposed version.
Any issues with the document are, as mentioned in \cref{sec:richpictures}, meant to be handled outside of the system instead. If it is decided that a proposed version is unsalvagable, it can be deleted.

\begin{figure}[H]
	\centering
\begin{tikzpicture}[align=center, scale=0.85, transform shape]
	\node[n] (start) {};
	\node[predefined] (await) [below of=start, node distance=1.7cm] {Awaiting approval};
	\node(end)[n, right of=await, node distance=8cm] {};
	\node(End)[c,fit=(end)] at (end) {};
	\draw[arrow] (start.south) -- node[right]{Version\\uploaded} (await.north);
	\path[arrow] (await) edge [loop below, out=355, in=340, looseness=4] node [below]{Approve} (await);
	\path[arrow] (await) edge [loop below, out=185, in=205, looseness=3] node [below] {Replace the file} (await);
	\draw[arrow] (await.8) -- node[above]{All reviewers\\have approved} (End.north west);
	\draw[arrow] (await.east) -- node[below]{Delete} (End.west);
\end{tikzpicture}
\end{figure}

\subsection{Department}
A department is an organizational unit, used to administrate which users are attached to which documents.
As such, you can create a department, add users to it, and add documents to it as well.
It is also possible to delete users and documents from the department, as long as there are any attached to the department.
One of the most important functions of the department is the abillity for it to notify all the attached users of a new version of a document.
This was also the reason the Department class was first decided upon, see \cref{sec:Classes}

\begin{figure}[H]
	\centering
\begin{tikzpicture}[align=center, scale=0.85, transform shape]
	\node[n] (start) {};
	\node[predefined] (await) [below of=start, node distance=1.5cm] {Existing};
	\node(end)[n, right of=await, node distance=4cm] {};
	\node(End)[c,fit=(end)] at (end) {};
	\draw[arrow] (start.south) --node[right] {Create} (await.north);
	\path[arrow] (await) edge [loop below, out=270, in=230, looseness=4] node[below, xshift=-0.2cm] {Add/remove\\user} (await);
	\path[arrow] (await) edge [loop below, out=350, in=325, looseness=3] node[auto] {Add/remove\\ Document} (await);
	\path[arrow] (await) edge [loop left, looseness=4] node[auto] {Notify} (await);
	\draw[arrow] (await.east) --node[above] {Delete} (End.west);
en Notify er ikke helt god -astrid
\end{tikzpicture}
\end{figure} 

\subsection{Version}
A version is another central concept in the application.
As mentioned before, the handbook consists of documents, and each document can have multiple versions.
And thus, after creation, version objects are supposed to be quite immutable once created, as to preserve the history of individual documents.
Due to the immutable structure of the versions, the actual versions never change.

\begin{figure}[H]
	\centering
\begin{tikzpicture}[align=center, scale=0.85, transform shape]
	\node[n] (start) {};
	\node[predefined] (await) [below of=start, node distance=1.5cm] {Awaiting approval};
	\node[predefined] (approved) [right of=await, node distance=5cm] {In handbook};
	\node(end)[n, right of=approved, node distance=4cm] {};
	\node(End)[c,fit=(end)] at (end) {};	
	\draw[arrow] (start.south) --node[right] {Created} (await.north);
	\draw[arrow] (await.east) --node[above] {Change\\approved} (approved.west);
	\draw[arrow] (approved.east) --node[below] {Version\\archived} (End.west);
\end{tikzpicture}
\end{figure}