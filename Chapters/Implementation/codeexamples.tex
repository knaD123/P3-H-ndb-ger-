\section{Code examples}
In this section examples of the code will be presented.
As written in \cref{sec:systemdesign} the system is based on the ASP.NET Core MVC architecture including EF Core.
How the system utilizes the MVC architecture and the EF Core framework will be shown in the following code examples.

Firstly the MVC implementation is explored with the Document class.
How the Document object is being treated throughout the system will be presented throught the model, view, and controller components.
Furthermore it will be shown how the document data will be stored in the database with the repository pattern and EF Core.
Here it will be shown in part how a new document is added into the system throughout the components.

\subsection{Model}

In the model component it Document object is defined.
As it is primarily the document object that will be highlighted in the code examples, how the document object is presented in the code snippet below:
\\
\begin{lstlisting}
[Key]
public int ID { get; set; } //for the database

public virtual Chapter Chapter { get; set; } //first part of document ID

public int ChapterNumber
{
	get
	{
		return Chapter != null ? Chapter.Number : -1;
	}
}

[Range(1, Int32.MaxValue)]
[Display(Name = "Section")]
public int SectionNumber { get; set; } //second part of Document ID

[Required]
public string Title { get; set; } // TODO: Unique title of document

public virtual ICollection<Version> Versions { get; set; } //List of all past versions of the document (the "archive")
public virtual IEnumerable<DocumentDepartment> DocumentDepartments { get; set; }

public bool Archived { get; set; }

public Document()
{
	Archived = false;
}
\end{lstlisting}

The document object includes an ID, chapter, chapter number, section number, section number, versions, and whether or not it has been archived as seen in the code snippet above.
The \texttt{key} attribute on line 1 specifies that the \texttt{int ID} is a key that the database can use to identify the specific document.
Likewise the \texttt{Range} attribute specifies that the section number is included in a an \texttt{enumerable} class consisting of integers from 1 to infinity.
And the \texttt{Display} attribute on line 15 specifies that the section number should be displayed with the name \textit{section}.
This \texttt{display} attribute is used with ASP.NET Core and is used to modify the text on the screen when a controller is rendered to the view.

In the code snippet below it is shown the logic of how a new version of a document is added.
\\

\begin{lstlisting}
public void AddVersion(Version version)
{
	version.ValidFromDate = DateTime.Now;
	version.Approved = true;
	Versions.Add(version);
	int nVersions = Versions.Count;
	var oldVersion = Versions.ElementAtOrDefault(nVersions - 1);

	if (nVersions >= 2 && oldVersion != null)
	{
		oldVersion.ValidUntilDate = DateTime.Now;
	}
	foreach (DocumentDepartment dd in DocumentDepartments)
	{
		dd.Department.Notify("The document " + Title + " has been updated", "http://localhost:5000/document/" + Chapter.Number.ToString() + "." + SectionNumber.ToString());
	}
}
\end{lstlisting}

As is seen on line 3 and 4 when a new version of a document is added it receives a new date from which is it valid and that the \texttt{Approved} bool is set to \texttt{true}.
On line 5 the new version is added to the list of versions.
The control sequence from line 9 - 12 ensures that the previous version's date is set to the day that it has been replaced by the new version.
When a new version has been added all of the affected department in the system will get notified as seen in the loop from line 13 - 16.

\subsection{Controller}

In the code examples below it is shown how a \texttt{HttpGet} and \texttt{HttpPost} methods are implemented in the controller component.
Firstly it is shown how the document object is arranged so that it can be rendered to the view through the \texttt{HttpGet} method.
\\

\begin{lstlisting}
[HttpGet("")]
[HttpGet("document/")]
public async Task<IActionResult> Index()
{
	var user = await this._userManager.GetUserAsync(HttpContext.User);
	var documentIndex = new ViewModels.DocumentIndex();
	foreach (var ud in user.UserDepartments)
	{
		foreach (var dd in ud.Department.DocumentDepartments)
		{
			if (!dd.Document.Archived) documentIndex.AssignedDocuments.Add(dd.Document);
		}
	}
	List<Document> documentList = await this._documentRepository.ListNonArchivedAsync();
	foreach (var d in documentIndex.AssignedDocuments)
	{
		documentList.Remove(d);
	}

	foreach (var d in documentList)
	{
		if (!documentIndex.UnassignedDocuments.ContainsKey(d.Chapter))
		documentIndex.UnassignedDocuments[d.Chapter] = new List<Document>();
		documentIndex.UnassignedDocuments[d.Chapter].Add(d);
	}

	@ViewData["Title"] = "Table of Contents";

	return View(documentIndex);
}
\end{lstlisting}

The \texttt{HttpGet} method means that the view component retrieves data from the controller.
Something to take not of is the \texttt{async Task<IActionResult>} on line 3.
The \texttt{Task} class represents a single operation \cite{microsoft}.
This operation does not return a value and usually executes asynchronously.
The \texttt{IActionResult} result method ensures that the controller returns a model to the view.

From line 5 - 13 it is determined which documents that the current user is assigned to.
This is done by first retrieving the current user, which is defined through \texttt{HttpContext.User}, by finding a match in the list of users found in \texttt{_userManager}.
Afterwards program sorts through the departments that the user is a part of to find which documents that the user is assigned to.
This is done through the loop from line 9 - 12.

After the user's assigned documents has been sorted a list of all the non-archived documents will occur in the view.
This is done through the two loops from line 14 - 25.
All of the non-archived documents are returned to the view through \texttt{var documentIndex} on line 29.

In the code snippet below it is shown what happens when the user adds a new document or versionto the system through the \texttt{HttpPost} method.
\\
\begin{lstlisting}
[HttpPost("document/add/")]
public async Task<IActionResult> Add(Document document,
	int chapterNumber,
	IFormFile versionFile,
	IFormFile workingFile,
	string requireApprovalCheck,
	string approvers)
{
	bool requireApproval = (requireApprovalCheck == "on");
	@ViewData["Title"] = "New document";
	@ViewBag.Chapters = await _chapterRepository.GetChapters();

	if (ModelState.IsValid)
	{
		document = this._documentRepository.ToProxy(document);
		var chapter = await this._chapterRepository.GetChapter(chapterNumber);

		if (chapter == null)
		{
			ModelState.AddModelError(string.Empty, "Problem happend getting the chapter.");

			return View(document);
		}

		document.Chapter = chapter;

		try
		{
			document = await this._documentRepository.AddAsync(document);
		}
		catch (DocumentAddException e)
		{
			ModelState.AddModelError(string.Empty, e.ModelError);
			return View(document);
		}
		if (versionFile != null)
		{
			OBHandbooks.Models.Version version = new OBHandbooks.Models.Version();
			version.SetDocument(document);
			version = await this._versionRepository.AddAsync(version);

			version.VersionFile = new HandbookFile(versionFile);
			if (workingFile != null) version.WorkingFile = new HandbookFile(workingFile);

			if (requireApproval)
			{
				HandbookApproval approval = await CreateApproval(
				approvers,
				await this._userManager.GetUserAsync(HttpContext.User));
				approval.DocumentVersion = version;
				version.DateSubmittedToApproval = DateTime.Now;
				document.Versions.Add(version);
				approval.CheckForApproval();
				await this._approvalRepository.UpdateAsync(approval);
			}
			else
			{
				document.AddVersion(version);
			}
			await this._documentRepository.UpdateAsync(document);
			await this._versionRepository.UpdateAsync(version);
		}
		return RedirectToAction(nameof(Index));

	}

	return View(document);
}
\end{lstlisting}

As the \texttt{HttpGet} method retrieves data from the controller, the \texttt{HttpPost} method ensures that new data is submitted into the controller and into the components further down the layers.

The information from the view is transfered into the controller method through the parameters shown from line 2 - 7.
When the document is posted to the controller the \texttt{Add} method ensures that the new document version needs to be approved line 9.
Afterwards the actual document receives its chapter number on line 16 and is saved into the document repository on line 29 through the \texttt{AddAsync} method.

The new version of the document is stored into the repository from line 36 - 62.
Here the new version is saved into the version repository on line 40 with the \texttt{AddAsync} method.
Notice that the \texttt{_documentRepository.AddAsync} and \texttt{_versionRepository.AddAsync} are similarly named, but saves a document and a version respectively.

The new document version is not yet valid to the readers as it needs to be approved.
From line 45 - 55 the program creates an approval object or objects where it is determined which users need to approve the new version.
If the new version does not need an approval, it is simply saved into the repositories on line 60 - 61.

\subsection{View}

\begin{lstlisting}
@foreach (KeyValuePair<Models.Chapter, List<Models.Document>> kvp in Model.UnassignedDocuments) {
<button class="btn btn-outline-dark text-left pl-4 mt-1" onclick="listExpand()" style="width: 100%; margin-top: 10px;" type="button" data-toggle="collapse" data-target="#collapseChapter-@kvp.Key.Number" aria-expanded="false" aria-controls="collapseExample">
<div class="row">
	@kvp.Key.Number | @kvp.Key.Name
	<i class="fas fa-chevron-down ml-auto mr-4 arrow-toggle"></i>
</div>
</button>
<div class="collapse card" id="collapseChapter-@kvp.Key.Number">
	<div class="card-body" style="padding: 0.1rem">
		<table class="table table-striped table-sm">
			<thead class="border font-weight-bold">
				<tr>
					<td>ID</td>
					<td>Title</td>
					<td>Version</td>
					<td>Date</td>
					<td>Read status</td>
					@if (User.IsInRole("Administrator"))
					{
						<td></td>
					}
				</tr>
			</thead>
		<tbody>
			@await Html.PartialAsync("_DocumentListBody", @kvp.Value)
		</tbody>
		</table>
	</div>
</div>
}

\end{lstlisting}
This is the view that represents a list of assigned documents. This will render to a button, which when clicked expands a list of documetns in the chapter the button represents.
It's written in the razor templating langugae, which is HTML, but with the added feature of allowing the documents to be dynamic when prefacing a statement with the \texttt{@} character.
An interesting thing to note is on line 25, where the statement \texttt{@await Html.PartialAsync("_DocumentListBody", @kvp.Value)} is executed.
This statement asks the razor templating engine to render the body of a list of documetns, with the argument that is the list of documents in the current chapter.

\subsection{Repositories}

\begin{lstlisting}
public async Task<Document> AddAsync(Document document)
{
	if (this._dbContext.documents.Where(
			d => d.Chapter == document.Chapter && d.SectionNumber == document.SectionNumber
			).SingleOrDefault() != null)
	{
		throw new DocumentAddException("Chapter/section number is not unique");
	}

	if (this._dbContext.documents.Where(d => d.Title == document.Title).FirstOrDefault() != null)
	{
		throw new DocumentAddException("Title is not unique");
	}
	var ret = this._dbContext.documents.Add(document); //documents er en attribute af typen DbSet
	await this._dbContext.SaveChangesAsync();
	return ret.Entity;
}
\end{lstlisting}
This is a method in the \texttt{DocumentRepository}, which allows for adding documents to the database.
It first performs to checks, to see if the database already contains a document.
If that is the case, it throws the custom \texttt{DocumentAddException}.
Otherwise, it adds the document to the database, and returns the tracked database entry.
The reason that this is important, and that one can't just use the \texttt{Document} object supplied as an argument, is due to the use of lazy-loading proxies in the application.
Lazy-loading proxies funcion so that when objects have relationships with other object, these are implemented as virtual properties.

\todo[inline]{Skal vi forklare hvad lazy loading betyder, eller er det basic nok til vi kan slippe?}

In the case of the \texttt{AddAsync} method, this means that you can construct a raw \texttt{Document} object, pass it to the method, and then get a proxy object in return.
This allows for assigning to the virtual members, which is a functionality that is heavily used in the controllers.
\subsection{Entity Framework Core}
Entity Framework Core is used to communicate with the database, which on it's own is not object-oriented.
For more information on EF Core see \cref{sec:efcore}.

Postgresql was chosen as the database for the project.
The decision was purely based on prior experience with setting up the database.
Postgresql is often better suited for more advanced applications, [CITATION NEEDED] but as this application mostly used CRUD operations with few to no complicated queries, this was hardly relevant.
\subsubsection{HandbookContext}
The \texttt{HandbookContext} was the class that handled communication between the database and the application.
It subclasses IdentityDbContext, which in turn subclasses a normal DbContext.
\lstinputlisting[caption={HandbookContext}, language={[Sharp]C}]{OBHandbooks/OBHandbooks/Infrastructure/HandbookContext.cs}
It consists of a large amount of properties of the type \texttt{DbSet<MODEL>}, which each represents a collection of entities to be stored in the database. The overwritten \texttt{OnModelCreating} function also supplies extra information about the relationsips in the database, such as unique properties, foreign keys, or delete behaviours. For example, on line 31, it can be seen that when a \texttt{HandbookApproval} is to be deleted, that should also result in any associated \texttt{Version} objects being deleted.

\subsubsection{Repository Pattern}
In this project the repository pattern was used.
It abstracts communication with the database, allowing for easily replacing it if necessary, and mocking it during unit tests.

An example of the interface to a a repository is as follows
\lstinputlisting[language={[Sharp]C}]{OBHandbooks/OBHandbooks/Repositories/IApprovalRepository.cs}
The interface is then in turn implemented by a class which is added to the controllers using dependency injection.
As an example, the document controller is instantiated with each of the repositories it uses.
\lstinputlisting[firstline=29, lastline=44, language={[Sharp]{C}}]{OBHandbooks/OBHandbooks/Controllers/DocumentController.cs}
When unit testing the controllers, the Moq framework is used to create mock objects of the interfaces, enabling the tests to focus on the controllers.

\subsubsection{Database Design}
\todo[inline]{Andreas: Gør den der ting jeg bad dig om}
The database design should hopefully look a lot like the class diagram.

