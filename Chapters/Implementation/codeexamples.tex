\section{Code examples}\label{sec:codeexamples}
In this section examples of the code will be presented.
As written in \cref{sec:systemdesign} the system is based on the ASP.NET Core MVC architecture including EF Core.
How the system utilizes the MVC architecture and the EF Core framework will be shown in the following code examples.

First, the MVC implementation is explored with the \texttt{Document} class.
How the \texttt{Document} object is being treated throughout the system will be explored through the model, view, and controller components.
Furthermore it will be shown how the document data will be stored in the database with the repository pattern and EF Core.
Here it will be shown in part how a new document is added into the system through the different components.

\subsection{Model}

In the model component the \texttt{Document} object is defined.
It is primarily the document object that will be highlighted in the code examples.
How the document object is presented is shown in the code snippet below:
%Taniya: synes at vi enten skal kalde det code snippet eller listing
% Henrik: Synes den sidste sætning er lidt kryptisk?
% Anja: Har omformuleret
% Henrik: Maaske importer code-snippets ligesom Rasmus har gjort? (ikke vigtigt)
\\
\begin{lstlisting}[caption={Document Model: Document object}, label={lst:doc}]
[Key]
public int ID { get; set; } //for the database

public virtual Chapter Chapter { get; set; } //first part of document ID

public int ChapterNumber
{
	get
	{
		return Chapter != null ? Chapter.Number : -1;
	}
}

[Range(1, Int32.MaxValue)]
[Display(Name = "Section")]
public int SectionNumber { get; set; } //second part of Document ID

[Required]
public string Title { get; set; } // TODO: Unique title of document

public virtual ICollection<Version> Versions { get; set; } //List of all past versions of the document (the "archive")
public virtual IEnumerable<DocumentDepartment> DocumentDepartments { get; set; }

public bool Archived { get; set; }

public Document()
{
	Archived = false;
}
\end{lstlisting}

%Taniya: måske fjern TODO kommentar i listning?
%Taniya: Logikken i linje 10? 


The document object includes an ID, chapter, chapter number, section number, versions, and whether or not it has been archived as seen in the code snippet above, \cref{lst:doc}.
The \texttt{key} attribute on line 1 specifies that the \texttt{int ID} is a primary key that the database can use to identify the specific document.
Likewise, the \texttt{Range} attribute specifies that the section number is included in a an \texttt{Enumerable} class consisting of integers from 1 to the max value of a 32 bit \texttt{Integer}.
And the \texttt{Display} attribute on line 15 specifies that the section number should be displayed with the name \textit{section}.
This \texttt{display} attribute is used with ASP.NET Core and is used to modify the text when the model is used in a view.
%Taniya: måske også forklar noget med hvorfor vi bruger virtual Chapter Chapter og  IEnumerable<DocumentDepartment> DocumentDepartments? 

In the code snippet below, \cref{lst:addversion}, it is shown the logic of how a new version of a document is added.
\\

\begin{lstlisting}[caption={Document Model: AddVersion}, label={lst:addversion}]
public void AddVersion(Version version)
{
	version.ValidFromDate = DateTime.Now;
	version.Approved = true;
	Versions.Add(version);
	int nVersions = Versions.Count;
	var oldVersion = Versions.ElementAtOrDefault(nVersions - 1);

	if (nVersions >= 2 && oldVersion != null)
	{
		oldVersion.ValidUntilDate = DateTime.Now;
	}
	foreach (DocumentDepartment dd in DocumentDepartments)
	{
		dd.Department.Notify("The document " + Title + " has been updated", "http://localhost:5000/document/" + Chapter.Number.ToString() + "." + SectionNumber.ToString());
	}
}
\end{lstlisting}

When a new version of a document is added the version's ValidFromDate is set to the current date and the \texttt{Approved} bool is set to \texttt{true} as seen on line 3 and 4.
% Henrik: Maaske i stedet for "it receives a new date ...", saa skriv "the version's ValidFromDate property is set to the current date"?
On line 5 the new version is added to the list of versions.
The control sequence from line 9 - 12 ensures that the previous version's ValidUntilDate is set to the current date.
% Taniya: kan vi skrive The Selection Control Structure (fagterm) istedet for The control sequence?
% Henrik: Same thing her - Masske "the previous version's ValidUntilDate property is set to ..."
When a new version has been added all of the affected department in the system will get notified as seen in the loop from line 13 - 16.
%Taniya: Iteration Control Structure - foreach loop 

\subsection{Controller}

In \cref{lst:doccontroller,lst:doccontrolleradd} it is shown how a \texttt{HttpGet} and \texttt{HttpPost} methods are implemented in the controller component.
Firstly, it is shown how the document object is arranged so that it can be passed to the view through the \texttt{HttpGet} method.
\\

\begin{lstlisting}[caption={Document Controller: Index}, label={lst:doccontroller}]
[HttpGet("")]
[HttpGet("document/")]
public async Task<IActionResult> Index()
{
	var user = await this._userManager.GetUserAsync(HttpContext.User);
	var documentIndex = new ViewModels.DocumentIndex();
	foreach (var ud in user.UserDepartments)
	{
		foreach (var dd in ud.Department.DocumentDepartments)
		{
			if (!dd.Document.Archived) documentIndex.AssignedDocuments.Add(dd.Document);
		}
	}
	List<Document> documentList = await this._documentRepository.ListNonArchivedAsync();
	foreach (var d in documentIndex.AssignedDocuments)
	{
		documentList.Remove(d);
	}

	foreach (var d in documentList)
	{
		if (!documentIndex.UnassignedDocuments.ContainsKey(d.Chapter))
		documentIndex.UnassignedDocuments[d.Chapter] = new List<Document>();
		documentIndex.UnassignedDocuments[d.Chapter].Add(d);
	}

	@ViewData["Title"] = "Table of Contents";

	return View(documentIndex);
}
\end{lstlisting}

The \texttt{HttpGet} method means that the method only can accept \texttt{Get} requests.
Something to take note of in \cref{lst:doccontroller} is the \texttt{async Task<IActionResult>} on line 3.
The \texttt{Task} class represents a single operation \cite{microsoft}.
This operation does not return a value and usually executes asynchronously.
The \texttt{IActionResult} method means that it returns a result.
% Taniya: return hva for noget result?
%The \texttt{IActionResult} result method ensures that the controller returns a model to the view.
% Henrik: Jeg er ikke enig mht det der staar om HttpGet og IActionResult
% HttpGet sørger bare for at denne action method kun kan blive tilgaaet af en GET request
% IActionResult garanterer ikke at controlleren returns en model til viewet
% IActionResult er en return-type ting, der fortæller at denne action method returnerer et result.

From line 5-13 it is determined which documents that the current user is assigned to.
This is done by first retrieving the current user, which is defined through \texttt{HttpContext.User}, by finding a match in the list of users found in \texttt{_userManager}.
Afterwards, the program iterates through the departments a user is a part of to find which documents he is assigned to.
This is done through the loop on line 9 - 12.

After the user is assigned documents has been sorted a list of all the non-archived documents will occur in the view.
%Anna: er forirreet over hvad der står og fåortstå det ikke rigtigt
This is done through the two foreach loops from line 14 - 25.
All of the non-archived documents are returned to the view through \texttt{var documentIndex} on line 29.
% Henrik: Synes starten er lidt kryptisk + jeg forstaar ikke hvor det her "sorting" kommer fra?

In \cref{lst:doccontrolleradd} it is shown what happens when the user adds a new document or version to the system through the \texttt{Add} method.
% Henrik: Der bliver ikke tilføjet noget gennem HttpPost method, det sker gennem Add action method
% HttpPost fortæller bare, at denne action method kun kan tilgaaes med en POST request
\\
\begin{lstlisting}[caption={Document Controller: Add}, label={lst:doccontrolleradd}]
[HttpPost("document/add/")]
public async Task<IActionResult> Add(Document document,
	int chapterNumber,
	IFormFile versionFile,
	IFormFile workingFile,
	string requireApprovalCheck,
	string approvers)
{
	bool requireApproval = (requireApprovalCheck == "on");
	@ViewData["Title"] = "New document";
	@ViewBag.Chapters = await _chapterRepository.GetChapters();

	if (ModelState.IsValid)
	{
		document = this._documentRepository.ToProxy(document);
		var chapter = await this._chapterRepository.GetChapter(chapterNumber);

		if (chapter == null)
		{
			ModelState.AddModelError(string.Empty, "Problem happend getting the chapter.");

			return View(document);
		}

		document.Chapter = chapter;

		try
		{
			document = await this._documentRepository.AddAsync(document);
		}
		catch (DocumentAddException e)
		{
			ModelState.AddModelError(string.Empty, e.ModelError);
			return View(document);
		}
		if (versionFile != null)
		{
			OBHandbooks.Models.Version version = new OBHandbooks.Models.Version();
			version.SetDocument(document);
			version = await this._versionRepository.AddAsync(version);

			version.VersionFile = new HandbookFile(versionFile);
			if (workingFile != null) version.WorkingFile = new HandbookFile(workingFile);

			if (requireApproval)
			{
				HandbookApproval approval = await CreateApproval(
				approvers,
				await this._userManager.GetUserAsync(HttpContext.User));
				approval.DocumentVersion = version;
				version.DateSubmittedToApproval = DateTime.Now;
				document.Versions.Add(version);
				approval.CheckForApproval();
				await this._approvalRepository.UpdateAsync(approval);
			}
			else
			{
				document.AddVersion(version);
			}
			await this._documentRepository.UpdateAsync(document);
			await this._versionRepository.UpdateAsync(version);
		}
		return RedirectToAction(nameof(Index));

	}

	return View(document);
}
\end{lstlisting}
%Anna: Evt sætte denne specifikke listing i appendiks? (er emget lang for meget lidt der fås ud af den)

The \texttt{HttpPost} method ensures that it can only receive \texttt{Post} requests.

%Taniya: måske forklar hvaf IFormfile er for noget?
The information from the view is transfered into the controller method through the parameters shown from line 2-7.
When the document is posted to the controller the \texttt{Add} method ensures that the new document version needs to be approved, see line 9.
% Henrik: Paa linje 9 bliver der bare lavet en variable - true hvis strengen er "on", ellers false
Afterwards the actual document receives its chapter number on line 16 and is saved into the document repository on line 29 through the \texttt{AddAsync} method.
% Henrik: Den faar faktisk sit chapter number paa linje 25 - linje 9 henter bare et chapter fra databasen
A more detailed explanation of the \texttt{AddAsync} method can be seen in \cref{lst:docrepadd}.

The new version of the document is stored into the repository on line 36-62.
Here the new version is saved into the version repository on line 40 with the \texttt{AddAsync} method.
Notice that the \texttt{_documentRepository.AddAsync} and \texttt{_versionRepository.AddAsync} are similarly named, but respectively saves a document and a version.

The new document version is not yet valid to the readers as it needs to be approved.
From line 45-55 the program creates an approval object or objects where it is determined which users need to approve the new version.
If the new version does not need an approval, it is simply saved into the repositories on line 60-61.

\subsection{View}

In \cref{lst:docview} it is shown how the \texttt{view} component utilizes data given from the \texttt{controller} component.

\begin{lstlisting}[caption={Document View: Index}, label={lst:docview}]
@foreach (KeyValuePair<Models.Chapter, List<Models.Document>> kvp in Model.UnassignedDocuments) {
<button class="btn btn-outline-dark text-left pl-4 mt-1" onclick="listExpand()" style="width: 100%; margin-top: 10px;" type="button" data-toggle="collapse" data-target="#collapseChapter-@kvp.Key.Number" aria-expanded="false" aria-controls="collapseExample">
<div class="row">
	@kvp.Key.Number | @kvp.Key.Name
	<i class="fas fa-chevron-down ml-auto mr-4 arrow-toggle"></i>
</div>
</button>
<div class="collapse card" id="collapseChapter-@kvp.Key.Number">
	<div class="card-body" style="padding: 0.1rem">
		<table class="table table-striped table-sm">
			<thead class="border font-weight-bold">
				<tr>
					<td>ID</td>
					<td>Title</td>
					<td>Version</td>
					<td>Date</td>
					<td>Read status</td>
					@if (User.IsInRole("Administrator"))
					{
						<td></td>
					}
				</tr>
			</thead>
		<tbody>
			@await Html.PartialAsync("_DocumentListBody", @kvp.Value)
		</tbody>
		</table>
	</div>
</div>
}

\end{lstlisting}

\cref{lst:docview} is the view that represents a list of assigned documents.
This will render to a button, which when clicked expands a list of documents in a given chapter.
It is written in the razor templating langugae, which is HTML with the added feature of allowing the documents to be dynamic when prefacing a statement with the \texttt{@} character.
% Henrik: @ betyder faktisk "her starter noget C# kode", saa tænker at maaske flet ind at razor gør det muligt at skrive C# kode inde i ens HTML?
%Anna: vi skal i hvert fald være præcise så kunne være en ide
An interesting thing to note is line 25, where the statement \texttt{@await Html.PartialAsync("_DocumentListBody", @kvp.Value)} is executed.
This statement asks the razor templating engine to render the body of a list of documents, with the argument that is the list of documents in the current chapter.
% Henrik: Første argument er navnet paa det view den skal render (filnavn)
%Anna: kan godt være det bare er mig, men synes den sidste sætning er kringlet og er ikke iskker på jeg forstå den ordentligt

\subsection{Repositories}

In the \texttt{repositories} component the data is processed.
In \cref{lst:docrepadd} it is shown how a document is added to the database.

\begin{lstlisting}[caption={Document repository: AddAsync}, label={lst:docrepadd}]
public async Task<Document> AddAsync(Document document)
{
	if (this._dbContext.documents.Where(
			d => d.Chapter == document.Chapter && d.SectionNumber == document.SectionNumber
			).SingleOrDefault() != null)
	{
		throw new DocumentAddException("Chapter/section number is not unique");
	}

	if (this._dbContext.documents.Where(d => d.Title == document.Title).FirstOrDefault() != null)
	{
		throw new DocumentAddException("Title is not unique");
	}
	var ret = this._dbContext.documents.Add(document); //documents er en attribute af typen DbSet
	await this._dbContext.SaveChangesAsync();
	return ret.Entity;
}
\end{lstlisting}


This is a method in the \texttt{DocumentRepository}, which allows for adding documents to the database.
%Anna: what is mentioned. man kan ikke starte en sætning med This uden at det vser tilabge til noget, så hvad er det der menes?
It first performs two checks, to see if the database already contains a document.
If that is the case, it throws the custom \texttt{DocumentAddException}.
Otherwise, it adds the document to the database, and returns the tracked database entry.
The reason why this is important, and that one can't just use the \texttt{Document} object supplied as an argument, is due to the use of lazy-loading proxies in the application.
Lazy-loading proxies funcions in such a way that when objects have relationships with other objects, these are implemented as virtual properties.
% Taniya: menes der function (funcion)?  Og hvilke linje er det vi taler om ? 

\todo[inline]{Skal vi forklare hvad lazy loading betyder, eller er det basic nok til vi kan slippe?}
%Anna: tænker i hvert fald det er fint at have lidt om det
% Taniya: må gerne forklar det kort, hvis muligt

In the case of the \texttt{AddAsync} method, this means that you can construct a raw \texttt{Document} object, pass it to the method, and then get a proxy object in return.
This allows for assigning to the virtual members, which is a functionality that is heavily used in the controllers.
%Anna: vil mene afsnittet ovenfor hører sammen med den ovenover.
\subsection{Entity Framework Core}
Entity Framework Core is used to communicate with the database, which on it's own is not object-oriented.
For more information on EF Core see \cref{sec:efcore}.

Postgresql was chosen as the database for the project.
The decision was purely based on prior experience with setting up the database.
Postgresql is often better suited for more advanced applications, [CITATION NEEDED] but as this application mostly used CRUD operations with few to no complicated queries, this was hardly relevant.
\subsubsection{HandbookContext}
The \texttt{HandbookContext} was the class that handled communication between the database and the application.
It subclasses IdentityDbContext, which in turn subclasses a normal DbContext.
\lstinputlisting[caption={HandbookContext}, language={[Sharp]C}]{OBHandbooks/OBHandbooks/Infrastructure/HandbookContext.cs}
It consists of a large amount of properties of the type \texttt{DbSet<MODEL>}, which each represents a collection of entities to be stored in the database. The overwritten \texttt{OnModelCreating} function also supplies extra information about the relationsips in the database, such as unique properties, foreign keys, or delete behaviours. For example, on line 31, it can be seen that when a \texttt{HandbookApproval} is to be deleted, that should also result in any associated \texttt{Version} objects being deleted.

\subsubsection{Repository Pattern}
In this project the repository pattern was used.
It abstracts communication with the database, allowing for easily replacing it if necessary, and mocking it during unit tests.

An example of the interface to a a repository is as follows
\lstinputlisting[language={[Sharp]C}]{OBHandbooks/OBHandbooks/Repositories/IApprovalRepository.cs}
The interface is then in turn implemented by a class which is added to the controllers using dependency injection.
As an example, the document controller is instantiated with each of the repositories it uses.
\lstinputlisting[firstline=29, lastline=44, language={[Sharp]{C}}]{OBHandbooks/OBHandbooks/Controllers/DocumentController.cs}
When unit testing the controllers, the Moq framework is used to create mock objects of the interfaces, enabling the tests to focus on the controllers.

\subsubsection{Database Design}
\todo[inline]{Andreas: Gør den der ting jeg bad dig om}
The database design should hopefully look a lot like the class diagram.

