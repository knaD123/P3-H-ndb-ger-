\section{Code examples}\label{sec:codeexamples}
In this section examples of the code will be presented.
As written in \cref{sec:systemdesign} the system is based on the ASP.NET Core MVC architecture including EF Core.
How the system utilizes the MVC architecture and the EF Core framework will be shown in the following code examples.

First, the MVC implementation is explored with the \texttt{Document} class.
How the \texttt{Document} object behaves in the system will be explored through the model, view, and controller components.
Furthermore it will be shown how the document data will be created, and then stored in the database with the repository pattern and EF Core.

\subsection{Model}

In the model component the \texttt{Document} object is defined.
It is primarily the document object that will be highlighted in the code examples.
How the document object is presented is shown in the code listing below:

\newpage
\begin{lstlisting}[caption={Document Model: Document object}, label={lst:doc}]
[Key]
public int ID { get; set; } //for the database

public virtual Chapter Chapter { get; set; } //first part of document ID

public int ChapterNumber
{
	get
	{
		return Chapter != null ? Chapter.Number : -1;
	}
}

[Range(1, Int32.MaxValue)]
[Display(Name = "Section")]
public int SectionNumber { get; set; } //second part of Document ID

[Required]
public string Title { get; set; }

public virtual ICollection<Version> Versions { get; set; } //List of all past versions of the document (the "archive")
public virtual IEnumerable<DocumentDepartment> DocumentDepartments { get; set; }

public bool Archived { get; set; }

public Document()
{
	Archived = false;
}
\end{lstlisting}

The document object includes an ID, chapter, section number, versions, and whether or not it has been archived as seen in the code listing above, \cref{lst:doc}.
The \texttt{Key} attribute on line 1 specifies that the \texttt{int ID} is a primary key that the database can use to identify the specific document.
Likewise, the \texttt{Range} attribute specifies that the section number is restricted to a range from 1 to the max value of a 32 bit integer.
And the \texttt{Display} attribute on line 15 specifies that the section number should be displayed with the name \textit{section}.
This \texttt{Display} attribute is used to modify the text when the model displayed in a view.

In \cref{lst:addversion} it is shown the logic of how a new version of a document is added to a document after it has been approved.

When a new version of a document is added the version's \texttt{ValidFromDate} is set to the current date and the \texttt{Approved} bool is set to \texttt{true} as seen on line 3 and 4.
On line 5 the new version is added to the list of versions.
The selection control structure from line 9 - 12 ensures that the previous version's ValidUntilDate is set to the current date.
When a new version has been added all of the affected departments in the system will get notified as seen in the iteration control structure from line 13 - 16.
\\

\begin{lstlisting}[caption={Document Model: AddVersion}, label={lst:addversion}]
public void AddVersion(Version version)
{
	version.ValidFromDate = DateTime.Now;
	version.Approved = true;
	Versions.Add(version);
	int nVersions = Versions.Count;
	var oldVersion = Versions.ElementAtOrDefault(nVersions - 1);

	if (nVersions >= 2 && oldVersion != null)
	{
		oldVersion.ValidUntilDate = DateTime.Now;
	}
	foreach (DocumentDepartment dd in DocumentDepartments)
	{
		dd.Department.Notify("The document " + Title + " has been updated", "http://localhost:5000/document/" + Chapter.Number.ToString() + "." + SectionNumber.ToString());
	}
}
\end{lstlisting}

\subsection{Controller}

In \cref{lst:doccontroller,lst:doccontrolleradd} it is shown how a \texttt{HttpGet} and \texttt{HttpPost} methods are implemented in the controller component.
Firstly, it is shown how the document object is arranged so that it can be passed to the view through the \texttt{HttpGet} method.
\\

\begin{lstlisting}[caption={Document Controller: Index}, label={lst:doccontroller}]
[HttpGet("")]
[HttpGet("document/")]
public async Task<IActionResult> Index()
{
	var user = await this._userManager.GetUserAsync(HttpContext.User);
	var documentIndex = new ViewModels.DocumentIndex();
	foreach (var ud in user.UserDepartments)
	{
		foreach (var dd in ud.Department.DocumentDepartments)
		{
			if (!dd.Document.Archived) documentIndex.AssignedDocuments.Add(dd.Document);
		}
	}
	List<Document> documentList = await this._documentRepository.ListNonArchivedAsync();
	foreach (var d in documentIndex.AssignedDocuments)
	{
		documentList.Remove(d);
	}

	foreach (var d in documentList)
	{
		if (!documentIndex.UnassignedDocuments.ContainsKey(d.Chapter))
		documentIndex.UnassignedDocuments[d.Chapter] = new List<Document>();
		documentIndex.UnassignedDocuments[d.Chapter].Add(d);
	}

	@ViewData["Title"] = "Table of Contents";

	return View(documentIndex);
}
\end{lstlisting}

The \texttt{HttpGet} method means that the method only can be accessed by a \texttt{GET} requests.
Something to take note of in \cref{lst:doccontroller} is the \texttt{async Task<IActionResult>} on line 3.
The \texttt{Task} class represents a single operation \cite{microsoft} performed asynchronously.
The \texttt{IActionResult} type parameter means that upon completion, it returns an \texttt{IActionResult} object.

From line 5-13 it is determined which documents that the current user is assigned to.
This is done by retreiving the current user from the \texttt{_userManager}, by finding a user in the list that matches the one found in the \texttt{HttpContext}.
Afterwards, the program iterates through the departments which the user is a part of to find which documents they are assigned to.
This is done through the iterative control structure on line 9 - 12.

After the list of documents the user has not been assigned to have been organized into chapters, it will be added to the viewmodel.
This is done through the two foreach loops from line 14 - 25.
All of the non-archived documents are returned to the view through \texttt{var documentIndex} on line 29.

The \texttt{HttpPost} method defines this method as the handler for POST requests to this URL.

The information from the view is transfered into the controller method through the parameters shown from line 2-7.
When the document is posted to the controller the \texttt{Add} allows the user to decide whether or not the initial version needs to go through the approval process. This is decided through the \texttt{requireApproval} variable.
Afterwards the actual document receives its chapter number on line 24 and is saved into the document repository on line 29 through the \texttt{AddAsync} method.
This method returns a proxy object as described further down in \cref{sec:repositories}.
A more detailed explanation of the \texttt{AddAsync} method can be seen in \cref{lst:docrepadd}.

In \cref{lst:doccontrolleradd} it is shown what happens when the user adds a new document or version to the system through the \texttt{Add} method.
\\
\lstinputlisting[caption={Document Controller: Add}, label={lst:doccontrolleradd}, firstline=152, lastline=237]{OBHandbooks/OBHandbooks/Controllers/DocumentController.cs}


The \texttt{IFormFile} parameter represents a file uploaded through a form. It includes data such as extension, content type, and name.
If the variable \texttt{versionFile} is not null, an initial version is added.
The new version of the document is stored into the repository on line 36-62.
Here the new version is saved into the version repository on line 40 with the \texttt{AddAsync} method.
Notice that the \texttt{_documentRepository.AddAsync} and \texttt{_versionRepository.AddAsync} are similarly named, but respectively saves a document and a version.

If the approval checkbox is enabled, the new document version is not yet valid to the readers.
From line 45-55 the program creates an approval object or objects where it is determined which users need to approve the new version.
If the new version does not need an approval, it is simply saved into the repositories on line 60-61.

\subsection{View}

In \cref{lst:docview} it is shown how the view component utilizes data given from the controller component.

\cref{lst:docview} is the view that represents a list of assigned documents.
This will render to a button, which when clicked expands a list of documents in a given chapter.
It is written in the razor templating language, which is HTML with the added feature of allowing the documents to be dynamic when prefacing a statement with the \texttt{@} character.
The code preceded by the \texttt{@} character is in C\#.
An interesting thing to note is line 25, where the statement \texttt{@await Html.PartialAsync("_DocumentListBody", @kvp.Value)} is executed.
This statement asks the razor templating engine to render the body of a list of documents, with the model in variable \texttt{kvp.Value}.

\begin{lstlisting}[caption={Document View: Index}, label={lst:docview}]
@foreach (KeyValuePair<Models.Chapter, List<Models.Document>> kvp in Model.UnassignedDocuments) {
<button class="btn btn-outline-dark text-left pl-4 mt-1" onclick="listExpand()" style="width: 100%; margin-top: 10px;" type="button" data-toggle="collapse" data-target="#collapseChapter-@kvp.Key.Number" aria-expanded="false" aria-controls="collapseExample">
<div class="row">
	@kvp.Key.Number | @kvp.Key.Name
	<i class="fas fa-chevron-down ml-auto mr-4 arrow-toggle"></i>
</div>
</button>
<div class="collapse card" id="collapseChapter-@kvp.Key.Number">
	<div class="card-body" style="padding: 0.1rem">
		<table class="table table-striped table-sm">
			<thead class="border font-weight-bold">
				<tr>
					<td>ID</td>
					<td>Title</td>
					<td>Version</td>
					<td>Date</td>
					<td>Read status</td>
					@if (User.IsInRole("Administrator"))
					{
						<td></td>
					}
				</tr>
			</thead>
		<tbody>
			@await Html.PartialAsync("_DocumentListBody", @kvp.Value)
		</tbody>
		</table>
	</div>
</div>
}

\end{lstlisting}

\subsection{Repositories}\label{sec:repositories}

In the \texttt{repositories} component the data is processed.
In \cref{lst:docrepadd} it is shown how a document is added to the database.

\begin{lstlisting}[caption={Document repository: AddAsync}, label={lst:docrepadd}]
public async Task<Document> AddAsync(Document document)
{
	if (this._dbContext.documents.Where(
			d => d.Chapter == document.Chapter && d.SectionNumber == document.SectionNumber
			).SingleOrDefault() != null)
	{
		throw new DocumentAddException("Chapter/section number is not unique");
	}

	if (this._dbContext.documents.Where(d => d.Title == document.Title).FirstOrDefault() != null)
	{
		throw new DocumentAddException("Title is not unique");
	}
	var ret = this._dbContext.documents.Add(document); //documents er en attribute af typen DbSet
	await this._dbContext.SaveChangesAsync();
	return ret.Entity;
}
\end{lstlisting}


\Cref{lst:docrepadd} displays a method in the \texttt{DocumentRepository}, which allows for adding documents to the database.
The method first performs two checks, to see if the database already contains a document.
If that is the case, it throws the custom \texttt{DocumentAddException}.
Otherwise, it adds the document to the database, and returns the tracked database entry.
The reason why this is important, and that one cannot just use the \texttt{Document} object supplied as an argument, is due to the use of lazy-loading proxies in the application.
Lazy loading refers to a technique where the resources are only loaded when actually used, contrasted to eager loading, when they are initialized as soon as possible.
Lazy-loading proxies function in such a way that when objects have relationships with other objects, these are implemented as virtual properties.
In the case of the \texttt{AddAsync} method, this means that you can construct a raw \texttt{Document} object, pass it to the method, and then get a proxy object in return.
This allows for assigning to the virtual members, which is a functionality that is heavily used in the controllers.
\subsection{Entity Framework Core}
Entity Framework Core is used to communicate with the database, which on it's own is not object-oriented.
For more information on EF Core see \cref{sec:efcore}.

Postgresql was chosen as the database for the project.
The decision was purely based on prior experience with setting up the database.
Postgresql is often better suited for more advanced applications, [CITATION NEEDED] but as this application mostly used CRUD operations with few to no complicated queries, this was hardly relevant.
\subsubsection{HandbookContext}
The \texttt{HandbookContext} was the class that handled communication between the database and the application.
It subclasses IdentityDbContext, which in turn subclasses a normal DbContext.
\lstinputlisting[caption={HandbookContext}, language={[Sharp]C}]{OBHandbooks/OBHandbooks/Infrastructure/HandbookContext.cs}
It consists of a large amount of properties of the type \texttt{DbSet<MODEL>}, which each represents a collection of entities to be stored in the database. The overwritten \texttt{OnModelCreating} function also supplies extra information about the relationsips in the database, such as unique properties, foreign keys, or delete behaviours. For example, on line 31, it can be seen that when a \texttt{HandbookApproval} is to be deleted, that should also result in any associated \texttt{Version} objects being deleted.

\subsubsection{Repository Pattern}
In this project the repository pattern was used.
It abstracts communication with the database, allowing for easily replacing it if necessary, and mocking it during unit tests.

An example of the interface to a a repository is as follows
\lstinputlisting[language={[Sharp]C}]{OBHandbooks/OBHandbooks/Repositories/IApprovalRepository.cs}
The interface is then in turn implemented by a class which is added to the controllers using dependency injection.
As an example, the document controller is instantiated with each of the repositories it uses.
\lstinputlisting[firstline=29, lastline=44, language={[Sharp]{C}}]{OBHandbooks/OBHandbooks/Controllers/DocumentController.cs}
When unit testing the controllers, the Moq framework is used to create mock objects of the interfaces, enabling the tests to focus on the controllers.

\subsubsection{Database Design}
\todo[inline]{Andreas: Gør den der ting jeg bad dig om}
The database design should hopefully look a lot like the class diagram.

