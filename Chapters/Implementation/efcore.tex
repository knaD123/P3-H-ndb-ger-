\subsection{Entity Framework Core}
EF Core is used to communicate with the postgresql database, which on its own is not object-oriented.
For more information on EF Core see \cref{sec:efcore}.
Postgresql was chosen as the database for this development project.
The decision was purely based on prior experiences with setting up such a database.

\subsubsection{HandbookContext}
The \texttt{HandbookContext} is the class that handled communication between the database and the application.
It subclasses IdentityDbContext, which in turn subclasses a normal DbContext.
\lstinputlisting[caption={The properties of HandbookContext}, language={[Sharp]C}, firstline=12, lastline=20, label={lst:contextprop}, autogobble=true, tabsize=2]{OBHandbooks/OBHandbooks/Infrastructure/HandbookContext.cs}
\lstinputlisting[caption={First part of the OnModelCreating mehtod}, language={[Sharp]C}, firstline=27, lastline=39, label={lst:contextcreate}, tabsize=4, autogobble=true]{OBHandbooks/OBHandbooks/Infrastructure/HandbookContext.cs}
It consists of a large amount of properties of the type \texttt{DbSet<MODEL>}, which each represents a collection of entities to be stored in the database.
The overwritten \texttt{OnModelCreating} function also supplies extra information about the relationships in the database, such as unique properties, foreign keys, or delete behaviours.
For example, on line 8, it can be seen that when a \texttt{HandbookApproval} is to be deleted it should also result in any associated \texttt{Version} objects being deleted.
