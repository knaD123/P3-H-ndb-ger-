\subsection{Entity Framework Core}
Entity Framework Core is used to communicate with the database, which on it's own is not object-oriented.
For more information on EF Core see \cref{sec:efcore}.

Postgresql was chosen as the database for the project.
The decision was purely based on prior experience with setting up the database.
Postgresql is often better suited for more advanced applications, [CITATION NEEDED] but as this application mostly used CRUD operations with few to no complicated queries, this was hardly relevant.
\subsubsection{HandbookContext}
The \texttt{HandbookContext} was the class that handled communication between the database and the application.
It subclasses IdentityDbContext, which in turn subclasses a normal DbContext.
\lstinputlisting[caption={HandbookContext}, language={[Sharp]C}]{OBHandbooks/OBHandbooks/Infrastructure/HandbookContext.cs}
It consists of a large amount of properties of the type \texttt{DbSet<MODEL>}, which each represents a collection of entities to be stored in the database. The overwritten \texttt{OnModelCreating} function also supplies extra information about the relationsips in the database, such as unique properties, foreign keys, or delete behaviours. For example, on line 31, it can be seen that when a \texttt{HandbookApproval} is to be deleted, that should also result in any associated \texttt{Version} objects being deleted.

\subsubsection{Repository Pattern}
In this project the repository pattern was used.
It abstracts communication with the database, allowing for easily replacing it if necessary, and mocking it during unit tests.

An example of the interface to a a repository is as follows
\lstinputlisting[language={[Sharp]C}]{OBHandbooks/OBHandbooks/Repositories/IApprovalRepository.cs}
The interface is then in turn implemented by a class which is added to the controllers using dependency injection.
As an example, the document controller is instantiated with each of the repositories it uses.
\lstinputlisting[firstline=29, lastline=44, language={[Sharp]{C}}]{OBHandbooks/OBHandbooks/Controllers/DocumentController.cs}
When unit testing the controllers, the Moq framework is used to create mock objects of the interfaces, enabling the tests to focus on the controllers.

\subsubsection{Database Design}
\todo[inline]{Andreas: Gør den der ting jeg bad dig om}
The database design should hopefully look a lot like the class diagram.
