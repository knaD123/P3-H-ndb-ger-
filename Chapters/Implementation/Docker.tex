\section{Docker}
\todo[inline]{Find en eller anden bog om docker, og sæt den som kilde til det hele}
The project was developed using Docker.
Docker is a containerization technology.
Much like virtualization, it allows for having isolated processes, but virtualization goes much further than containerization.
While a virtual machine simulates an entire machine, a container shares the same kernel as the host system, which is far lighter on resources


Using Docker allowed for easier management of dependencies, control over resetting databases, and guarantees of identical runtime environments.
The Dockerfile looks as follows
\lstinputlisting{OBHandbooks/Dockerfile}
This Dockerfile defined how the isolated userspace in which the ASP.NET Core project would run would be set up.
As can be seen, most of it consists of running commands in the container, until line 13-18 which set some environment variables, copies the code into the container, and builds it.

The Docker image generated from this Docker file was then used in a \texttt{Docker-compose.yml} file, which is used to create deployment structures consisting of multiple containers.
\lstinputlisting{OBHandbooks/docker-compose.yml}
The docker file is developed in the YAML language.
Here, two Docker containers are actually used.
The first container, defined on line three, uses the image described previously.
It opens up ports 5000 and 5001 to the host machine, and mounts the current directory in the \texttt{/Handbooks} folder on the container.
Finally, it defines the command to run inside the container as it is started, and links it to the second container, the database container.

The second container used is way more simple
It simply uses the standard \texttt{postgres} image, and sets the password to be \texttt{mysecretpassword} via an environment variable.
This password should naturally not be used in production.
